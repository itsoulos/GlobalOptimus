%% LyX 2.4.3 created this file.  For more info, see https://www.lyx.org/.
%% Do not edit unless you really know what you are doing.
\documentclass[journal,article,submit,pdftex,moreauthors]{Definitions/mdpi}
\usepackage{textcomp}
\usepackage[utf8]{inputenc}
\usepackage{color}
\usepackage{float}
\usepackage{url}
\usepackage{graphicx}

\makeatletter

%%%%%%%%%%%%%%%%%%%%%%%%%%%%%% LyX specific LaTeX commands.

\Title{Combining constructed artificial neural networks with parameter constraint
techniques to achieve better generalization properties}

\TitleCitation{Combining constructed artificial neural networks with parameter constraint
techniques to achieve better generalization properties}

\Author{Ioannis G. Tsoulos$^{1,*}$, Vasileios Charilogis$^{2}$, Dimitrios
Tsalikakis$^{3}$}

\AuthorNames{Ioannis G. Tsoulos, Vasileios Charilogis, Dimitrios Tsalikakis}

\AuthorCitation{Tsoulos, I.G.; Charilogis, V.; Tsalikakis D.}


\address{$^{1}$\quad{}Department of Informatics and Telecommunications,
University of Ioannina, Greece;itsoulos@uoi.gr\\
$^{2}$\quad{}Department of Informatics and Telecommunications, University
of Ioannina, Greece; v.charilog@uoi.gr\\
$^{3}\quad$Department of Engineering Informatics and Telecommunications,
University of Western Macedonia, 50100 Kozani, Greece; tsalikakis@gmail.com}


\corres{Correspondence: itsoulos@uoi.gr}


\abstract{This study presents a novel hybrid approach combining grammatical
evolution with constrained genetic algorithms to overcome key limitations
in automated neural network design. The proposed method addresses
two critical challenges: the tendency of grammatical evolution to
converge to suboptimal architectures due to local optima, and the
common overfitting problems in evolved networks. Our solution employs
grammatical evolution for initial architecture generation while implementing
a specialized genetic algorithm that simultaneously optimizes network
parameters within dynamically adjusted bounds. The genetic component
incorporates innovative penalty mechanisms in its fitness function
to control neuron activation patterns and prevent overfitting. Comprehensive
testing across 53 diverse datasets shows our method achieves superior
performance compared to traditional optimization techniques, with
an average classification error of 21.18\% versus 36.45\% for ADAM,
while maintaining better generalization capabilities. The constrained
optimization approach proves particularly effective in preventing
premature convergence, and the penalty system successfully mitigates
overfitting even in complex, high-dimensional problems. Statistical
validation confirms these improvements are significant (p \textless{}
1.1e-08) and consistent across multiple domains including medical
diagnosis, financial prediction, and physical system modeling. This
work provides a robust framework for automated neural network construction
that balances architectural innovation with parameter optimization
while addressing fundamental challenges in evolutionary machine learning.}


\keyword{Grammatical Evolution; Genetic Programming; Neural networks; Local
Optimization}

\newcommand*\LyXZeroWidthSpace{\hspace{0pt}}
\DeclareTextSymbolDefault{\textquotedbl}{T1}
%% Because html converters don't know tabularnewline
\providecommand{\tabularnewline}{\\}
\floatstyle{ruled}
\newfloat{algorithm}{tbp}{loa}
\providecommand{\algorithmname}{Algorithm}
\floatname{algorithm}{\protect\algorithmname}

%%%%%%%%%%%%%%%%%%%%%%%%%%%%%% Textclass specific LaTeX commands.
\newenvironment{lyxcode}
	{\par\begin{list}{}{
		\setlength{\rightmargin}{\leftmargin}
		\setlength{\listparindent}{0pt}% needed for AMS classes
		\raggedright
		\setlength{\itemsep}{0pt}
		\setlength{\parsep}{0pt}
		\normalfont\ttfamily}%
	 \item[]}
	{\end{list}}

%%%%%%%%%%%%%%%%%%%%%%%%%%%%%% User specified LaTeX commands.
%  LaTeX support: latex@mdpi.com 
%  For support, please attach all files needed for compiling as well as the log file, and specify your operating system, LaTeX version, and LaTeX editor.

%=================================================================


% For posting an early version of this manuscript as a preprint, you may use "preprints" as the journal and change "submit" to "accept". The document class line would be, e.g., \documentclass[preprints,article,accept,moreauthors,pdftex]{mdpi}. This is especially recommended for submission to arXiv, where line numbers should be removed before posting. For preprints.org, the editorial staff will make this change immediately prior to posting.

%--------------------
% Class Options:
%--------------------
%----------
% journal
%----------
% Choose between the following MDPI journals:
% acoustics, actuators, addictions, admsci, adolescents, aerospace, agriculture, agriengineering, agronomy, ai, algorithms, allergies, alloys, analytica, animals, antibiotics, antibodies, antioxidants, applbiosci, appliedchem, appliedmath, applmech, applmicrobiol, applnano, applsci, aquacj, architecture, arts, asc, asi, astronomy, atmosphere, atoms, audiolres, automation, axioms, bacteria, batteries, bdcc, behavsci, beverages, biochem, bioengineering, biologics, biology, biomass, biomechanics, biomed, biomedicines, biomedinformatics, biomimetics, biomolecules, biophysica, biosensors, biotech, birds, bloods, blsf, brainsci, breath, buildings, businesses, cancers, carbon, cardiogenetics, catalysts, cells, ceramics, challenges, chemengineering, chemistry, chemosensors, chemproc, children, chips, cimb, civileng, cleantechnol, climate, clinpract, clockssleep, cmd, coasts, coatings, colloids, colorants, commodities, compounds, computation, computers, condensedmatter, conservation, constrmater, cosmetics, covid, crops, cryptography, crystals, csmf, ctn, curroncol, currophthalmol, cyber, dairy, data, dentistry, dermato, dermatopathology, designs, diabetology, diagnostics, dietetics, digital, disabilities, diseases, diversity, dna, drones, dynamics, earth, ebj, ecologies, econometrics, economies, education, ejihpe, electricity, electrochem, electronicmat, electronics, encyclopedia, endocrines, energies, eng, engproc, ent, entomology, entropy, environments, environsciproc, epidemiologia, epigenomes, est, fermentation, fibers, fintech, fire, fishes, fluids, foods, forecasting, forensicsci, forests, foundations, fractalfract, fuels, futureinternet, futureparasites, futurepharmacol, futurephys, futuretransp, galaxies, games, gases, gastroent, gastrointestdisord, gels, genealogy, genes, geographies, geohazards, geomatics, geosciences, geotechnics, geriatrics, hazardousmatters, healthcare, hearts, hemato, heritage, highthroughput, histories, horticulturae, humanities, humans, hydrobiology, hydrogen, hydrology, hygiene, idr, ijerph, ijfs, ijgi, ijms, ijns, ijtm, ijtpp, immuno, informatics, information, infrastructures, inorganics, insects, instruments, inventions, iot, j, jal, jcdd, jcm, jcp, jcs, jdb, jeta, jfb, jfmk, jimaging, jintelligence, jlpea, jmmp, jmp, jmse, jne, jnt, jof, joitmc, jor, journalmedia, jox, jpm, jrfm, jsan, jtaer, jzbg, kidney, kidneydial, knowledge, land, languages, laws, life, liquids, literature, livers, logics, logistics, lubricants, lymphatics, machines, macromol, magnetism, magnetochemistry, make, marinedrugs, materials, materproc, mathematics, mca, measurements, medicina, medicines, medsci, membranes, merits, metabolites, metals, meteorology, methane, metrology, micro, microarrays, microbiolres, micromachines, microorganisms, microplastics, minerals, mining, modelling, molbank, molecules, mps, msf, mti, muscles, nanoenergyadv, nanomanufacturing, nanomaterials, ncrna, network, neuroglia, neurolint, neurosci, nitrogen, notspecified, nri, nursrep, nutraceuticals, nutrients, obesities, oceans, ohbm, onco, oncopathology, optics, oral, organics, organoids, osteology, oxygen, parasites, parasitologia, particles, pathogens, pathophysiology, pediatrrep, pharmaceuticals, pharmaceutics, pharmacoepidemiology, pharmacy, philosophies, photochem, photonics, phycology, physchem, physics, physiologia, plants, plasma, pollutants, polymers, polysaccharides, poultry, powders, preprints, proceedings, processes, prosthesis, proteomes, psf, psych, psychiatryint, psychoactives, publications, quantumrep, quaternary, qubs, radiation, reactions, recycling, regeneration, religions, remotesensing, reports, reprodmed, resources, rheumato, risks, robotics, ruminants, safety, sci, scipharm, seeds, sensors, separations, sexes, signals, sinusitis, skins, smartcities, sna, societies, socsci, software, soilsystems, solar, solids, sports, standards, stats, stresses, surfaces, surgeries, suschem, sustainability, symmetry, synbio, systems, taxonomy, technologies, telecom, test, textiles, thalassrep, thermo, tomography, tourismhosp, toxics, toxins, transplantology, transportation, traumacare, traumas, tropicalmed, universe, urbansci, uro, vaccines, vehicles, venereology, vetsci, vibration, viruses, vision, waste, water, wem, wevj, wind, women, world, youth, zoonoticdis 

%---------
% article
%---------
% The default type of manuscript is "article", but can be replaced by: 
% abstract, addendum, article, book, bookreview, briefreport, casereport, comment, commentary, communication, conferenceproceedings, correction, conferencereport, entry, expressionofconcern, extendedabstract, datadescriptor, editorial, essay, erratum, hypothesis, interestingimage, obituary, opinion, projectreport, reply, retraction, review, perspective, protocol, shortnote, studyprotocol, systematicreview, supfile, technicalnote, viewpoint, guidelines, registeredreport, tutorial
% supfile = supplementary materials

%----------
% submit
%----------
% The class option "submit" will be changed to "accept" by the Editorial Office when the paper is accepted. This will only make changes to the frontpage (e.g., the logo of the journal will get visible), the headings, and the copyright information. Also, line numbering will be removed. Journal info and pagination for accepted papers will also be assigned by the Editorial Office.

%------------------
% moreauthors
%------------------
% If there is only one author the class option oneauthor should be used. Otherwise use the class option moreauthors.

%---------
% pdftex
%---------
% The option pdftex is for use with pdfLaTeX. If eps figures are used, remove the option pdftex and use LaTeX and dvi2pdf.

%=================================================================
% MDPI internal commands - do not modify
\firstpage{1} 
 
\setcounter{page}{\@firstpage} 

\pubvolume{1}
\issuenum{1}
\articlenumber{0}
\pubyear{2024}
\copyrightyear{2024}
%\externaleditor{Academic Editor: Firstname Lastname} % For journal Automation, please change Academic Editor to "Communicated by"
\datereceived{}
\daterevised{ } % Comment out if no revised date
\dateaccepted{}
\datepublished{}
%\datecorrected{} % Corrected papers include a "Corrected: XXX" date in the original paper.
%\dateretracted{} % Corrected papers include a "Retracted: XXX" date in the original paper.
\hreflink{https://doi.org/} % If needed use \linebreak
%\doinum{}
%------------------------------------------------------------------
% The following line should be uncommented if the LaTeX file is uploaded to arXiv.org
%\pdfoutput=1

%=================================================================
% Add packages and commands here. The following packages are loaded in our class file: fontenc, inputenc, calc, indentfirst, fancyhdr, graphicx, epstopdf, lastpage, ifthen, lineno, float, amsmath, setspace, enumitem, mathpazo, booktabs, titlesec, etoolbox, tabto, xcolor, soul, multirow, microtype, tikz, totcount, changepage, attrib, upgreek, cleveref, amsthm, hyphenat, natbib, hyperref, footmisc, url, geometry, newfloat, caption

%=================================================================
%% Please use the following mathematics environments: Theorem, Lemma, Corollary, Proposition, Characterization, Property, Problem, Example, ExamplesandDefinitions, Hypothesis, Remark, Definition, Notation, Assumption
%% For proofs, please use the proof environment (the amsthm package is loaded by the MDPI class).

%=================================================================
% The fields PACS, MSC, and JEL may be left empty or commented out if not applicable
%\PACS{J0101}
%\MSC{}
%\JEL{}

%%%%%%%%%%%%%%%%%%%%%%%%%%%%%%%%%%%%%%%%%%
% Only for the journal Diversity
%\LSID{\url{http://}}

%%%%%%%%%%%%%%%%%%%%%%%%%%%%%%%%%%%%%%%%%%
% Only for the journal Applied Sciences:
%\featuredapplication{Authors are encouraged to provide a concise description of the specific application or a potential application of the work. This section is not mandatory.}
%%%%%%%%%%%%%%%%%%%%%%%%%%%%%%%%%%%%%%%%%%

%%%%%%%%%%%%%%%%%%%%%%%%%%%%%%%%%%%%%%%%%%
% Only for the journal Data:
%\dataset{DOI number or link to the deposited data set in cases where the data set is published or set to be published separately. If the data set is submitted and will be published as a supplement to this paper in the journal Data, this field will be filled by the editors of the journal. In this case, please make sure to submit the data set as a supplement when entering your manuscript into our manuscript editorial system.}

%\datasetlicense{license under which the data set is made available (CC0, CC-BY, CC-BY-SA, CC-BY-NC, etc.)}

%%%%%%%%%%%%%%%%%%%%%%%%%%%%%%%%%%%%%%%%%%
% Only for the journal Toxins
%\keycontribution{The breakthroughs or highlights of the manuscript. Authors can write one or two sentences to describe the most important part of the paper.}

%%%%%%%%%%%%%%%%%%%%%%%%%%%%%%%%%%%%%%%%%%
% Only for the journal Encyclopedia
%\encyclopediadef{Instead of the abstract}
%\entrylink{The Link to this entry published on the encyclopedia platform.}
%%%%%%%%%%%%%%%%%%%%%%%%%%%%%%%%%%%%%%%%%%

%%%%%%%%%%%%%%%%%%%%%%%%%%%%%%%%%%%%%%%%%%
% Only for the journal Advances in Respiratory Medicine
%\addhighlights{yes}
%\renewcommand{\addhighlights}{%

%\noindent This is an obligatory section in “Advances in Respiratory Medicine”, whose goal is to increase the discoverability and readability of the article via search engines and other scholars. Highlights should not be a copy of the abstract, but a simple text allowing the reader to quickly and simplified find out what the article is about and what can be cited from it. Each of these parts should be devoted up to 2~bullet points.\vspace{3pt}\\
%\textbf{What are the main findings?}
% \begin{itemize}[labelsep=2.5mm,topsep=-3pt]
% \item First bullet.
% \item Second bullet.
% \end{itemize}\vspace{3pt}
%\textbf{What is the implication of the main finding?}
% \begin{itemize}[labelsep=2.5mm,topsep=-3pt]
% \item First bullet.
% \item Second bullet.
% \end{itemize}
%}
%%%%%%%%%%%%%%%%%%%%%%%%%%%%%%%%%%%%%%%%%%

\makeatother

\begin{document}
\maketitle

\section{Introduction}

A basic machine learning technique with a wide range of applications
in data classification and regression problems is artificial neural
networks \citep{nn1,nn2}. Artificial neural networks are parametric
machine learning model, in which learning is achieved by effectively
adjusting their parameters through any optimization technique. The
optimization procedure minimizes the so - called training error of
an artificial neural network and it is defined as: 
\begin{equation}
E\left(N\left(\overrightarrow{x},\overrightarrow{w}\right)\right)=\sum_{i=1}^{M}\left(N\left(\overrightarrow{x}_{i},\overrightarrow{w}\right)-y_{i}\right)^{2}\label{eq:eq1}
\end{equation}
In this equation the function $N\left(\overrightarrow{x},\overrightarrow{w}\right)$
represents the artificial neural network which is applied on a vector
$\overrightarrow{x}$ and the vector $\overrightarrow{w}$ denotes
the parameter vector of the neural network. The set $\left(\overrightarrow{x_{i}},y_{i}\right),\ i=1,...,M$
represents the training set of the objective problem and the values
$y_{i}$ are the expected outputs for each pattern $\overrightarrow{x_{i}}$. 

Artificial neural networks have been applied in a wide series of problems
appeared in real - world problems, such as image processing \citep{nn_image},
time series forecasting \citep{nn_timeseries}, credit card analysis
\citep{nn_credit}, problems derived from physics \citep{nnphysics1,nnphysics2}
etc. Due to the widespread use of these machine learning models, a
number of techniques have been proposed to minimize the equation \ref{eq:eq1},
such as the Back Propagation algorithm \citep{bpnn1,bpnn2}, the RPROP
algorithm \citep{rpropnn-1,rpropnn-2}, the ADAM optimization method
\citep{nn_adam} etc. Also, recently a series of more advanced global
optimization methods have been proposed to tackle the training of
neural networks. Among them one can locate the incorporation of Genetic
Algorithms \citep{geneticnn1}, the usage of the Particle Swarm Optimization
(PSO) method \citep{psonn}, the Simulated Annealing method \citep{nn_siman},
the Differential Evolution technique \citep{weight_de1}, the Artificial
Bee Colony (ABC) method \citep{nn_abc} etc. Furthermore, Sexton et
al suggested the usage of the tabu search algorithm for optimal neural
network training \citep{tabunn}, Zhang et al proposed a hybrid algorithm
that incorporated the PSO method and the Back Propagation algorithm
to efficient train artificial neural networks \citep{nn_hybrid}.
Also, recently Zhao et al introduced a new Cascaded Forward Algorithm
to train artificial neural networks \citep{nn_cascade}. Furthermore,
due to the rapid spread of the use of parallel computing techniques,
a series of computational techniques have emerged that exploit parallel
computing structures for faster training of artificial neural networks
\citep{nn_gpu1,nn_gpu2}.

However, the above techniques, although extremely effective, nevertheless
have a number of problems such as, for example, trapping in local
minima of the error function or the phenomenon of overifitting, where
the artificial neural network exhibits reduced performance when applied
to data that was not present during the training process. The overfitting
problem has been studied by many researchers that have proposed a
series of methods to handle this problem, such as weight sharing \citep{nnsharing1,nnsharing2},
pruning \citep{nnprunning1,nnprunning2}, early stopping \citep{nnearly1,nnearly2},
weight decaying \citep{nndecay1,nndecay2} etc. Also, many researchers
propose as a solution to the above problem the dynamic creation of
the architecture of artificial neural networks using programming techniques.
For example the Genetic Algorithms were proposed to create dynamically
the optimal architecture of neural networks \citep{nn_arch1,nn_arch2}
or the PSO method \citep{nn_arch3}.\textbf{ }Siebel et al, suggested
the usage of evolutionary reinforcement learning for the optimal design
of artificial neural networks \citep{nn_ereinf}. Also, Jaafra et
al provided a review on the usage of Reinforcement learning for neural
architecture search \citep{nn_reinf}. In the same direction of research,
Pham et al proposed a method for efficient identification of the architecture
of neural networks through parameters sharing \citep{nn_param_sharing}.
Also, the method of Stochastic Neural Architecture search was suggested
by Xie et al in a recent publication \citep{nn_snas}. Moreover, Zhou
et al introduced a Bayesian approach for neural architecture search
\citep{nn_bayes}.

Recently, genetic algorithms have been incorporated to identify the
optimal set of parameters of neural networks for drug discovery \citep{nn_drug}.
Kim et al proposed \citep{nn_estimates} genetic algorithms to train
neural networks for predicting preliminary cost estimates. Moreover,
Kalogirou proposed the usage of genetic algorithms for effective training
of neural networks for the optimization of solar systems \citep{nn_solar}.
The ability of neural networks to perform feature selection with the
assistance of genetic algorithms was also studied in the work of Tong
et al \citep{nn_feature}. Recently, Ruehle provided a study of the
string landscape using genetic algorithms to train artificial neural
networks \citep{nn_string}.

A method that was proposed relatively recently and it is based on
Grammatical Evolution \citep{ge1}, dynamically identifies both the
optimal architecture of artificial neural networks and the optimal
values of its parameters \citep{nnc}. This method has been applied
in a series of problems in the recent literature, such as problems
presented in chemistry \citep{nnc_amide1}, identification of the
solution of differential equations \citep{nnc_de}, medical problems
\citep{nnc_feas}, problems related to education \citep{nnc_student},
autism screening \citep{nnc_autism} etc. A key advantage of this
technique is that it can isolate from the initial features of the
problem those that are most important in training the model, thus
significantly reducing the required number of parameters that need
to be identified.

However, the method of constructing artificial neural networks can
easily get trapped in local minima of the training error since it
does not have any technique to avoid them. Furthermore, although the
method can get quite close to a minimum of the training error, it
often does not reach it since there is no technique in the method
to train the generated parameters. In this technique, it is proposed
to enhance the original method of constructing artificial neural networks
by periodically applying a modified genetic algorithm to randomly
selected chromosomes of Grammatical Evolution. This modified genetic
algorithm preserves the architecture created by the Grammatical Evolution
method and effectively locates the parameters of the artificial neural
network by reducing the training error. In addition, the proposed
genetic algorithm through appropriate penalty factors imposed on the
fitness function prevents the artificial neural network from overfitting.

The motivation of the proposed method is the need to address two main
challenges in training artificial neural networks: getting trapped
in local minima and the phenomenon of overfitting. Getting trapped
in local minima limits the model's ability to minimize training error,
leading to poor performance on test data. Overfitting similarly reduces
generalization, as the model adapts excessively to the training data.
The proposed method combines Grammatical Evolution with a modified
genetic algorithm to address these problems. Grammatical Evolution
is used for the dynamic construction of the neural network's architecture,
while the genetic algorithm optimizes the network's parameters while
preserving its structure. Additionally, penalty factors are introduced
in the cost function to prevent overfitting. A key innovation is the
use of an algorithm that measures the network's tendency to lose generalization
capability when neuron activations become saturated. This is achieved
by monitoring the input values of the sigmoid function and imposing
penalties when they exceed a specified range. Experimental tests showed
that the method outperforms other techniques such as ADAM, BFGS, and
RBF networks, in both classification and regression problems. Statistical
analysis confirmed the significant improvement in performance, with
very low p-values in the comparisons.

For the suggested work, the main contributions are as follows:
\begin{enumerate}
\item A novel hybrid framework that effectively combines grammatical evolution
for neural architecture search with constrained genetic algorithms
for parameter optimization, addressing both structural design and
weight training simultaneously.
\item An innovative penalty mechanism within the genetic algorithm's fitness
function that dynamically monitors and controls neuron activation
patterns to prevent overfitting, demonstrated to reduce test error
by an average of 15.27\% compared to standard approaches.
\item Comprehensive experimental validation across 53 diverse datasets showing
statistically significant improvements ($p<1.1\times10^{-8}$) over
traditional optimization methods, with particular effectiveness in
medical and financial domains where overfitting risks are critical.
\item Detailed analysis of the method's computational characteristics and
scalability, providing practical guidelines for implementation in
real-world scenarios with resource constraints.
\end{enumerate}
Although the method can construct the correct network structure, its
parameters often remain suboptimal. This means the network fails to
fully exploit its architecture's potential, resulting in lower performance
compared to other approaches. Furthermore, the absence of efficient
training mechanisms leads to increased training times, making the
method less practical for applications requiring quick results. In
specific application scenarios, these limitations become even more
apparent. For instance, with high-dimensional data, the method struggles
to identify relationships between features while computational times
become prohibitive. With limited training data, the constructed networks
tend to overfit, resulting in poor generalization to new data. For
real-time applications, the high computational complexity makes the
method impractical. Compared to other approaches like traditional
neural networks with backpropagation, modern deep learning architectures,
or meta-learning methods, grammatical evolution appears inferior in
several aspects. It requires significantly more computational resources,
consistently achieves lower performance, and presents scalability
limitations. These factors restrict the method's application in production
systems where stability and result predictability are crucial. The
practical implications of these limitations are substantial. The method
requires extensive hyperparameter tuning to produce acceptable results,
while its performance can be unpredictable and vary significantly
between different runs. For successful application to real-world problems,
additional processing and result validation are often necessary. Despite
its limitations, the method offers interesting capabilities for the
automatic construction of neural network architectures. However, to
become truly competitive against existing approaches, it requires
the development of more sophisticated optimization algorithms to reduce
local minima trapping, the integration of efficient parameter training
mechanisms, and improvements in method scalability. Only by addressing
these issues can grammatical evolution emerge as an attractive alternative
in the field of automated neural network design.

\textcolor{red}{The main contributions of the proposed work can be
summarized as follows:}
\begin{enumerate}
\item \textcolor{red}{Periodic application of an optimization technique
to randomly selected chromosomes with the aim of improving the performance
of the selected neural network but also of faster finding the global
minimum of the error function.}
\item \textcolor{red}{The training of the artificial neural network by the
optimization method is done in such a way as not to destroy the architecture
of the neural network that Grammatical Evolution has already constructed.}
\item \textcolor{red}{The training of the artificial neural network from
the optimization function is carried out using a modified fitness
function, where an attempt is made to adapt the network parameters
without losing its generalization properties.}
\end{enumerate}
The remaining of this article is organized as follows: in section
\ref{sec:Method-description} the proposed method and the accompanied
genetic algorithm are introduced, in section \ref{sec:Results} the
experimental datasets and the series of experiments conducted are
listed and discussed thoroughly followed by the section \ref{sec:Discussion},
where a discussion on the experimental results is provided and final
in section 5 some conclusions are discussed.

\section{Method description\label{sec:Method-description}}

This section provides a details description of the original neural
network construction method and continuous with the proposed genetic
algorithm and concludes with the overall algorithm.

\subsection{The neural construction method \label{subsec:The-neural-construction}}

The neural construction method utilizes the technique of Grammatical
Evolution to produce artificial neural networks. Grammatical Evolution
is an evolutionary process where the chromosomes are vectors of positive
integers. These integers represent rules from a Backs - Naur form
(BNF) grammar \citep{bnf1} of the target language.\textbf{ }The method
was incorporated in various cases, such as data fitting\textbf{ }\citep{ge_program1,ge_program2},
composition of music \citep{ge_music}, video games \citep{ge_pacman,ge_supermario},
energy problems \citep{ge_energy}, cryptography \citep{ge_crypt},
economics \citep{ge_trading} etc. Any BNF grammar is defined as a
set\textbf{ $G=\left(N,T,S,P\right)$ }where the letters have the
following definitions:
\begin{itemize}
\item The set $N$ represents the non - terminal symbols of the grammar.
\item The set $T$ contains the terminal symbols of the grammar. 
\item The start symbol of the grammar is denoted as $S$.
\item The production rules of the grammar are enclosed in the set $P$
\end{itemize}
The Grammatical Evolution production procedure initiates from the
starting symbol $S$ and following a series of steps, the method creates
valid programs by replacing non-terminal symbols with the right hand
of the selected production rule. The selection scheme has as:
\begin{itemize}
\item \textbf{Read} the next element V from the chromosome that is being
processed.
\item \textbf{Select} the next production rule following the equation:Rule
= V mod $N_{R}$. The symbol $N_{R}$ represents the total number
of production rules for the under processing non -- terminal symbol. 
\end{itemize}
The process of producing valid programs through the Grammatical Evolution
method is depicted graphically in Figure \ref{fig:geProcess}.

\begin{figure}[H]
\begin{centering}
\includegraphics[scale=0.5]{GEFC}
\par\end{centering}
\caption{The Grammatical Evolution process used to produce valid programs.\label{fig:geProcess}}

\end{figure}
The grammar used for the neural construction procedure is shown in
Figure \ref{fig:nncGrammar}. The numbers shown in parentheses are
the increasing numbers of the production rules for each non - terminal
symbol. The constant $d$ denotes the number of features in every
pattern of the input dataset.

\begin{figure}[H]
\begin{lyxcode}
S:=\textless Expression\textgreater ~~~~~~~~~~~~~~~~~~~~~~~~~~(0)

\textless Expression\textgreater ::=\textless Neuron\textgreater ~~~~~~~~~~~~~~~~~~~~(0)

~~~~~~~~~~~\textbar ~\textless Neuron\textgreater ~+~\textless Expression\textgreater ~~~~~~~(1)

\textless Neuron\textgreater ::=\textless Number\textgreater{*}sig(\textless Sum\textgreater +\textless Number\textgreater )~(0)

\textless Sum\textgreater ::=~\textless Number\textgreater{*}\textless Xlist\textgreater ~~~~~~~~~~~~(0)

~~~~~~~~~~~\textbar ~~~~\textless Sum\textgreater +\textless Sum\textgreater ~~~~~~~~~~~(1)

\textless Xlist\textgreater ::=~x1~~~~~~~~(0)

~~~~~~~~~~~~~\textbar ~~~~x2~~(1)

~~~~~~~~~~~~~..............

~~~~~~~~~~~~~\textbar ~~~~xd~~(d-1)

\textless Number\textgreater ::=~(\textless Dlist\textgreater .\textless Dlist\textgreater )~~~~~~~~(0)

~~~~~~~~~~~~~\textbar ~~~~(-\textless Dlist\textgreater .\textless Dlist\textgreater )~(1)

\textless Dlist\textgreater ::=~\textless Digit\textgreater ~~~~~~~~~~~~(0)

~~~~~~~~~~~~~\textbar ~\textless Digit\textgreater\textless Dlist\textgreater ~(1)

\textless Digit\textgreater ::=~0~~~~~~(0)

~~~~~~~~~~~~~\textbar ~~1~(1)

~~~~~~~~~~~~~...........

~~~~~~~~~~~~~\textbar ~~9~(9)
\end{lyxcode}
\caption{The proposed grammar for the construction of artificial neural networks
through Grammatical Evolution.\label{fig:nncGrammar}}
\end{figure}
The used grammar produces artificial neural networks with the following
form:

\begin{equation}
N\left(\overrightarrow{x},\overrightarrow{w}\right)=\sum_{i=1}^{H}w_{(d+2)i-(d+1)}\sigma\left(\sum_{j=1}^{d}x_{j}w_{(d+2)i-(d+1)+j}+w_{(d+2)i}\right)\label{eq:nn}
\end{equation}
The term $H$ stands for the number of processing units (weights)
of the neural network. The function $\sigma(x)$ represents the sigmoid
function.\textbf{ }The total number of parameters for this network
are computed through the following equation:
\begin{equation}
n=\left(d+2\right)H
\end{equation}
For example the following form:
\begin{equation}
N(x)=1.9\mbox{sig}\left(10.5x_{1}+3.2x_{3}+1.4\right)+2.1\mbox{sig}\left(2.2x_{2}-3.3x_{3}+3.2\right)
\end{equation}
denotes a produced neural network for a problem with 3 inputs $\left(x_{1},x_{2},x_{3}\right)$
and the number of processing nodes is $H=2$. The neural network produced
can be shown graphically in Figure \ref{fig:nnExample}.

\begin{figure}[H]
\begin{centering}
\includegraphics[scale=0.75]{example_diagram}
\par\end{centering}
\caption{An example of a produced neural network.\label{fig:nnExample}}

\end{figure}


\subsection{The used genetic algorithm \label{subsec:The-used-genetic}}

In the original method of constructing artificial neural networks,
it is proposed in this work to introduce the concept of local search,
through the periodic application of a genetic algorithm which should
maintain the structure of the neural network constructed by the original
method. Additionally, a second goal of this genetic algorithm should
be to avoid the problem of overfitting that could arise from simply
applying a local optimization method to the previous artificial neural
network. For the first goal of the modified genetic algorithm consider
the example neural network shown before:
\begin{equation}
N(x)=1.9\mbox{sig}\left(10.5x_{1}+3.2x_{3}+1.4\right)+2.1\mbox{sig}\left(2.2x_{2}-3.3x_{3}+3.2\right)
\end{equation}
The weight vector $\overrightarrow{w}$ for this neural network would
be 
\begin{equation}
\overrightarrow{w}=\left[1.9,10.5,0.0,3.2,1.4,2.1,0.0,2.2,-3.3,3.2\right]\label{eq:exampleNN}
\end{equation}
In order to protect the structure of this artificial neural network,
the modified genetic algorithm should allow changes in the parameters
of this network within a value interval, which can be considered to
be the pair of vectors $\left[\overrightarrow{L,}\overrightarrow{R}\right]$.
The elements for the vector $\overrightarrow{L}$ are defined as 
\begin{equation}
L_{i}=-F\times\left|w_{i}\right|,\ i=1,\ldots,n\label{eq:createL}
\end{equation}
where $F$ is positive number with $F>1$. Likewise the right bound
for the parameters $\overrightarrow{R}$ is defined from the following
equation:
\begin{equation}
R_{i}=F\times\left|w_{i}\right|,i=1,\ldots,n\label{eq:createR}
\end{equation}
For the example weight vector of equation \ref{eq:exampleNN} and
for $F=2$ the following vectors are used:
\[
\begin{array}{ccc}
L & = & \left[-3.8,-21.0,0.0,-6.4,-2.8,-4.2,0.0,-4.4,-6.6,-6.4\right]\\
R & = & \left[\ \ \ 3.8,\ \ \ 21.0,0.0,\ \ \ 6.4,\ \ \ \ 2.8,\ \ \ \ 4.2,0.0,\ \ \ 4.4,\ \ \ 6.6,\ \ \ \ 6.4\right]
\end{array}
\]
The modified genetic algorithm should also prevent the artificial
neural networks it trains from the phenomenon of overfitting, which
would lead to poor results on the test dataset. For this reason a
quantity derived from the publication of Anastasopoulos et al. \citep{nnt_bound}
is utilized here. The sigmoid function, that is used as the activation
function of neural networks is defined as:
\begin{equation}
\sigma(x)=\frac{1}{1+\exp(-x)}
\end{equation}
A plot for this function is shown in Figure \ref{fig:plotsigma}.

\begin{figure}[H]
\begin{centering}
\includegraphics[scale=0.75]{sig}
\par\end{centering}
\caption{Plot of the sigmoid function $\sigma(x)$.\label{fig:plotsigma}}

\end{figure}
As is clear from the equation and the figure, as the value of the
parameter x increases, the function tends very quickly to 1. On the
other hand, the function will take values very close to 0 as the parameter
x decreases. This means that the function very quickly loses the generalizing
abilities it has and therefore large changes in the value of the parameter
x will not cause proportional variations in the value of the sigmoid
function. Therefore, the quantity $B\left(N\left(\overrightarrow{x},\overrightarrow{w}\right),a\right)$
was introduced in that paper to measure this effect. This quantity
is calculated through the process of Algorithm \ref{alg:CalculationBound}.
\textcolor{red}{This function may be used to avoid overfitting, by
limiting the parameters of the neural network to intervals that depend
on the objective problem. The user defined parameter $a$ is used
here as a limit for the input value of the sigmoid unit. If this value
exceeds $a$, then probably the neural network has a reduced generalization
ability, since the sigmoid output will be the same regardless of any
change in the input value.}

\begin{algorithm}[H]
\caption{The algorithm used to calculate the bounding quantity for neural network
$N(x,w)$.\label{alg:CalculationBound}}

\textbf{function} $\mbox{evalB}\left(N\left(\overrightarrow{x},\overrightarrow{w}\right),a\right)$
\begin{enumerate}
\item \textbf{Inputs}: The Neural network $N\left(\overrightarrow{x},\overrightarrow{w}\right)$
and the double precision value $a,\ a>1$.
\item \textbf{Set} $s=0$
\item \textbf{For} $i=1..H$ \textbf{Do}
\begin{enumerate}
\item \textbf{For} $j=1..M$ \textbf{Do}
\begin{enumerate}
\item \textbf{Calculate} $v=\sum_{kT=1}^{d}w_{(d+2)i-(d+i)+k}x_{jk}+w_{(d+2)i}$
\item \textbf{If} $\left|v\right|>a$ \textbf{set} $s=s+1$
\end{enumerate}
\item \textbf{EndFor}
\end{enumerate}
\item \textbf{EndFor}
\item \textbf{Return} $\frac{s}{H\star M}$
\end{enumerate}
\textbf{End Function}
\end{algorithm}
The overall proposed modified genetic algorithm is shown in Algorithm
\ref{alg:The-modified-Genetic}.

\begin{algorithm}[H]
\caption{The modified Genetic Algorithm.\label{alg:The-modified-Genetic}}

\textbf{Function} $\mbox{mGA}\left(\overrightarrow{L},\overrightarrow{R,}a,\lambda\right)$
\begin{enumerate}
\item \textbf{Input}s: The bound vectors $\overrightarrow{L,}\overrightarrow{R}$
and the bounding factor $a$ and $\lambda$ a positive value with
$\lambda>1$.
\item \textbf{Set} as $N_{K}$the number of allowed generations and as $N_{G}$
the number of used chromosomes.
\item \textbf{Set} as $p_{S}$ the selection rate and as $p_{M}$ the mutation
rate.
\item \textbf{Initialize} $N_{G}$ chromosomes inside the bounding boxes
$\overrightarrow{L,}\overrightarrow{R}$.
\item \textbf{Set} $k=0$, the generation number.
\item \textbf{For} $i=1,\ldots,N_{G}$\label{enu:For}
\begin{enumerate}
\item \textbf{Obtain} the corresponding neural network $N_{i}\left(\overrightarrow{x},\overrightarrow{g_{i}}\right)$
for the chromosome $g_{i}$.
\item \textbf{Set} $e_{i}=\sum_{j=1}^{M}\left(N_{i}\left(\overrightarrow{x}_{j},\overrightarrow{w_{i}}\right)-y_{j}\right)^{2}$
\item \textbf{Set} $B_{i}=\mbox{eval}\left(N_{i}\left(\overrightarrow{x},\overrightarrow{g_{i}}\right),a\right)$
using the algorithm \ref{alg:CalculationBound}.
\item \textbf{Set} $f_{i}=e_{i}\times\left(1+\lambda B_{i}^{2}\right)$
as the fitness value of chromosome $g_{i}$
\end{enumerate}
\item \textbf{End For}
\item \textbf{Select} the best $\left(1-p_{s}\right)\times N_{G}$ chromosomes,
that will be copied intact to the next generation. The remaining will
be substituted by individuals produced by crossover and mutation.
\item \textbf{Set} $k=k+1$
\item \textbf{If} $k\le N_{K}$ \textbf{goto} step \ref{enu:For}.
\end{enumerate}
\textbf{End function}
\end{algorithm}


\subsection{The overall algorithm}

The overall algorithm uses the procedures presented previously to
achieve greater accuracy in calculations as well as to avoid overfitting
phenomena. The steps of the overall algorithm have as follows:
\begin{enumerate}
\item \textbf{Initialization}.
\begin{enumerate}
\item \textbf{Set} as $N_{C}$ the number of chromosomes for the Grammatical
Evolution procedure and as $N_{G}$ the maximum number of allowed
generations.
\item \textbf{Set} as $p_{S}$ the selection rate and as $p_{M}$ the mutation
rate.
\item \textbf{Let} $N_{I}$ be the number of chromosomes to which the modified
genetic algorithm will be periodically applied. 
\item \textbf{Let} $N_{T}$ be the number of generations that will pass
before applying the modified genetic algorithm to randomly selected
chromosomes.
\item \textbf{Set} the weight factor $F$ with $F>1$.
\item \textbf{Set} the values $N_{K},\ a,\ \lambda$ used in the modified
genetic algorithm.
\item \textbf{Initialize} randomly the $N_{C}$ chromosomes as sets of randomly
selected integers.
\item \textbf{Set} the generation number $k=0$ 
\end{enumerate}
\item \textbf{Fitness Calculation}.
\begin{enumerate}
\item \textbf{For} $i=1,\ldots,N_{C}$ \textbf{do}
\begin{enumerate}
\item \textbf{Obtain} the chromosome $g_{i}$
\item \textbf{Create} the corresponding neural network $N_{i}\left(\overrightarrow{x},\overrightarrow{w}\right)$
using Grammatical Evolution.
\item \textbf{Set} the fitness value $f_{i}=\sum_{j=1}^{M}\left(N_{i}\left(\overrightarrow{x}_{j},\overrightarrow{w}\right)-y_{j}\right)^{2}$
\end{enumerate}
\item \textbf{End For}
\end{enumerate}
\item \textbf{Genetic Operations}.
\begin{enumerate}
\item \textbf{Select} the best $\left(1-p_{s}\right)\times N_{G}$ chromosomes,
that will be copied intact to the next generation. 
\item \textbf{Create} $p_{S}N$ chromosomes using one - point crossover.
For every couple $\left(c_{1},c_{2}\right)$ of produced offsprings
two distinct chromosomes are selected from the current population
using tournament selection. An example of the one - point crossover
procedure is shown graphically in Figure \ref{fig:onePoint}.
\item For every chromosome and for each element select a random number $r\le1$.
Alter the current element when $r\le p_{M}$
\end{enumerate}
\item \textbf{Local search.}
\begin{enumerate}
\item \textbf{If} $k\ \mbox{mod}\ N_{T}=0$ \textbf{then}
\begin{enumerate}
\item \textbf{Set} $S=\left\{ g_{r_{1}},g_{r_{2}},\ldots,g_{r_{N_{I}}}\right\} $
a group of $N_{I}$ randomly selected chromosomes from the genetic
population.
\item \textbf{For} every member $g\in S$ \textbf{do}
\begin{enumerate}
\item \textbf{Obtain} the corresponding neural network $N_{g}\left(\overrightarrow{x},\overrightarrow{w}\right)$
for the chromosome $g$.
\item \textbf{Create} the left bound vector $\overrightarrow{L_{g}}$ and
the right bound vector $\overrightarrow{R_{g}}$ for $g$ using the
equations \ref{eq:createL},\ref{eq:createR} respectively.
\item \textbf{Set} $g=\mbox{mga}\left(\overrightarrow{L_{g}},\overrightarrow{R_{g},}a,\lambda\right)$
using the steps of algorithm \ref{alg:The-modified-Genetic}.
\end{enumerate}
\item \textbf{End For}
\end{enumerate}
\item \textbf{Endif}
\end{enumerate}
\item \textbf{Termination Check}.
\begin{enumerate}
\item \textbf{Set} $k=k+1$
\item \textbf{If} $k\le N_{G}$ goto \textbf{Fitness Calculation}.
\end{enumerate}
\item \textbf{Application to the test set}.
\begin{enumerate}
\item \textbf{Obtain} the chromosome $g^{*}$ with the lowest fitness value
and create through Grammatical Evolution the corresponding neural
network $N^{*}\left(\overrightarrow{x},\overrightarrow{w}\right)$
\item \textbf{Apply} the neural network $N^{*}\left(\overrightarrow{x},\overrightarrow{w}\right)$
and report the corresponding error value.
\end{enumerate}
%
\end{enumerate}
\begin{figure}[H]
\begin{centering}
\includegraphics[scale=0.5]{onepoint_crossover}
\par\end{centering}
\caption{An example of the one - point crossover procedure.\label{fig:onePoint}}

\end{figure}
Also the main steps of the overall algorithm are graphically illustrated
in Figure \ref{fig:flow}.
\begin{figure}[H]
\begin{centering}
\includegraphics[scale=0.5]{flowChart}
\par\end{centering}
\caption{The flowchart of the overall algorithm.\label{fig:flow}}

\end{figure}


\section{Experimental results\label{sec:Results}}

The validation of the proposed method was performed using a wide series
of classification and regression datasets, available from various
sources from the Internet. These datasets were downloaded from:
\begin{enumerate}
\item The UCI database, \url{https://archive.ics.uci.edu/}(accessed on
22 January 2025)\citep{uci}
\item The Keel website, \url{https://sci2s.ugr.es/keel/datasets.php}(accessed
on 22 January 2025)\citep{Keel}.
\item The Statlib URL \url{https://lib.stat.cmu.edu/datasets/index}(accessed
on 22 January 2025). 
\end{enumerate}

\subsection{Experimental datasets }

The following datasets were utilized in the conducted experiments:
\begin{enumerate}
\item \textbf{Appendictis} which is a medical dataset \citep{appendicitis}. 
\item \textbf{Alcohol}, which is dataset regarding alcohol consumption \citep{alcohol}. 
\item \textbf{Australian}, which is a dataset produced from various bank
transactions \citep{australian}.
\item \textbf{Balance} dataset \citep{balance}, produced from various psychological
experiments.
\item \textbf{Cleveland}, a medical dataset which was discussed in a series
of papers \citep{cleveland1,cleveland2}. 
\item \textbf{Circular} dataset, which is an artificial dataset.
\item \textbf{Dermatology}, a medical dataset for dermatology problems \citep{dermatology}.
\item \textbf{Ecoli}, which is related to protein problems \citep{ecoli}.
\item \textbf{Glass} dataset, that contains measurements from glass component
analysis. 
\item \textbf{Haberman}, a medical dataset related to breast cancer.
\item \textbf{Hayes-roth} dataset \citep{hayes-roth}.
\item \textbf{Heart}, which is a dataset related to heart diseases \citep{heart}.
\item \textbf{HeartAttack}, which is a medical dataset for the detection
of heart diseases
\item \textbf{Housevotes}, a dataset which is related to the Congressional
voting in USA \citep{housevotes}.
\item \textbf{Ionosphere}, a dataset that contains measurements from the
ionosphere \citep{ion1,ion2}.
\item \textbf{Liverdisorder}, a medical dataset that was studied thoroughly
in a series of papers\citep{liver,liver1}.
\item \textbf{Lymography} \citep{lymography}.
\item \textbf{Mammographic}, which is a medical dataset used for the prediction
of breast cancer \citep{mammographic}.
\item \textbf{Parkinsons}, which is a medical dataset used for the detection
of Parkinson's disease \citep{parkinsons1,parkinsons2}.
\item \textbf{Pima}, which is a medical dataset for the detection of diabetes\citep{pima}.
\item \textbf{Phoneme}, a dataset that contains sound measurements.
\item \textbf{Popfailures}, a dataset related to experiments regarding climate
\citep{popfailures}.
\item \textbf{Regions2}, a medical dataset applied to liver problems \citep{regions2}.
\item \textbf{Saheart}, which is a medical dataset concerning heart diseases\citep{saheart}.
\item \textbf{Segment} dataset \citep{segment}.
\item \textbf{Statheart}, a medical dataset related to heart diseases.
\item \textbf{Spiral}, an artificial dataset with two classes.
\item \textbf{Student}, which is a dataset regarding experiments in schools
\citep{student}.
\item \textbf{Transfusion}, which is a medical dataset \citep{transfusion}.
\item \textbf{Wdbc}, which is a medical dataset regarding breast cancer
\citep{wdbc1,wdbc2}.
\item \textbf{Wine}, a dataset regarding measurements about the quality
of wines \citep{wine1,wine2}.
\item \textbf{EEG}, which is dataset regardingEEG recordings \citep{eeg1,eeg2}.
From this dataset the following cases were used: Z\_F\_S, ZO\_NF\_S,
ZONF\_S and Z\_O\_N\_F\_S.
\item \textbf{Zoo}, which is a dataset regarding animal classification \citep{zoo}
.
\end{enumerate}
Moreover a series of regression datasets was adopted in the conducted
experiments. The list with the regression datasets has as follows:
\begin{enumerate}
\item \textbf{Abalone}, which is a dataset about the age of abalones \citep{abalone}.
\item \textbf{Airfoil}, a dataset founded in NASA \citep{airfoil}.
\item \textbf{Auto}, a dataset related to the consumption of fuels from
cars.
\item \textbf{BK}, which is used to predict the points scored in basketball
games. 
\item \textbf{BL}, a dataset that contains measurements from electricity
experiments.
\item \textbf{Baseball}, which is a dataset used to predict the income of
baseball players.
\item \textbf{Concrete}, which is a civil engineering dataset \citep{concrete}.
\item \textbf{DEE}, a dataset that is used to predict the price of electricity.
\item \textbf{Friedman}, which is an artificial dataset\citep{friedman}.
\item \textbf{FY, }which is a dataset regarding the longevity of fruit flies. 
\item \textbf{HO}, a dataset located in the STATLIB repository.
\item \textbf{Housing}, regarding the price of houses \citep{housing}.
\item \textbf{Laser}, which contains measurements from various physics experiments.
\item \textbf{LW}, a dataset regarding the weight of babes.
\item \textbf{Mortgage}, a dataset that contains measurements from the economy
of USA.
\item \textbf{PL} dataset, located in the STALIB repository.
\item \textbf{Plastic}, a dataset regarding problems occurred with the pressure
on plastics.
\item \textbf{Quake}, a dataset regarding the measurements of earthquakes.
\item \textbf{SN}, a dataset related to trellising and pruning.
\item \textbf{Stock}, which is a dataset regarding stocks.
\item \textbf{Treasury}, a dataset that contains measurements from the economy
of USA.
\end{enumerate}

\subsection{Experiments}

The software used in the experiment was coded in C++ with the assistance
of the freely available Optimus environment \citep{optimus}. Every
experiments was conducted 30 times and each time different seed for
the random generator was used. The experiments were validated using
the ten - fold cross validation technique. The average classification
error, as measured in the corresponding test set was reported for
the classification datasets. This error is calculated through the
following formula:
\begin{equation}
E_{C}\left(N\left(\overrightarrow{x},\overrightarrow{w}\right)\right)=100\times\frac{\sum_{i=1}^{N}\left(\mbox{class}\left(N\left(\overrightarrow{x_{i}},\overrightarrow{w}\right)\right)-y_{i}\right)}{N}
\end{equation}
Here the test set $T$ is a set $T=\left(x_{i},y_{i}\right),\ i=1,\ldots,N$.
Likewise, the average regression error is reported for the regression
datasets. This error can be obtained using the following equation:
\begin{equation}
E_{R}\left(N\left(\overrightarrow{x},\overrightarrow{w}\right)\right)=\frac{\sum_{i=1}^{N}\left(N\left(\overrightarrow{x_{i}},\overrightarrow{w}\right)-y_{i}\right)^{2}}{N}
\end{equation}
The experiments were executed on\textbf{ }an AMD Ryzen 5950X with
128GB of RAM and the used operating system was Debian Linux. The values
for the parameters of the proposed method are shown in Table \ref{tab:expValues}. 

\begin{table}[H]
\caption{The values for the parameters of the proposed method.\label{tab:expValues}}

\centering{}%
\begin{tabular}{|c|c|c|}
\hline 
PARAMETER & MEANING & VALUE\tabularnewline
\hline 
\hline 
$N_{C}$ & Chromosomes & 500\tabularnewline
\hline 
$N_{G}$ & Maximum number of generations & 500\tabularnewline
\hline 
$N_{K}$ & Number of generations for the modified Genetic Algorithm & 50\tabularnewline
\hline 
$p_{S}$ & Selection rate & 0.1\tabularnewline
\hline 
$p_{M}$ & Mutation rate & 0.05\tabularnewline
\hline 
$N_{T}$ & Generations before local search & 20\tabularnewline
\hline 
$N_{I}$ & Chromosomes participating in local search & 20\tabularnewline
\hline 
$a$ & Bounding factor & 10.0\tabularnewline
\hline 
$F$ & Scale factor for the margins & 2.0\tabularnewline
\hline 
$\lambda$ & Value used for penalties & 100.0\tabularnewline
\hline 
\end{tabular}
\end{table}
The parameter values \LyXZeroWidthSpace\LyXZeroWidthSpace have been
chosen in such a way that there is a balance between the speed of
the proposed method and its efficiency. In the following tables that
describe the experimental results the following notation is used:
\begin{enumerate}
\item The column DATASET represents the used dataset.
\item The column ADAM represents the incorporation of the ADAM optimization
method \citep{nn_adam} to train a neural network with $H=10$ processing
nodes.
\item The column BFGS stands for the usage of a BFGS variant of Powell \citep{powell}
to train an artificial neural network with $H=10$ processing nodes.
\item The column GENETIC represents the incorporation of a Genetic Algorithm
with the same parameter set as provided in Table \ref{tab:expValues}
to train a neural network with $H=10$ processing nodes.
\item The column RBF describes the experimental results obtained by the
application of a Radial Basis Function (RBF) network \citep{rbf1,rbf2}
with $H=10$ hidden nodes.
\item The column NNC stands for the usage of the original neural construction
method.
\item The column NEAT represents the usage of the NEAT method (NeuroEvolution
of Augmenting Topologies ) \citep{neat}.
\item The column PRUNE stands for the the usage of OBS pruning method \citep{prune},
as coded in Fast Compressed Neural Networks library \citep{fcn}.
\item The column DNN represents the application of the deep neural network
provided in the Tiny Dnn library, which is available from \url{https://github.com/tiny-dnn/tiny-dnn}(accessed
on 7 September 2025). The network was trained using the AdaGrad optimizer
\citep{adagrad}.
\item The column PROPOSED denotes the usage of the proposed method.
\item The row AVERAGE represents the average classification or regression
error for all datasets in the corresponding table.
\end{enumerate}
\textcolor{red}{Based on the table \ref{tab:expClass} with 36 classification
datasets and nine methods (lower percentage implies lower error),
the PROPOSED method attains the lowest mean error (21.18\%) and the
best average rank across datasets (1.83 on a 1--9 scale). The current
work achieves the best result in 18 out of 36 datasets. The remaining
per-dataset best results are distributed as follows: RBF 4, PRUNE
4, DNN 3, NEAT 3, NNC 3, and GENETIC 1, while ADAM and BFGS record
no first-place finishes. In head-to-head, dataset-wise comparisons,
the proposed method outperforms each alternative on the vast majority
of datasets: versus ADAM on 35/36 datasets, versus BFGS on 33/36,
versus GENETIC on 32/36, versus RBF on 30/36, versus NEAT on 33/36,
versus PRUNE on 31/36, versus DNN on 32/36, and versus NNC on 32/36.
The average absolute error reduction of PROPOSED relative to each
competitor, computed as “competitor \textminus{} PROPOSED” in percentage
points and then averaged over the 36 datasets, is 14.39 (ADAM), 13.51
(BFGS), 6.12 (GENETIC), 8.60 (RBF), 11.90 (NEAT), 7.53 (PRUNE), 6.64
(DNN), and 2.60 (NNC). The corresponding mean relative reductions,
computed as (competitor \textminus{} PROPOSED)/competitor and then
averaged per dataset, are 39.24\%, 34.29\%, 22.83\%, 26.50\%, 37.09\%,
20.91\%, 22.95\%, and 12.65\%. The few datasets where the proposed
method is worse are concentrated in specific cases: ADAM only on POPFAILURES;
BFGS on CIRCULAR, PHONEME, and POPFAILURES; GENETIC on CIRCULAR, GLASS,
LIVERDISORDER, and PHONEME; RBF on APPENDICITIS, CIRCULAR, GLASS,
HABERMAN, LIVERDISORDER, and SPIRAL; NEAT on ECOLI, HABERMAN, and
LIVERDISORDER; PRUNE on ALCOHOL, DERMATOLOGY, IONOSPHERE, LYMOGRAPHY,
and POPFAILURES; DNN on HABERMAN, HOUSEVOTES, SEGMENT, and ZONF\_S;
and NNC on AUSTRALIAN, HEARTATTACK, HOUSEVOTES, and IONOSPHERE. The
table’s AVERAGE row is consistent with this picture: PROPOSED has
the lowest mean error (21.18\%) among all methods. Relative to the
second-best mean in that row (NNC: 24.79\%), PROPOSED reduces error
by 3.61 percentage points, corresponding to about a 14.56\% relative
reduction. Median errors tell the same story: PROPOSED has a median
of 19.24\%, lower than NNC (21.21\%) and DNN (25.83\%). Overall, PROPOSED
shows consistent superiority in terms of error, as reflected by the
mean, the across-dataset ranking, and the dataset-wise head-to-head
counts, with only a small number of dataset-specific exceptions.}

\begin{table}[H]
\caption{Experimental results using a variety of machine learning methods for
the classification datasets.\label{tab:expClass}}

\centering{}{\scriptsize{}%
\begin{tabular}{|c|c|c|c|c|c|c|c|c|c|}
\hline 
{\scriptsize DATASET} & {\scriptsize ADAM} & {\scriptsize BFGS} & {\scriptsize GENETIC} & {\scriptsize RBF} & {\scriptsize NEAT} & {\scriptsize PRUNE} & {\scriptsize DNN} & {\scriptsize NNC} & {\scriptsize PROPOSED}\tabularnewline
\hline 
\hline 
{\scriptsize APPENDICITIS} & {\scriptsize 16.50\%} & {\scriptsize 18.00\%} & {\scriptsize 24.40\%} & {\scriptsize 12.23\%} & {\scriptsize 17.20\%} & {\scriptsize 15.97\%} & {\scriptsize 17.30\%} & {\scriptsize 14.40\%} & {\scriptsize 14.30\%}\tabularnewline
\hline 
{\scriptsize ALCOHOL} & {\scriptsize 57.78\%} & {\scriptsize 41.50\%} & {\scriptsize 39.57\%} & {\scriptsize 49.32\%} & {\scriptsize 66.80\%} & {\scriptsize 15.75\%} & {\scriptsize 39.04\%} & {\scriptsize 37.72\%} & {\scriptsize 35.60\%}\tabularnewline
\hline 
{\scriptsize AUSTRALIAN} & {\scriptsize 35.65\%} & {\scriptsize 38.13\%} & {\scriptsize 32.21\%} & {\scriptsize 34.89\%} & {\scriptsize 31.98\%} & {\scriptsize 43.66\%} & {\scriptsize 35.03\%} & {\scriptsize 14.46\%} & {\scriptsize 14.55\%}\tabularnewline
\hline 
{\scriptsize BALANCE} & {\scriptsize 12.27\%} & {\scriptsize 8.64\%} & {\scriptsize 8.97\%} & {\scriptsize 33.53\%} & {\scriptsize 23.14\%} & {\scriptsize 9.00\%} & {\scriptsize 24.56\%} & {\scriptsize 23.65\%} & {\scriptsize 7.84\%}\tabularnewline
\hline 
{\scriptsize CLEVELAND} & {\scriptsize 67.55\%} & {\scriptsize 77.55\%} & {\scriptsize 51.60\%} & {\scriptsize 67.10\%} & {\scriptsize 53.44\%} & {\scriptsize 51.48\%} & {\scriptsize 63.28\%} & {\scriptsize 50.93\%} & {\scriptsize 46.41\%}\tabularnewline
\hline 
{\scriptsize CIRCULAR} & {\scriptsize 19.95\%} & {\scriptsize 6.08\%} & {\scriptsize 5.99\%} & {\scriptsize 5.98\%} & {\scriptsize 35.18\%} & {\scriptsize 12.76\%} & {\scriptsize 21.87\%} & {\scriptsize 12.66\%} & {\scriptsize 6.92\%}\tabularnewline
\hline 
{\scriptsize DERMATOLOGY} & {\scriptsize 26.14\%} & {\scriptsize 52.92\%} & {\scriptsize 30.58\%} & {\scriptsize 62.34\%} & {\scriptsize 32.43\%} & {\scriptsize 9.02\%} & {\scriptsize 24.26\%} & {\scriptsize 21.54\%} & {\scriptsize 20.54\%}\tabularnewline
\hline 
{\scriptsize ECOLI} & {\scriptsize 64.43\%} & {\scriptsize 69.52\%} & {\scriptsize 54.67\%} & {\scriptsize 59.48\%} & {\scriptsize 43.44\%} & {\scriptsize 60.32\%} & {\scriptsize 60.79\%} & {\scriptsize 49.88\%} & {\scriptsize 48.82\%}\tabularnewline
\hline 
{\scriptsize GLASS} & {\scriptsize 61.38\%} & {\scriptsize 54.67\%} & {\scriptsize 52.86\%} & {\scriptsize 50.46\%} & {\scriptsize 55.71\%} & {\scriptsize 66.19\%} & {\scriptsize 56.05\%} & {\scriptsize 56.09\%} & {\scriptsize 53.52\%}\tabularnewline
\hline 
{\scriptsize HABERMAN} & {\scriptsize 29.00\%} & {\scriptsize 29.34\%} & {\scriptsize 28.66\%} & {\scriptsize 25.10\%} & {\scriptsize 24.04\%} & {\scriptsize 29.38\%} & {\scriptsize 25.73\%} & {\scriptsize 27.53\%} & {\scriptsize 26.80\%}\tabularnewline
\hline 
{\scriptsize HAYES-ROTH} & {\scriptsize 59.70\%} & {\scriptsize 37.33\%} & {\scriptsize 56.18\%} & {\scriptsize 64.36\%} & {\scriptsize 50.15\%} & {\scriptsize 45.44\%} & {\scriptsize 44.65\%} & {\scriptsize 33.69\%} & {\scriptsize 31.00\%}\tabularnewline
\hline 
{\scriptsize HEART} & {\scriptsize 38.53\%} & {\scriptsize 39.44\%} & {\scriptsize 28.34\%} & {\scriptsize 31.20\%} & {\scriptsize 39.27\%} & {\scriptsize 27.21\%} & {\scriptsize 30.67\%} & {\scriptsize 15.67\%} & {\scriptsize 15.45\%}\tabularnewline
\hline 
{\scriptsize HEARTATTACK} & {\scriptsize 45.55\%} & {\scriptsize 46.67\%} & {\scriptsize 29.03\%} & {\scriptsize 29.00\%} & {\scriptsize 32.34\%} & {\scriptsize 29.26\%} & {\scriptsize 32.97\%} & {\scriptsize 20.87\%} & {\scriptsize 21.77\%}\tabularnewline
\hline 
{\scriptsize HOUSEVOTES} & {\scriptsize 7.48\%} & {\scriptsize 7.13\%} & {\scriptsize 6.62\%} & {\scriptsize 6.13\%} & {\scriptsize 10.89\%} & {\scriptsize 5.81\%} & {\scriptsize 3.13\%} & {\scriptsize 3.17\%} & {\scriptsize 3.78\%}\tabularnewline
\hline 
{\scriptsize IONOSPHERE} & {\scriptsize 16.64\%} & {\scriptsize 15.29\%} & {\scriptsize 15.14\%} & {\scriptsize 16.22\%} & {\scriptsize 19.67\%} & {\scriptsize 11.32\%} & {\scriptsize 12.57\%} & {\scriptsize 11.29\%} & {\scriptsize 11.94\%}\tabularnewline
\hline 
{\scriptsize LIVERDISORDER} & {\scriptsize 41.53\%} & {\scriptsize 42.59\%} & {\scriptsize 31.11\%} & {\scriptsize 30.84\%} & {\scriptsize 30.67\%} & {\scriptsize 49.72\%} & {\scriptsize 32.21\%} & {\scriptsize 32.35\%} & {\scriptsize 31.32\%}\tabularnewline
\hline 
{\scriptsize LYMOGRAPHY} & {\scriptsize 39.79\%} & {\scriptsize 35.43\%} & {\scriptsize 28.42\%} & {\scriptsize 25.50\%} & {\scriptsize 33.70\%} & {\scriptsize 22.02\%} & {\scriptsize 24.07\%} & {\scriptsize 25.29\%} & {\scriptsize 23.72\%}\tabularnewline
\hline 
{\scriptsize MAMMOGRAPHIC} & {\scriptsize 46.25\%} & {\scriptsize 17.24\%} & {\scriptsize 19.88\%} & {\scriptsize 21.38\%} & {\scriptsize 22.85\%} & {\scriptsize 38.10\%} & {\scriptsize 19.83\%} & {\scriptsize 17.62\%} & {\scriptsize 16.74\%}\tabularnewline
\hline 
{\scriptsize PARKINSONS} & {\scriptsize 24.06\%} & {\scriptsize 27.58\%} & {\scriptsize 18.05\%} & {\scriptsize 17.41\%} & {\scriptsize 18.56\%} & {\scriptsize 22.12\%} & {\scriptsize 21.32\%} & {\scriptsize 12.74\%} & {\scriptsize 12.63\%}\tabularnewline
\hline 
{\scriptsize PHONEME} & {\scriptsize 29.43\%} & {\scriptsize 15.58\%} & {\scriptsize 15.55\%} & {\scriptsize 23.32\%} & {\scriptsize 22.34\%} & {\scriptsize 29.35\%} & {\scriptsize 22.68\%} & {\scriptsize 22.50\%} & {\scriptsize 21.52\%}\tabularnewline
\hline 
{\scriptsize PIMA} & {\scriptsize 34.85\%} & {\scriptsize 35.59\%} & {\scriptsize 32.19\%} & {\scriptsize 25.78\%} & {\scriptsize 34.51\%} & {\scriptsize 35.08\%} & {\scriptsize 32.63\%} & {\scriptsize 28.07\%} & {\scriptsize 23.34\%}\tabularnewline
\hline 
{\scriptsize POPFAILURES} & {\scriptsize 5.18\%} & {\scriptsize 5.24\%} & {\scriptsize 5.94\%} & {\scriptsize 7.04\%} & {\scriptsize 7.05\%} & {\scriptsize 4.79\%} & {\scriptsize 6.83\%} & {\scriptsize 6.98\%} & {\scriptsize 5.72\%}\tabularnewline
\hline 
{\scriptsize REGIONS2} & {\scriptsize 29.85\%} & {\scriptsize 36.28\%} & {\scriptsize 29.39\%} & {\scriptsize 38.29\%} & {\scriptsize 33.23\%} & {\scriptsize 34.26\%} & {\scriptsize 33.42\%} & {\scriptsize 26.18\%} & {\scriptsize 23.81\%}\tabularnewline
\hline 
{\scriptsize SAHEART} & {\scriptsize 34.04\%} & {\scriptsize 37.48\%} & {\scriptsize 34.86\%} & {\scriptsize 32.19\%} & {\scriptsize 34.51\%} & {\scriptsize 37.70\%} & {\scriptsize 35.11\%} & {\scriptsize 29.80\%} & {\scriptsize 28.04\%}\tabularnewline
\hline 
{\scriptsize SEGMENT} & {\scriptsize 49.75\%} & {\scriptsize 68.97\%} & {\scriptsize 57.72\%} & {\scriptsize 59.68\%} & {\scriptsize 66.72\%} & {\scriptsize 60.40\%} & {\scriptsize 32.04\%} & {\scriptsize 53.50\%} & {\scriptsize 48.20\%}\tabularnewline
\hline 
{\scriptsize SPIRAL} & {\scriptsize 47.67\%} & {\scriptsize 47.99\%} & {\scriptsize 48.66\%} & {\scriptsize 44.87\%} & {\scriptsize 48.66\%} & {\scriptsize 50.38\%} & {\scriptsize 45.64\%} & {\scriptsize 48.01\%} & {\scriptsize 44.95\%}\tabularnewline
\hline 
{\scriptsize STATHEART} & {\scriptsize 44.04\%} & {\scriptsize 39.65\%} & {\scriptsize 27.25\%} & {\scriptsize 31.36\%} & {\scriptsize 44.36\%} & {\scriptsize 28.37\%} & {\scriptsize 30.22\%} & {\scriptsize 18.08\%} & {\scriptsize 17.93\%}\tabularnewline
\hline 
{\scriptsize STUDENT} & {\scriptsize 5.13\%} & {\scriptsize 7.14\%} & {\scriptsize 5.61\%} & {\scriptsize 5.49\%} & {\scriptsize 10.20\%} & {\scriptsize 10.84\%} & {\scriptsize 6.93\%} & {\scriptsize 6.70\%} & {\scriptsize 4.05\%}\tabularnewline
\hline 
{\scriptsize TRANSFUSION} & {\scriptsize 25.68\%} & {\scriptsize 25.84\%} & {\scriptsize 24.87\%} & {\scriptsize 26.41\%} & {\scriptsize 24.87\%} & {\scriptsize 29.35\%} & {\scriptsize 25.92\%} & {\scriptsize 25.77\%} & {\scriptsize 23.16\%}\tabularnewline
\hline 
{\scriptsize WDBC} & {\scriptsize 35.35\%} & {\scriptsize 29.91\%} & {\scriptsize 8.56\%} & {\scriptsize 7.27\%} & {\scriptsize 12.88\%} & {\scriptsize 15.48\%} & {\scriptsize 9.43\%} & {\scriptsize 7.36\%} & {\scriptsize 4.95\%}\tabularnewline
\hline 
{\scriptsize WINE} & {\scriptsize 29.40\%} & {\scriptsize 59.71\%} & {\scriptsize 19.20\%} & {\scriptsize 31.41\%} & {\scriptsize 25.43\%} & {\scriptsize 16.62\%} & {\scriptsize 27.18\%} & {\scriptsize 13.59\%} & {\scriptsize 9.94\%}\tabularnewline
\hline 
{\scriptsize Z\_F\_S} & {\scriptsize 47.81\%} & {\scriptsize 39.37\%} & {\scriptsize 10.73\%} & {\scriptsize 13.16\%} & {\scriptsize 38.41\%} & {\scriptsize 17.91\%} & {\scriptsize 9.27\%} & {\scriptsize 14.53\%} & {\scriptsize 7.97\%}\tabularnewline
\hline 
{\scriptsize Z\_O\_N\_F\_S} & {\scriptsize 78.79\%} & {\scriptsize 65.67\%} & {\scriptsize 64.81\%} & {\scriptsize 48.70\%} & {\scriptsize 77.08\%} & {\scriptsize 71.29\%} & {\scriptsize 67.80\%} & {\scriptsize 48.62\%} & {\scriptsize 39.28\%}\tabularnewline
\hline 
{\scriptsize ZO\_NF\_S} & {\scriptsize 47.43\%} & {\scriptsize 43.04\%} & {\scriptsize 21.54\%} & {\scriptsize 9.02\%} & {\scriptsize 43.75\%} & {\scriptsize 15.57\%} & {\scriptsize 8.50\%} & {\scriptsize 13.54\%} & {\scriptsize 6.94\%}\tabularnewline
\hline 
{\scriptsize ZONF\_S} & {\scriptsize 11.99\%} & {\scriptsize 15.62\%} & {\scriptsize 4.36\%} & {\scriptsize 4.03\%} & {\scriptsize 5.44\%} & {\scriptsize 3.27\%} & {\scriptsize 2.52\%} & {\scriptsize 2.64\%} & {\scriptsize 2.60\%}\tabularnewline
\hline 
{\scriptsize ZOO} & {\scriptsize 14.13\%} & {\scriptsize 10.70\%} & {\scriptsize 9.50\%} & {\scriptsize 21.93\%} & {\scriptsize 20.27\%} & {\scriptsize 8.53\%} & {\scriptsize 16.20\%} & {\scriptsize 8.70\%} & {\scriptsize 6.60\%}\tabularnewline
\hline 
{\scriptsize\textbf{AVERAGE}} & {\scriptsize\textbf{36.45\%}} & {\scriptsize\textbf{35.71\%}} & {\scriptsize\textbf{28.25\%}} & {\scriptsize\textbf{30.73\%}} & {\scriptsize\textbf{32.19\%}} & {\scriptsize\textbf{27.94\%}} & {\scriptsize\textbf{27.82\%}} & {\scriptsize\textbf{24.79\%}} & {\scriptsize\textbf{21.18\%}}\tabularnewline
\hline 
\end{tabular}}{\scriptsize\par}
\end{table}
Additionally, the average classification error for all methods is
illustrated in Figure \ref{fig:The-average-classification}. \textcolor{red}{Additionally,
the dimension for each classification dataset and the number of distinct
classes are provided in Table \ref{tab:dimension}.}

\begin{table}[H]
\caption{Dimension and number of classes for each classification dataset.\label{tab:dimension}}

\centering{}%
\begin{tabular}{|c|c|c|}
\hline 
DATASET & FEATURES & CLASSES\tabularnewline
\hline 
\hline 
APPENDICITIS & 7 & 2\tabularnewline
\hline 
ALCOHOL & 154 & 4\tabularnewline
\hline 
AUSTRALIAN & 14 & 2\tabularnewline
\hline 
BALANCE & 4 & 3\tabularnewline
\hline 
CLEVELAND & 13 & 5\tabularnewline
\hline 
CIRCULAR & 5 & 2\tabularnewline
\hline 
DERMATOLOGY & 34 & 6\tabularnewline
\hline 
ECOLI & 7 & 8\tabularnewline
\hline 
GLASS & 9 & 6\tabularnewline
\hline 
HABERMAN & 3 & 2\tabularnewline
\hline 
HAYES-ROTH & 5 & 3\tabularnewline
\hline 
HEART & 13 & 2\tabularnewline
\hline 
HEARTATTACK & 13 & 2\tabularnewline
\hline 
HOUSEVOTES & 16 & 2\tabularnewline
\hline 
IONOSPHERE & 34 & 2\tabularnewline
\hline 
LIVERDISORDER & 6 & 2\tabularnewline
\hline 
LYMOGRAPHY & 18 & 4\tabularnewline
\hline 
MAMMOGRAPHIC & 5 & 2\tabularnewline
\hline 
PARKINSONS & 22 & 2\tabularnewline
\hline 
PHONEME & 5 & 2\tabularnewline
\hline 
PIMA & 8 & 2\tabularnewline
\hline 
POPFAILURES & 18 & 2\tabularnewline
\hline 
REGIONS2 & 18 & 5\tabularnewline
\hline 
SAHEART & 9 & 2\tabularnewline
\hline 
SEGMENT & 19 & 7\tabularnewline
\hline 
SPIRAL & 2 & 2\tabularnewline
\hline 
STATHEART & 13 & 2\tabularnewline
\hline 
STUDENT & 5 & 4\tabularnewline
\hline 
TRANSFUSION & 4 & 2\tabularnewline
\hline 
WDBC & 30 & 2\tabularnewline
\hline 
WINE & 13 & 3\tabularnewline
\hline 
Z\_F\_S & 21 & 3\tabularnewline
\hline 
Z\_O\_N\_F\_S & 21 & 5\tabularnewline
\hline 
ZO\_NF\_S & 21 & 3\tabularnewline
\hline 
ZONF\_S & 21 & 2\tabularnewline
\hline 
ZOO & 16 & 7\tabularnewline
\hline 
\end{tabular}
\end{table}

\begin{figure}[H]
\begin{centering}
\includegraphics[scale=0.8]{plot_all}
\par\end{centering}
\caption{The average classification error for all used datasets. \label{fig:The-average-classification}}

\end{figure}
Also, a line plot is provided in Figure \ref{fig:linePlot}for a series
of selected datasets to depict the effectiveness of the proposed method.
\begin{figure}[H]
\begin{centering}
\includegraphics[scale=0.8]{plot_line}
\par\end{centering}
\caption{Line plot for a series of classification datasets.\label{fig:linePlot}}

\end{figure}

\textcolor{red}{In table \ref{tab:expRegression} with 21 regression
datasets and nine methods (lower values indicate lower error), the
proposed method achieves the lowest mean error (4.28 versus 6.29 for
NNC in the AVERAGE row) and the lowest median (0.036). Its average
rank across datasets is 1.67 on a 1--9 scale, lower than any competitor
(next best: NNC 3.83, DNN 4.69, RBF 4.79). At the dataset level, the
current work is the strict winner on 12 of 21 datasets and ties for
best on 2 more (AIRFOIL with PRUNE and LW with NNC), thus topping
14/21 datasets overall. The remaining first places are taken by ADAM
(ABALONE and FY), GENETIC (FRIEDMAN and STOCK), RBF (BK and DEE),
and NNC (HO). The gap from the dataset-wise best alternative is typically
small: PROPOSED is within 5\% of the best on 15/21 datasets and within
10\% on 16/21, with the largest shortfalls appearing mainly on FRIEDMAN
(5.34 versus 1.249) and, more mildly, on DEE (0.22 versus 0.17) and
STOCK (4.69 versus 3.88). In head-to-head, dataset-wise comparisons,
PROPOSED has lower error than each alternative in the vast majority
of cases: versus ADAM it is better on 19/21 datasets (worse on ABALONE
and FY), versus BFGS on 20/21 (worse only on FRIEDMAN), versus GENETIC
on 19/21 (worse on FRIEDMAN and STOCK), versus RBF on 18/21 (worse
on BK, DEE, and FY), versus NEAT on 21/21, versus PRUNE on 19/21 with
1 tie (worse on FY), versus DNN on 18/21 (worse on BK, FRIEDMAN, and
FY), and versus NNC on 19/21 with 1 tie (worse on HO, tie on LW).
The average absolute error reduction of PROPOSED relative to each
competitor, computed as “competitor \textminus{} PROPOSED” and then
averaged over the 21 datasets, is approximately 18.18 (ADAM), 26.01
(BFGS), 5.03 (GENETIC), 5.74 (RBF), 10.37 (NEAT), 11.12 (PRUNE), 7.00
(DNN), and 2.01 (NNC); the corresponding mean relative reductions
are about 62\%, 64\%, 44\%, 53\%, 80\%, 50\%, 44\%, and 46\%. The
AVERAGE row is consistent with this picture: PROPOSED has the lowest
mean error (4.28), outperforming the second-best NNC by 2.01 units
(about a 32\% relative reduction when computed from the reported means)
and leaving larger margins against DNN, RBF, GENETIC, PRUNE, NEAT,
ADAM, and BFGS. Overall, PROPOSED exhibits consistent superiority
in terms of error as reflected by mean and median values, average
rank, and the per-dataset win counts, with the most notable exceptions
confined to a few datasets that appear to have distinct error scales.}

\begin{table}[H]
\caption{Experimental results using a variety of machine learning methods on
the regression datasets.\label{tab:expRegression}}

\centering{}{\footnotesize{}%
\begin{tabular}{|c|c|c|c|c|c|c|c|c|c|}
\hline 
{\footnotesize DATASET} & {\footnotesize ADAM} & {\footnotesize BFGS} & {\footnotesize GENETIC} & {\footnotesize RBF} & {\footnotesize NEAT} & {\footnotesize PRUNE} & {\footnotesize DNN} & {\footnotesize NNC} & {\footnotesize PROPOSED}\tabularnewline
\hline 
\hline 
{\footnotesize ABALONE} & {\footnotesize 4.30} & {\footnotesize 5.69} & {\footnotesize 7.17} & {\footnotesize 7.37} & {\footnotesize 9.88} & {\footnotesize 7.88} & {\footnotesize 6.91} & {\footnotesize 5.08} & {\footnotesize 4.47}\tabularnewline
\hline 
{\footnotesize AIRFOIL} & {\footnotesize 0.005} & {\footnotesize 0.003} & {\footnotesize 0.003} & {\footnotesize 0.27} & {\footnotesize 0.067} & {\footnotesize 0.002} & {\footnotesize 0.004} & {\footnotesize 0.004} & {\footnotesize 0.002}\tabularnewline
\hline 
{\footnotesize AUTO} & {\footnotesize 70.84} & {\footnotesize 60.97} & {\footnotesize 12.18} & {\footnotesize 17.87} & {\footnotesize 56.06} & {\footnotesize 75.59} & {\footnotesize 13.26} & {\footnotesize 17.13} & {\footnotesize 9.09}\tabularnewline
\hline 
{\footnotesize BK} & {\footnotesize 0.0252} & {\footnotesize 0.28} & {\footnotesize 0.027} & {\footnotesize 0.02} & {\footnotesize 0.15} & {\footnotesize 0.027} & {\footnotesize 0.02} & {\footnotesize 0.10} & {\footnotesize 0.023}\tabularnewline
\hline 
{\footnotesize BL} & {\footnotesize 0.622} & {\footnotesize 2.55} & {\footnotesize 5.74} & {\footnotesize 0.013} & {\footnotesize 0.05} & {\footnotesize 0.027} & {\footnotesize 0.006} & {\footnotesize 1.19} & {\footnotesize 0.001}\tabularnewline
\hline 
{\footnotesize BASEBALL} & {\footnotesize 77.90} & {\footnotesize 119.63} & {\footnotesize 103.60} & {\footnotesize 93.02} & {\footnotesize 100.39} & {\footnotesize 94.50} & {\footnotesize 110.22} & {\footnotesize 61.57} & {\footnotesize 48.13}\tabularnewline
\hline 
{\footnotesize CONCRETE} & {\footnotesize 0.078} & {\footnotesize 0.066} & {\footnotesize 0.0099} & {\footnotesize 0.011} & {\footnotesize 0.081} & {\footnotesize 0.0077} & {\footnotesize 0.021} & {\footnotesize 0.008} & {\footnotesize 0.005}\tabularnewline
\hline 
{\footnotesize DEE} & {\footnotesize 0.63} & {\footnotesize 2.36} & {\footnotesize 1.013} & {\footnotesize 0.17} & {\footnotesize 1.512} & {\footnotesize 1.08} & {\footnotesize 0.31} & {\footnotesize 0.26} & {\footnotesize 0.22}\tabularnewline
\hline 
{\footnotesize FRIEDMAN} & {\footnotesize 22.90} & {\footnotesize 1.263} & {\footnotesize 1.249} & {\footnotesize 7.23} & {\footnotesize 19.35} & {\footnotesize 8.69} & {\footnotesize 2.75} & {\footnotesize 6.29} & {\footnotesize 5.34}\tabularnewline
\hline 
{\footnotesize FY} & {\footnotesize 0.038} & {\footnotesize 0.19} & {\footnotesize 0.65} & {\footnotesize 0.041} & {\footnotesize 0.08} & {\footnotesize 0.042} & {\footnotesize 0.039} & {\footnotesize 0.11} & {\footnotesize 0.043}\tabularnewline
\hline 
{\footnotesize HO} & {\footnotesize 0.035} & {\footnotesize 0.62} & {\footnotesize 2.78} & {\footnotesize 0.03} & {\footnotesize 0.169} & {\footnotesize 0.03} & {\footnotesize 0.026} & {\footnotesize 0.015} & {\footnotesize 0.016}\tabularnewline
\hline 
{\footnotesize HOUSING} & {\footnotesize 80.99} & {\footnotesize 97.38} & {\footnotesize 43.26} & {\footnotesize 57.68} & {\footnotesize 56.49} & {\footnotesize 52.25} & {\footnotesize 65.18} & {\footnotesize 25.47} & {\footnotesize 15.47}\tabularnewline
\hline 
{\footnotesize LASER} & {\footnotesize 0.03} & {\footnotesize 0.015} & {\footnotesize 0.59} & {\footnotesize 0.03} & {\footnotesize 0.084} & {\footnotesize 0.007} & {\footnotesize 0.045} & {\footnotesize 0.025} & {\footnotesize 0.0049}\tabularnewline
\hline 
{\footnotesize LW} & {\footnotesize 0.028} & {\footnotesize 2.98} & {\footnotesize 1.90} & {\footnotesize 0.03} & {\footnotesize 0.03} & {\footnotesize 0.02} & {\footnotesize 0.023} & {\footnotesize 0.011} & {\footnotesize 0.011}\tabularnewline
\hline 
{\footnotesize MORTGAGE} & {\footnotesize 9.24} & {\footnotesize 8.23} & {\footnotesize 2.41} & {\footnotesize 1.45} & {\footnotesize 14.11} & {\footnotesize 12.96} & {\footnotesize 9.74} & {\footnotesize 0.30} & {\footnotesize 0.023}\tabularnewline
\hline 
{\footnotesize PL} & {\footnotesize 0.117} & {\footnotesize 0.29} & {\footnotesize 0.29} & {\footnotesize 2.118} & {\footnotesize 0.09} & {\footnotesize 0.032} & {\footnotesize 0.056} & {\footnotesize 0.047} & {\footnotesize 0.029}\tabularnewline
\hline 
{\footnotesize PLASTIC} & {\footnotesize 11.71} & {\footnotesize 20.32} & {\footnotesize 2.791} & {\footnotesize 8.62} & {\footnotesize 20.77} & {\footnotesize 17.33} & {\footnotesize 3.82} & {\footnotesize 4.20} & {\footnotesize 2.17}\tabularnewline
\hline 
{\footnotesize QUAKE} & {\footnotesize 0.07} & {\footnotesize 0.42} & {\footnotesize 0.04} & {\footnotesize 0.07} & {\footnotesize 0.298} & {\footnotesize 0.04} & {\footnotesize 0.098} & {\footnotesize 0.96} & {\footnotesize 0.036}\tabularnewline
\hline 
{\footnotesize SN} & {\footnotesize 0.026} & {\footnotesize 0.40} & {\footnotesize 2.95} & {\footnotesize 0.027} & {\footnotesize 0.174} & {\footnotesize 0.032} & {\footnotesize 0.027} & {\footnotesize 0.026} & {\footnotesize 0.024}\tabularnewline
\hline 
{\footnotesize STOCK} & {\footnotesize 180.89} & {\footnotesize 302.43} & {\footnotesize 3.88} & {\footnotesize 12.23} & {\footnotesize 12.23} & {\footnotesize 39.08} & {\footnotesize 12.95} & {\footnotesize 8.92} & {\footnotesize 4.69}\tabularnewline
\hline 
{\footnotesize TREASURY} & {\footnotesize 11.16} & {\footnotesize 9.91} & {\footnotesize 2.93} & {\footnotesize 2.02} & {\footnotesize 15.52} & {\footnotesize 13.76} & {\footnotesize 11.41} & {\footnotesize 0.43} & {\footnotesize 0.068}\tabularnewline
\hline 
{\footnotesize\textbf{AVERAGE}} & {\footnotesize\textbf{22.46}} & {\footnotesize\textbf{30.29}} & {\footnotesize\textbf{9.31}} & {\footnotesize\textbf{10.02}} & {\footnotesize\textbf{14.65}} & {\footnotesize\textbf{15.40}} & {\footnotesize\textbf{11.28}} & {\footnotesize\textbf{6.29}} & {\footnotesize\textbf{4.28}}\tabularnewline
\hline 
\end{tabular}}{\footnotesize\par}
\end{table}

\textcolor{red}{The statistical comparison depicted in Figure \ref{fig:statClass},
indicates that all pairwise comparisons between the current method
and the alternative models are highly statistically significant. Under
the conventional star notation, {*}{*}{*}{*} denotes p\textless 0.0001,
while the {*}{*}{*}{*}{*} flag for PROPOSED versus RBF signals even
stronger statistical evidence in favor of PROPOSED. Overall, the findings
confirm that PROPOSED consistently achieves lower error than every
comparator across the classification datasets, with a negligible likelihood
that these differences are due to chance.}

\begin{figure}[H]
\begin{centering}
\includegraphics[scale=0.75]{new1}
\par\end{centering}
\caption{Statistical comparison of the machine learning models for the classification
datasets.\label{fig:statClass}}

\end{figure}
\textcolor{red}{In Figure \ref{fig:statRegression}, the statistical
comparison on the regression datasets indicates that the proposed
model differs significantly from all alternative methods. Evidence
is particularly strong against ADAM, BFGS, RBF, PRUNE, and NNC (p={*},
i.e., p\ensuremath{\le}0.001), with the strongest signal observed
against NEAT (p=, i.e., p\ensuremath{\le}0.0001). Comparisons against
GENETIC and DNN are also statistically significant but comparatively
weaker (p=, i.e., p\ensuremath{\le}0.01). Taken together, the results
consistently support that the proposed model attains lower error than
every competing approach across the examined datasets, with a very
low probability that the observed differences are due to chance. Note
that significance levels reflect the strength of statistical evidence
rather than effect magnitude; for a fuller interpretation it is advisable
to also report absolute or relative error differences and suitable
effect size metrics.}

\begin{figure}[H]
\begin{centering}
\includegraphics[scale=0.75]{new2}
\par\end{centering}
\caption{Statistical comparison between the used methods for the regression
datasets.\label{fig:statRegression}}
\end{figure}


\subsection{Experiments with different crossover mechanism}

Also, in order to illustrated the robustness of the proposed method,
another experiment was conducted were the uniform crossover procedure
was used for the Neural Network Construction method and the proposed
one instead of the one - point crossover. The experimental results
for the classification datasets are shown in Table \ref{tab:experClassUniformOnePointVsCrossover}and
for regression datasets in Table \ref{tab:expersRegressionOnePointVsUniform}.
\textcolor{red}{Although, the one - point crossover mechanism is proposed
as the crossover procedure in the original article of Grammatical
Evolution.}

\begin{table}[H]
\caption{Experimental results for the classification datasets were a comparison
is made against the original one - point crossover method and the
uniform crossover procedure.\label{tab:experClassUniformOnePointVsCrossover}}

\raggedright{}{\footnotesize{}%
\begin{tabular}{|c|c|c|c|c|}
\hline 
{\scriptsize DATASET} & {\scriptsize NNC ONE-POINT} & {\scriptsize NNC -UNIFORM} & {\scriptsize PROPOSED ONE-POINT} & {\scriptsize PROPOSED UNIFORM}\tabularnewline
\hline 
\hline 
{\footnotesize APPENDICITIS} & {\footnotesize 14.40\%} & {\footnotesize 14.20\%} & {\footnotesize 14.30\%} & {\footnotesize 14.40\%}\tabularnewline
\hline 
{\footnotesize ALCOHOL} & {\footnotesize 37.72\%} & {\footnotesize 42.34\%} & {\footnotesize 35.60\%} & {\footnotesize 39.70\%}\tabularnewline
\hline 
{\footnotesize AUSTRALIAN} & {\footnotesize 14.46\%} & {\footnotesize 14.13\%} & {\footnotesize 14.55\%} & {\footnotesize 14.35\%}\tabularnewline
\hline 
{\footnotesize BALANCE} & {\footnotesize 23.65\%} & {\footnotesize 20.73\%} & {\footnotesize 7.84\%} & {\footnotesize 7.61\%}\tabularnewline
\hline 
{\footnotesize CLEVELAND} & {\footnotesize 50.93\%} & {\footnotesize 51.45\%} & {\footnotesize 46.41\%} & {\footnotesize 46.28\%}\tabularnewline
\hline 
{\footnotesize CIRCULAR} & {\footnotesize 12.66\%} & {\footnotesize 17.59\%} & {\footnotesize 6.92\%} & {\footnotesize 11.86\%}\tabularnewline
\hline 
{\footnotesize DERMATOLOGY} & {\footnotesize 21.54\%} & {\footnotesize 30.09\%} & {\footnotesize 20.54\%} & {\footnotesize 26.86\%}\tabularnewline
\hline 
{\footnotesize ECOLI} & {\footnotesize 49.88\%} & {\footnotesize 48.12\%} & {\footnotesize 48.82\%} & {\footnotesize 48.88\%}\tabularnewline
\hline 
{\footnotesize GLASS} & {\footnotesize 56.09\%} & {\footnotesize 57.43\%} & {\footnotesize 53.52\%} & {\footnotesize 52.43\%}\tabularnewline
\hline 
{\footnotesize HABERMAN} & {\footnotesize 27.53\%} & {\footnotesize 27.17\%} & {\footnotesize 26.80\%} & {\footnotesize 26.70\%}\tabularnewline
\hline 
{\footnotesize HAYES-ROTH} & {\footnotesize 33.69\%} & {\footnotesize 36.61\%} & {\footnotesize 31.00\%} & {\footnotesize 33.62\%}\tabularnewline
\hline 
{\footnotesize HEART} & {\footnotesize 15.67\%} & {\footnotesize 16.41\%} & {\footnotesize 15.45\%} & {\footnotesize 14.96\%}\tabularnewline
\hline 
{\footnotesize HEARTATTACK} & {\footnotesize 20.87\%} & {\footnotesize 21.50\%} & {\footnotesize 21.77\%} & {\footnotesize 21.27\%}\tabularnewline
\hline 
{\footnotesize HOUSEVOTES} & {\footnotesize 3.17\%} & {\footnotesize 3.44\%} & {\footnotesize 3.78\%} & {\footnotesize 3.43\%}\tabularnewline
\hline 
{\footnotesize IONOSPHERE} & {\footnotesize 11.29\%} & {\footnotesize 11.80\%} & {\footnotesize 11.94\%} & {\footnotesize 11.77\%}\tabularnewline
\hline 
{\footnotesize LIVERDISORDER} & {\footnotesize 32.35\%} & {\footnotesize 32.65\%} & {\footnotesize 31.32\%} & {\footnotesize 32.32\%}\tabularnewline
\hline 
{\footnotesize LYMOGRAPHY} & {\footnotesize 25.29\%} & {\footnotesize 28.21\%} & {\footnotesize 23.72\%} & {\footnotesize 25.14\%}\tabularnewline
\hline 
{\footnotesize MAMMOGRAPHIC} & {\footnotesize 17.62\%} & {\footnotesize 18.04\%} & {\footnotesize 16.74\%} & {\footnotesize 16.24\%}\tabularnewline
\hline 
{\footnotesize PARKINSONS} & {\footnotesize 12.74\%} & {\footnotesize 11.63\%} & {\footnotesize 12.63\%} & {\footnotesize 12.74\%}\tabularnewline
\hline 
{\footnotesize PHONEME} & {\footnotesize 22.50\%} & {\footnotesize 23.46\%} & {\footnotesize 21.52\%} & {\footnotesize 21.32\%}\tabularnewline
\hline 
{\footnotesize PIMA} & {\footnotesize 28.07\%} & {\footnotesize 27.95\%} & {\footnotesize 23.34\%} & {\footnotesize 24.43\%}\tabularnewline
\hline 
{\footnotesize POPFAILURES} & {\footnotesize 6.98\%} & {\footnotesize 6.80\%} & {\footnotesize 5.72\%} & {\footnotesize 6.01\%}\tabularnewline
\hline 
{\footnotesize REGIONS2} & {\footnotesize 26.18\%} & {\footnotesize 25.71\%} & {\footnotesize 23.81\%} & {\footnotesize 25.21\%}\tabularnewline
\hline 
{\footnotesize SAHEART} & {\footnotesize 29.80\%} & {\footnotesize 30.52\%} & {\footnotesize 28.04\%} & {\footnotesize 29.13\%}\tabularnewline
\hline 
{\footnotesize SEGMENT} & {\footnotesize 53.50\%} & {\footnotesize 54.78\%} & {\footnotesize 48.20\%} & {\footnotesize 52.26\%}\tabularnewline
\hline 
{\footnotesize SPIRAL} & {\footnotesize 48.01\%} & {\footnotesize 48.35\%} & {\footnotesize 44.95\%} & {\footnotesize 45.03\%}\tabularnewline
\hline 
{\footnotesize STATHEART} & {\footnotesize 18.08\%} & {\footnotesize 18.85\%} & {\footnotesize 17.93\%} & {\footnotesize 18.59\%}\tabularnewline
\hline 
{\footnotesize STUDENT} & {\footnotesize 6.70\%} & {\footnotesize 6.15\%} & {\footnotesize 4.05\%} & {\footnotesize 4.10\%}\tabularnewline
\hline 
{\footnotesize TRANSFUSION} & {\footnotesize 25.77\%} & {\footnotesize 25.58\%} & {\footnotesize 23.16\%} & {\footnotesize 23.96\%}\tabularnewline
\hline 
{\footnotesize WDBC} & {\footnotesize 7.36\%} & {\footnotesize 8.07\%} & {\footnotesize 4.95\%} & {\footnotesize 6.31\%}\tabularnewline
\hline 
{\footnotesize WINE} & {\footnotesize 13.59\%} & {\footnotesize 14.41\%} & {\footnotesize 9.94\%} & {\footnotesize 11.76\%}\tabularnewline
\hline 
{\footnotesize Z\_F\_S} & {\footnotesize 14.53\%} & {\footnotesize 18.33\%} & {\footnotesize 7.97\%} & {\footnotesize 10.13\%}\tabularnewline
\hline 
{\footnotesize Z\_O\_N\_F\_S} & {\footnotesize 48.62\%} & {\footnotesize 51.10\%} & {\footnotesize 39.28\%} & {\footnotesize 44.90\%}\tabularnewline
\hline 
{\footnotesize ZO\_NF\_S} & {\footnotesize 13.54\%} & {\footnotesize 14.52\%} & {\footnotesize 6.94\%} & {\footnotesize 8.24\%}\tabularnewline
\hline 
{\footnotesize ZONF\_S} & {\footnotesize 2.64\%} & {\footnotesize 2.82\%} & {\footnotesize 2.60\%} & {\footnotesize 2.78\%}\tabularnewline
\hline 
{\footnotesize ZOO} & {\footnotesize 8.70\%} & {\footnotesize 10.40\%} & {\footnotesize 6.60\%} & {\footnotesize 8.70\%}\tabularnewline
\hline 
{\footnotesize\textbf{AVERAGE}} & {\footnotesize\textbf{24.79\%}} & {\footnotesize\textbf{24.76\%}} & {\footnotesize\textbf{21.18\%}} & {\footnotesize\textbf{22.32\%}}\tabularnewline
\hline 
\end{tabular}}{\footnotesize\par}
\end{table}
\begin{table}[H]

\caption{Experimental results for the regression datasets using two crossover
methods: the one point crossover and the uniform crossover. \label{tab:expersRegressionOnePointVsUniform}}

\centering{}{\footnotesize{}%
\begin{tabular}{|c|c|c|c|c|}
\hline 
{\footnotesize DATASET} & {\footnotesize NNC ONE-POINT} & {\footnotesize NNC UNIFORM} & {\footnotesize PROPOSED ONE-POINT} & {\footnotesize PROPOSED UNIFORM}\tabularnewline
\hline 
\hline 
{\footnotesize ABALONE} & {\footnotesize 5.08} & {\footnotesize 5.40} & {\footnotesize 4.47} & {\footnotesize 4.55}\tabularnewline
\hline 
{\footnotesize AIRFOIL} & {\footnotesize 0.004} & {\footnotesize 0.004} & {\footnotesize 0.002} & {\footnotesize 0.003}\tabularnewline
\hline 
{\footnotesize AUTO} & {\footnotesize 17.13} & {\footnotesize 20.06} & {\footnotesize 9.09} & {\footnotesize 11.10}\tabularnewline
\hline 
{\footnotesize BK} & {\footnotesize 0.10} & {\footnotesize 0.018} & {\footnotesize 0.023} & {\footnotesize 0.018}\tabularnewline
\hline 
{\footnotesize BL} & {\footnotesize 1.19} & {\footnotesize 0.018} & {\footnotesize 0.001} & {\footnotesize 0.001}\tabularnewline
\hline 
{\footnotesize BASEBALL} & {\footnotesize 61.57} & {\footnotesize 63.44} & {\footnotesize 48.13} & {\footnotesize 49.99}\tabularnewline
\hline 
{\footnotesize CONCRETE} & {\footnotesize 0.008} & {\footnotesize 0.009} & {\footnotesize 0.005} & {\footnotesize 0.006}\tabularnewline
\hline 
{\footnotesize DEE} & {\footnotesize 0.26} & {\footnotesize 0.28} & {\footnotesize 0.22} & {\footnotesize 0.24}\tabularnewline
\hline 
{\footnotesize FRIEDMAN} & {\footnotesize 6.29} & {\footnotesize 6.98} & {\footnotesize 5.34} & {\footnotesize 5.85}\tabularnewline
\hline 
{\footnotesize FY} & {\footnotesize 0.11} & {\footnotesize 0.04} & {\footnotesize 0.043} & {\footnotesize 0.04}\tabularnewline
\hline 
{\footnotesize HO} & {\footnotesize 0.015} & {\footnotesize 0.016} & {\footnotesize 0.016} & {\footnotesize 0.011}\tabularnewline
\hline 
{\footnotesize HOUSING} & {\footnotesize 25.47} & {\footnotesize 26.68} & {\footnotesize 15.47} & {\footnotesize 16.89}\tabularnewline
\hline 
{\footnotesize LASER} & {\footnotesize 0.025} & {\footnotesize 0.041} & {\footnotesize 0.0049} & {\footnotesize 0.008}\tabularnewline
\hline 
{\footnotesize LW} & {\footnotesize 0.011} & {\footnotesize 0.012} & {\footnotesize 0.011} & {\footnotesize 0.011}\tabularnewline
\hline 
{\footnotesize MORTGAGE} & {\footnotesize 0.30} & {\footnotesize 0.29} & {\footnotesize 0.023} & {\footnotesize 0.037}\tabularnewline
\hline 
{\footnotesize PL} & {\footnotesize 0.047} & {\footnotesize 0.046} & {\footnotesize 0.029} & {\footnotesize 0.024}\tabularnewline
\hline 
{\footnotesize PLASTIC} & {\footnotesize 4.20} & {\footnotesize 5.20} & {\footnotesize 2.17} & {\footnotesize 2.30}\tabularnewline
\hline 
{\footnotesize QUAKE} & {\footnotesize 0.96} & {\footnotesize 0.036} & {\footnotesize 0.036} & {\footnotesize 0.036}\tabularnewline
\hline 
{\footnotesize SN} & {\footnotesize 0.026} & {\footnotesize 0.026} & {\footnotesize 0.024} & {\footnotesize 0.024}\tabularnewline
\hline 
{\footnotesize STOCK} & {\footnotesize 8.92} & {\footnotesize 10.89} & {\footnotesize 4.69} & {\footnotesize 8.31}\tabularnewline
\hline 
{\footnotesize TREASURY} & {\footnotesize 0.43} & {\footnotesize 0.38} & {\footnotesize 0.068} & {\footnotesize 0.072}\tabularnewline
\hline 
{\footnotesize\textbf{AVERAGE}} & {\footnotesize\textbf{6.29}} & {\footnotesize\textbf{6.66}} & {\footnotesize\textbf{4.28}} & {\footnotesize\textbf{4.74}}\tabularnewline
\hline 
\end{tabular}}{\footnotesize\par}
\end{table}
Analysis of the results in Table \ref{tab:experClassUniformOnePointVsCrossover}
demonstrates that the proposed method systematically outperforms both
variants of NNC, regardless of the crossover type employed. Specifically,
the proposed method with one-point crossover achieves a significantly
lower average error rate (21.18\%) compared to NNC (24.79\%). A similar
performance gap is observed with uniform crossover, where the proposed
method maintains superiority (22.32\% vs NNC's 24.76\%). The advantage
is particularly pronounced in several datasets: for BALANCE (7.61-7.84\%
vs 20.73-23.65\%), CIRCULAR (6.92-11.86\% vs 12.66-17.59\%), Z\_F\_S
(7.97-10.13\% vs 14.53-18.33\%), and ZO\_NF\_S (6.94-8.24\% vs 13.54-14.52\%).
Notably, in datasets like WDBC and WINE, the proposed method reduces
the error by nearly half compared to NNC. Interestingly, the choice
of crossover type shows minimal impact on performance for both methods.
The marginal differences between one-point and uniform crossover (with
average errors remaining stable for each method) suggest that the
proposed method's superiority stems from its fundamental architecture
rather than the recombination technique. These experimental results
confirm the robustness of the proposed method, which maintains consistent
superiority across various classification datasets while significantly
reducing average error rates compared to standard NNC approaches.

Table \ref{tab:expersRegressionOnePointVsUniform} further validates
the proposed method's superiority for regression datasets. The proposed
method achieves lower average errors with both one-point (4.28) and
uniform crossover (4.74) compared to NNC (6.29 and 6.66 respectively).
The performance gap is especially notable in key datasets: AUTO (9.09-11.1
vs 17.13-20.06), HOUSING (15.47-16.89 vs 25.47-26.68), and PLASTIC
(2.17-2.3 vs 4.2-5.2). In several cases (AIRFOIL, LASER, MORTGAGE),
the proposed method reduces errors to a fraction of NNC's values.
Particularly impressive results appear in BL (0.001) and QUAKE (0.036),
where the proposed method achieves remarkably low, crossover-invariant
errors. The crossover type shows slightly less impact on the proposed
method (average difference of 0.46) than on NNC (0.37 difference),
though one-point crossover yields marginally better results for both.
These comprehensive results confirm that the proposed method maintains
its superiority in regression problems, delivering consistently and
significantly improved performance over NNC. Its ability to achieve
lower errors across diverse problems, independent of crossover selection,
solidifies its position as a more reliable and effective approach.

\subsection{Experiments with the critical parameter $N_{K}$}

Another experiment was conducted where the parameter $N_{K}$ was
altered in the range $[5,\ldots,50]$ and the results for the regression
datasets are depicted in Figure \ref{fig:nkExpers}.

\begin{figure}[H]
\begin{centering}
\includegraphics[scale=0.5]{nk_exper}
\par\end{centering}
\caption{Average regression error for the regression datasets and the proposed
method using a variety of values for the parameter $N_{K}$.\label{fig:nkExpers}}

\end{figure}
Additionally a series of experiments was conducted where the parameter
$N_{I}$ was changed from 10 to 40 and the results are graphically
presented in Figure \ref{fig:expersNI}.
\begin{figure}[H]
\begin{centering}
\includegraphics[scale=0.5]{ni_exper}
\par\end{centering}
\caption{Experimental results for the regression datasets and the proposed
method using a variety of values for the parameter $N_{I}$.\label{fig:expersNI}}

\end{figure}
 This figure presents the relationship between the number of chromosomes
($N_{I}$) participating in the secondary genetic algorithm and the
resulting regression error. We observe that as the number of chromosomes
increases, the error decreases, indicating improvement in the model's
performance. Specifically, for $N_{I}=5$, the error is 4.99, while
for $N_{I}=10$, the error drops to 4.89. This trend continues with
further increase in chromosomes: for $N_{I}=20$, the error reaches
4.27, and for $N_{I}=40$, the error reaches 4.18. This error reduction
shows that using more chromosomes in the secondary genetic algorithm
leads to better optimization of the neural network's parameters, resulting
in error minimization. However, the improvement is not linear. We
observe that the difference in error between $N_{I}=5$ and $N_{I}=10$
is 0.10, while between $N_{I}=20$ and $N_{I}=40$ it is only 0.09.
This may indicate that beyond a certain point, increasing chromosomes
has progressively smaller impact on error reduction. This phenomenon
may be due to factors such as algorithm convergence or the existence
of an optimization threshold beyond which improvement becomes more
difficult. Furthermore, the selection of $N_{I}$ may be influenced
by computational constraints. Using more chromosomes increases the
computational load, so the performance improvement must be balanced
against resource costs. For example, transitioning from $N_{I}=20$
to $N_{I}=40$ leads to error reduction of only 0.09, which may not
justify the doubling of computational cost in certain scenarios. In
summary, the table confirms that increasing the number of chromosomes
improves the model's performance, but its effect becomes smaller as
$N_{I}$ grows larger. This means that the optimal selection of $N_{I}$
depends on a combination of factors, such as the desired accuracy,
available computational resources, and the nature of the problem. 

\textcolor{red}{Also, in Figure \ref{fig:timeNI} the average execution
time for the regression dataset is plotted. In this graph the original
neural network construction method is depicted as well as the proposed
one using a series of values for the parameter $N_{I}$. As is was
expected, the execution time increases when the parameter $N_{I}$
is increased.}

\begin{figure}[H]
\begin{centering}
\includegraphics[scale=0.75]{nnc_ga_time}
\par\end{centering}
\caption{Average execution time for the regression datasets using the original
neural network construction method and the proposed one and different
values of parameter $N_{I}$.\label{fig:timeNI}}

\end{figure}


\subsection{A series of practical examples}

As practical applications of the proposed method to real - world problems
we consider two cases from the recent bibliography. In the first case,
consider the prediction of the duration of forest fires as presented
for the Greek territory in a recent publication \citep{ml_fire}.
Using data from the Greek Fire Service, an attempt is made to predict
the duration of forest fires for the years 2014-2023. Figure \ref{fig:mlFire}
depicts a comparison for the classification error for this problem
for the years 2014-2023 between the original Neural Network Construction
method, denoted as NNC and the proposed method which is denoted as
NNC\_GA in the plot. 
\begin{figure}[H]
\begin{centering}
\includegraphics[scale=0.5]{ml_fire}
\par\end{centering}
\caption{Comparison of the original NNC method and the proposed modification
(NNC\_GA) for the prediction of forest fires for the Greek teritory.
The horizontal axis denotes the year and the vertical the obtained
classification error.\label{fig:mlFire}}

\end{figure}
 As is evident, the proposed method has a lower error in estimating
the duration of forest fires in all years from 2014 to 2023 compared
to the original artificial neural network construction technique.
\textcolor{red}{Although, the proposed method requires significantly
more time than the original method as depicted in Figure \ref{fig:fire_time}.}

\begin{figure}[H]
\begin{centering}
\includegraphics[scale=0.75]{forest_time}
\par\end{centering}
\caption{Average execution time for the fores fires problem, using the original
neural network construction method and the proposed one.\label{fig:fire_time}}

\end{figure}

The second practical example is the PIRvision dataset \citep{pirvision},
that contains occupancy detection data that was collected from a Synchronized
Low-Energy Electronically-chopped Passive Infra-Red sensing node in
residential and office environments.\textbf{ }The dataset contains
15302 patterns and the dimension of each pattern is 59. The experimental
results validated with ten - fold cross validation, using a series
of methods and the proposed one are depicted in Figure \ref{fig:pirvision}.

\begin{figure}[H]
\begin{centering}
\includegraphics[scale=0.75]{pirvision}
\par\end{centering}
\caption{Average classification error for the PIRvision dataset using a series
of machine learning methods.\label{fig:pirvision}}

\end{figure}
It is evident that the proposed modification of the neural network
construction method outperforms significantly the other techniques
in terms of average classification error for this particular dataset.

\section{Discussion \label{sec:Discussion}}

This study presents an interesting approach combining grammatical
evolution with modified genetic algorithms for constructing artificial
neural networks. However, a comprehensive analysis of the results
and practical implications reveals several aspects that require further
investigation and critical examination. While the experimental findings
demonstrate certain improvements over traditional techniques, the
interpretation and significance of these improvements have not been
analyzed with the depth and critical thinking required for a complete
method evaluation. Regarding classification performance, the method
shows an average error rate of 21.18\% compared to ADAM's 36.45\%
and BFGS's 35.71\%. However, these comparative metrics conceal significant
performance variations across different datasets. For instance, on
Cleveland and Ecoli datasets, classification error reaches 46.41\%
and 48.82\% respectively, while on Housevotes and Zoo it drops below
7\%. This substantial performance variation suggests the method may
be highly sensitive to dataset-specific characteristics, which isn't
sufficiently analyzed in the results presentation. Furthermore, the
lack of analysis regarding variation across the 30 repetitions of
each experiment raises questions about the method's stability and
reliability in real-world applications.The statistical significance
of results, while supported by extremely low p-values (1.9e-07 to
1.1e-08), doesn't account for the dynamics of different problem types.
In noisy datasets or those with significant class imbalance like Haberman
and Liverdisorder, the method shows notable performance fluctuations
that remain unexplained. Additionally, the absence of analysis regarding
dataset characteristics affecting performance (such as dimensionality,
sample size, or degree of linear separability) makes it difficult
to determine the optimal conditions for the method's application.Computational
resources and execution times present another critical but underexplored
issue. 

While the study mentions using an AMD Ryzen 5950X system with 128GB
RAM, comprehensive reporting of computational requirements is missing.
Specifically, it would be essential to present average training times
per dataset category (classification vs regression), the method's
scalability regarding number of features and samples, memory consumption
during the grammatical evolution process, and the impact of various
parameters (like population size and generation count) on execution
times. This lack of information makes practical implementation assessment
challenging, especially for real-world problems where computational
resources and time constraints are crucial factors. Regarding limitations,
while the study acknowledges issues like local optima and overfitting,
their analysis remains superficial. For example, in datasets like
Z\_O\_N\_F\_S with 39.28\% error, it's not investigated whether this
results from insufficient solution space exploration due to grammatical
evolution parameters, limitations in the grammar used for architecture
generation, excessive network complexity leading to overfitting, or
inadequacies in parameter training mechanisms.Practical application
and robustness require more thorough examination. Beyond controlled
experimental scenarios, there's missing information about the ease
of applying the method to real-world, unprocessed datasets, the required
expertise for optimal parameter tuning, the method's resilience to
noisy, incomplete or imbalanced data, and the interpretability of
results and generated architectures.Moreover, comparisons with contemporary
approaches like transformers, convolutional neural networks, or reinforcement
learning methods in domains where they dominate (e.g., natural language
processing, computer vision, robotics) are completely absent from
the study. This evaluation gap significantly limits our understanding
of the method's relative value compared to state-of-the-art alternatives.
The method's generalizability to new application domains hasn't been
adequately explored. While results are presented across various fields
(medicine, physics, economics), critical information is missing about
the flexibility and adaptability of the used grammar across different
domains, required modifications for new data types (time series, graphs,
spatial data), knowledge transfer capability between different applications,
and domain knowledge requirements for appropriate grammar design.
In summary, while the proposed method introduces interesting mechanisms
for improving automated neural network design, this analysis reveals
numerous aspects needing further investigation. 

For a complete and objective evaluation, it would be necessary to
conduct a much more detailed analysis of result stability and variation,
comparisons with alternative contemporary approaches beyond the basic
techniques examined, thorough evaluation of scalability and computational
requirements for large datasets, in-depth investigation of real-world
implementation challenges and limitations, and analysis of generalization
and adaptation capability to new domains and data types. Only through
such a holistic and critical approach could we obtain a complete picture
of this methodology's value, capabilities and limitations. The current
results, while encouraging, leave significant gaps in our understanding
of how and under what conditions the method can truly provide value
compared to existing approaches in automated neural network design.

\section{Conclusions}

The article presents a method for constructing artificial neural networks
by integrating Grammatical Evolution (GE) with a modified Genetic
Algorithm (GA) to improve generalization properties and reduce overfitting.
The method proves to be effective in designing neural network architectures
and optimizing their parameters. The modified genetic algorithm avoids
local minima during training and addresses overfitting by applying
penalty factors to the fitness function, ensuring better generalization
to unseen data. The method was evaluated on a variety of classification
and regression datasets from diverse fields, including physics, chemistry,
medicine, and economics. Comparative results indicate that the proposed
method achieves lower error rates on average compared to traditional
optimization and machine learning techniques, highlighting its stability
and adaptability. The results, analyzed through statistical metrics
such as p-values, provide strong evidence of the method’s superiority
over competing models in both classification and regression tasks.
A key innovation of the method is the combination of dynamic architecture
generation and parameter optimization within a unified framework.
This approach not only enhances performance but also reduces the computational
complexity associated with manually designing neural networks. Additionally,
the use of constraint techniques in the genetic algorithm ensures
the preservation of the neural network structure while enabling controlled
optimization of parameters. Future explorations could focus on testing
the method on larger and more complex datasets, such as those encountered
in image recognition, natural language processing, and genomics, to
evaluate its scalability and effectiveness in real-world applications.
Furthermore, the integration of other global optimization methods,
such as Particle Swarm Optimization, Simulated Annealing, or Differential
Evolution, could be considered to further enhance the algorithm’s
robustness and convergence speed. Concurrently, the inclusion of regularization
techniques, such as dropout or batch normalization, could improve
the method’s generalization capabilities even further. Reducing computational
cost is another important area of investigation, and the method could
be adapted to leverage parallel computing architectures, such as GPUs
or distributed systems, making it feasible for training on large datasets
or for real-time applications. Finally, customizing the grammar used
in Grammatical Evolution based on the specific characteristics of
individual fields could improve the method’s performance in specialized
tasks, such as time-series forecasting or anomaly detection in cybersecurity.

The proposed technique attempts to maintain the parameters of artificial
neural networks within ranges of values \LyXZeroWidthSpace\LyXZeroWidthSpace in
which the neural network is likely to have good generalization properties.
A possible future improvement could be to find this interval of values
\LyXZeroWidthSpace\LyXZeroWidthSpace either with some technique that
utilizes derivatives or with interval arithmetic techniques \citep{interval1,interval2}.

\vspace{6pt}


\authorcontributions{V.C. and I.G.T. conducted the experiments, employing several datasets
and provided the comparative experiments. D.T. and V.C. performed
the statistical analysis and prepared the manuscript. All authors
have read and agreed to the published version of the manuscript.}

\funding{This research received no external funding.}

\institutionalreview{Not applicable.}

\institutionalreview{Not applicable.}

\institutionalreview{Not applicable.}

\acknowledgments{This research has been financed by the European Union : Next Generation
EU through the Program Greece 2.0 National Recovery and Resilience
Plan , under the call RESEARCH -- CREATE -- INNOVATE, project name
“iCREW: Intelligent small craft simulator for advanced crew training
using Virtual Reality techniques\textquotedbl{} (project code:TAEDK-06195).}

\conflictsofinterest{The authors declare no conflicts of interest.}

\begin{adjustwidth}{-\extralength}{0cm}{}

\reftitle{References}
\begin{thebibliography}{99}
\bibitem{nn1}Abiodun, O. I., Jantan, A., Omolara, A. E., Dada, K.
V., Mohamed, N. A., \& Arshad, H. (2018). State-of-the-art in artificial
neural network applications: A survey. Heliyon, 4(11).

\bibitem{nn2}Suryadevara, S., \& Yanamala, A. K. Y. (2021). A Comprehensive
Overview of Artificial Neural Networks: Evolution, Architectures,
and Applications. Revista de Inteligencia Artificial en Medicina,
12(1), 51-76.

\bibitem{nn_image}M. Egmont-Petersen, D. de Ridder, H. Handels, Image
processing with neural networks---a review, Pattern Recognition \textbf{35},
pp. 2279-2301, 2002.

\bibitem{nn_timeseries}G.Peter Zhang, Time series forecasting using
a hybrid ARIMA and neural network model, Neurocomputing \textbf{50},
pp. 159-175, 2003.

\bibitem{nn_credit}Z. Huang, H. Chen, C.-Jung Hsu, W.-Hwa Chen, S.
Wu, Credit rating analysis with support vector machines and neural
networks: a market comparative study, Decision Support Systems \textbf{37},
pp. 543-558, 2004.

\bibitem{nnphysics1}P. Baldi, K. Cranmer, T. Faucett et al, Parameterized
neural networks for high-energy physics, Eur. Phys. J. C \textbf{76},
2016.

\bibitem{nnphysics2}Baldi, P., Cranmer, K., Faucett, T., Sadowski,
P., \& Whiteson, D. (2016). Parameterized neural networks for high-energy
physics. The European Physical Journal C, 76(5), 1-7.

\bibitem{bpnn1}Vora, K., \& Yagnik, S. (2014). A survey on backpropagation
algorithms for feedforward neural networks. International Journal
of Engineering Development and Research, 1(3), 193-197.

\bibitem{bpnn2}K. Vora, S. Yagnik, A survey on backpropagation algorithms
for feedforward neural networks, International Journal of Engineering
Development and Research \textbf{1}, pp. 193-197, 2014.

\bibitem{rpropnn-1}Pajchrowski, T., Zawirski, K., \& Nowopolski,
K. (2014). Neural speed controller trained online by means of modified
RPROP algorithm. IEEE transactions on industrial informatics, 11(2),
560-568.

\bibitem{rpropnn-2}Hermanto, R. P. S., \& Nugroho, A. (2018). Waiting-time
estimation in bank customer queues using RPROP neural networks. Procedia
Computer Science, 135, 35-42.

\bibitem{nn_adam}D. P. Kingma, J. L. Ba, ADAM: a method for stochastic
optimization, in: Proceedings of the 3rd International Conference
on Learning Representations (ICLR 2015), pp. 1--15, 2015.

\bibitem{geneticnn1}Reynolds, J., Rezgui, Y., Kwan, A., \& Piriou,
S. (2018). A zone-level, building energy optimisation combining an
artificial neural network, a genetic algorithm, and model predictive
control. Energy, 151, 729-739.

\bibitem{psonn}Das, G., Pattnaik, P. K., \& Padhy, S. K. (2014).
Artificial neural network trained by particle swarm optimization for
non-linear channel equalization. Expert Systems with Applications,
41(7), 3491-3496.

\bibitem{nn_siman}Sexton, R. S., Dorsey, R. E., \& Johnson, J. D.
(1999). Beyond backpropagation: using simulated annealing for training
neural networks. Journal of Organizational and End User Computing
(JOEUC), 11(3), 3-10.

\bibitem{weight_de1}Wang, L., Zeng, Y., \& Chen, T. (2015). Back
propagation neural network with adaptive differential evolution algorithm
for time series forecasting. Expert Systems with Applications, 42(2),
855-863.

\bibitem{nn_abc}Karaboga, D., \& Akay, B. (2007, June). Artificial
bee colony (ABC) algorithm on training artificial neural networks.
In 2007 IEEE 15th Signal Processing and Communications Applications
(pp. 1-4). IEEE.

\bibitem{tabunn}R.S. Sexton, B. Alidaee, R.E. Dorsey, J.D. Johnson,
Global optimization for artificial neural networks: A tabu search
application. European Journal of Operational Research \textbf{106},
pp. 570-584, 1998.

\bibitem{nn_hybrid}J.-R. Zhang, J. Zhang, T.-M. Lok, M.R. Lyu, A
hybrid particle swarm optimization--back-propagation algorithm for
feedforward neural network training, Applied Mathematics and Computation
\textbf{185}, pp. 1026-1037, 2007.

\bibitem{nn_cascade}G. Zhao, T. Wang, Y. Jin, C. Lang, Y. Li, H.
Ling, The Cascaded Forward algorithm for neural network training,
Pattern Recognition \textbf{161}, 111292, 2025.

\bibitem{nn_gpu1}K-Su Oh, K. Jung, GPU implementation of neural networks,
Pattern Recognition \textbf{37}, pp. 1311-1314, 2004.

\bibitem{nn_gpu2}M. Zhang, K. Hibi, J. Inoue, GPU-accelerated artificial
neural network potential for molecular dynamics simulation, Computer
Physics Communications \textbf{285}, 108655, 2023. 

\bibitem{nnsharing1}S.J. Nowlan and G.E. Hinton, Simplifying neural
networks by soft weight sharing, Neural Computation 4, pp. 473-493,
1992.

\bibitem{nnsharing2}Nowlan, S. J., \& Hinton, G. E. (2018). Simplifying
neural networks by soft weight sharing. In The mathematics of generalization
(pp. 373-394). CRC Press.

\bibitem{nnprunning1}S.J. Hanson and L.Y. Pratt, Comparing biases
for minimal network construction with back propagation, In D.S. Touretzky
(Ed.), Advances in Neural Information Processing Systems, Volume 1,
pp. 177-185, San Mateo, CA: Morgan Kaufmann, 1989.

\bibitem{nnprunning2}M. Augasta and T. Kathirvalavakumar, Pruning
algorithms of neural networks --- a comparative study, Central European
Journal of Computer Science, 2003.

\bibitem{nnearly1}Lutz Prechelt, Automatic early stopping using cross
validation: quantifying the criteria, Neural Networks \textbf{11},
pp. 761-767, 1998.

\bibitem{nnearly2}X. Wu and J. Liu, A New Early Stopping Algorithm
for Improving Neural Network Generalization, 2009 Second International
Conference on Intelligent Computation Technology and Automation, Changsha,
Hunan, 2009, pp. 15-18.

\bibitem{nndecay1}N. K. Treadgold and T. D. Gedeon, Simulated annealing
and weight decay in adaptive learning: the SARPROP algorithm,IEEE
Transactions on Neural Networks \textbf{9}, pp. 662-668, 1998.

\bibitem{nndecay2}M. Carvalho and T. B. Ludermir, Particle Swarm
Optimization of Feed-Forward Neural Networks with Weight Decay, 2006
Sixth International Conference on Hybrid Intelligent Systems (HIS'06),
Rio de Janeiro, Brazil, 2006, pp. 5-5.

\bibitem{nn_arch1}J. Arifovic, R. Gençay, Using genetic algorithms
to select architecture of a feedforward artificial neural network,
Physica A: Statistical Mechanics and its Applications \textbf{289},
pp. 574-594, 2001.

\bibitem{nn_arch2}P.G. Benardos, G.C. Vosniakos, Optimizing feedforward
artificial neural network architecture, Engineering Applications of
Artificial Intelligence \textbf{20}, pp. 365-382, 2007.

\bibitem{nn_arch3}B.A. Garro, R.A. Vázquez, Designing Artificial
Neural Networks Using Particle Swarm Optimization Algorithms, Computational
Intelligence and Neuroscience, 369298, 2015. 

\bibitem[(2001)]{nn_ereinf}Siebel, N. T., \& Sommer, G. (2007). Evolutionary
reinforcement learning of artificial neural networks. International
Journal of Hybrid Intelligent Systems, 4(3), 171-183.

\bibitem[(2001)]{nn_reinf}Jaafra, Y., Laurent, J. L., Deruyver, A.,
\& Naceur, M. S. (2019). Reinforcement learning for neural architecture
search: A review. Image and Vision Computing, 89, 57-66.

\bibitem[(2001)]{nn_param_sharing}Pham, H., Guan, M., Zoph, B., Le,
Q., \& Dean, J. (2018, July). Efficient neural architecture search
via parameters sharing. In International conference on machine learning
(pp. 4095-4104). PMLR.

\bibitem[(2001)]{nn_snas}Xie, S., Zheng, H., Liu, C., \& Lin, L.
(2018). SNAS: stochastic neural architecture search. arXiv preprint
arXiv:1812.09926.

\bibitem[(2001)]{nn_bayes}Zhou, H., Yang, M., Wang, J., \& Pan, W.
(2019, May). Bayesnas: A bayesian approach for neural architecture
search. In International conference on machine learning (pp. 7603-7613).
PMLR.

\bibitem[(2001)]{nn_drug}L. Terfloth, J. Gasteige, Neural networks
and genetic algorithms in drug design, Drug Discovery Today \textbf{6},
pp. 102-108, 2001.

\bibitem[(2001)]{nn_estimates}Kim, G. H., Seo, D. S., \& Kang, K.
I. (2005). Hybrid models of neural networks and genetic algorithms
for predicting preliminary cost estimates. Journal of computing in
civil engineering, 19(2), 208-211.

\bibitem[(2001)]{nn_solar}Kalogirou, S. A. (2004). Optimization of
solar systems using artificial neural-networks and genetic algorithms.
Applied Energy, 77(4), 383-405.

\bibitem[(2001)]{nn_feature}Tong, D.L., Mintram, R. Genetic Algorithm-Neural
Network (GANN): a study of neural network activation functions and
depth of genetic algorithm search applied to feature selection. Int.
J. Mach. Learn. \& Cyber. 1, 75--87 (2010).

\bibitem[(2001)]{nn_string}Ruehle, F. Evolving neural networks with
genetic algorithms to study the string landscape. J. High Energ. Phys.
2017, 38 (2017).

\bibitem{ge1}M. O’Neill, C. Ryan, Grammatical evolution, IEEE Trans.
Evol. Comput. \textbf{5,}pp. 349--358, 2001.

\bibitem{nnc}I.G. Tsoulos, D. Gavrilis, E. Glavas, Neural network
construction and training using grammatical evolution, Neurocomputing
\textbf{72}, pp. 269-277, 2008.

\bibitem{nnc_amide1}G.V. Papamokos, I.G. Tsoulos, I.N. Demetropoulos,
E. Glavas, Location of amide I mode of vibration in computed data
utilizing constructed neural networks, Expert Systems with Applications
\textbf{36}, pp. 12210-12213, 2009.

\bibitem{nnc_de}I.G. Tsoulos, D. Gavrilis, E. Glavas, Solving differential
equations with constructed neural networks, Neurocomputing \textbf{72},
pp. 2385-2391, 2009.

\bibitem{nnc_feas}I.G. Tsoulos, G. Mitsi, A. Stavrakoudis, S. Papapetropoulos,
Application of Machine Learning in a Parkinson's Disease Digital Biomarker
Dataset Using Neural Network Construction (NNC) Methodology Discriminates
Patient Motor Status, Frontiers in ICT 6, 10, 2019.

\bibitem{nnc_student}V. Christou, I.G. Tsoulos, V. Loupas, A.T. Tzallas,
C. Gogos, P.S. Karvelis, N. Antoniadis, E. Glavas, N. Giannakeas,
Performance and early drop prediction for higher education students
using machine learning, Expert Systems with Applications \textbf{225},
120079, 2023.

\bibitem{nnc_autism}E.I. Toki, J. Pange, G. Tatsis, K. Plachouras,
I.G. Tsoulos, Utilizing Constructed Neural Networks for Autism Screening,
Applied Sciences \textbf{14}, 3053, 2024.

\bibitem{bnf1}J. W. Backus. The Syntax and Semantics of the Proposed
International Algebraic Language of the Zurich ACM-GAMM Conference.
Proceedings of the International Conference on Information Processing,
UNESCO, 1959, pp.125-132.

\bibitem{ge_program1}C. Ryan, J. Collins, M. O’Neill, Grammatical
evolution: Evolving programs for an arbitrary language. In: Banzhaf,
W., Poli, R., Schoenauer, M., Fogarty, T.C. (eds) Genetic Programming.
EuroGP 1998. Lecture Notes in Computer Science, vol 1391. Springer,
Berlin, Heidelberg, 1998.

\bibitem{ge_program2}M. O’Neill, M., C. Ryan, Evolving Multi-line
Compilable C Programs. In: Poli, R., Nordin, P., Langdon, W.B., Fogarty,
T.C. (eds) Genetic Programming. EuroGP 1999. Lecture Notes in Computer
Science, vol 1598. Springer, Berlin, Heidelberg, 1999.

\bibitem{ge_music}A.O. Puente, R. S. Alfonso, M. A. Moreno, Automatic
composition of music by means of grammatical evolution, In: APL '02:
Proceedings of the 2002 conference on APL: array processing languages:
lore, problems, and applications July 2002 Pages 148--155. 

\bibitem{ge_pacman}E. Galván-López, J.M. Swafford, M. O’Neill, A.
Brabazon, Evolving a Ms. PacMan Controller Using Grammatical Evolution.
In: , et al. Applications of Evolutionary Computation. EvoApplications
2010. Lecture Notes in Computer Science, vol 6024. Springer, Berlin,
Heidelberg, 2010.

\bibitem{ge_supermario}N. Shaker, M. Nicolau, G. N. Yannakakis, J.
Togelius, M. O'Neill, Evolving levels for Super Mario Bros using grammatical
evolution, 2012 IEEE Conference on Computational Intelligence and
Games (CIG), 2012, pp. 304-31.

\bibitem{ge_energy}D. Martínez-Rodríguez, J. M. Colmenar, J. I. Hidalgo,
R.J. Villanueva Micó, S. Salcedo-Sanz, Particle swarm grammatical
evolution for energy demand estimation, Energy Science and Engineering
\textbf{8}, pp. 1068-1079, 2020.

\bibitem{ge_crypt}C. Ryan, M. Kshirsagar, G. Vaidya, G. et al. Design
of a cryptographically secure pseudo random number generator with
grammatical evolution. Sci Rep \textbf{12}, 8602, 2022.

\bibitem{ge_trading}C. Martín, D. Quintana, P. Isasi, Grammatical
Evolution-based ensembles for algorithmic trading, Applied Soft Computing
\textbf{84}, 105713, 2019.

\bibitem{nnt_bound}Anastasopoulos, N., Tsoulos, I.G., Karvounis,
E. et al. Locate the Bounding Box of Neural Networks with Intervals.
Neural Process Lett 52, 2241--2251 (2020). 

\bibitem[(1989)]{uci} M. Kelly, R. Longjohn, K. Nottingham, The UCI
Machine Learning Repository, https://archive.ics.uci.edu.

\bibitem{Keel}J. Alcalá-Fdez, A. Fernandez, J. Luengo, J. Derrac,
S. García, L. Sánchez, F. Herrera. KEEL Data-Mining Software Tool:
Data Set Repository, Integration of Algorithms and Experimental Analysis
Framework. Journal of Multiple-Valued Logic and Soft Computing 17,
pp. 255-287, 2011.

\bibitem{appendicitis}Weiss, Sholom M. and Kulikowski, Casimir A.,
Computer Systems That Learn: Classification and Prediction Methods
from Statistics, Neural Nets, Machine Learning, and Expert Systems,
Morgan Kaufmann Publishers Inc, 1991.

\bibitem[Tzimourta(2018)]{alcohol}Tzimourta, K.D.; Tsoulos, I.; Bilero,
I.T.; Tzallas, A.T.; Tsipouras, M.G.; Giannakeas, N. Direct Assessment
of Alcohol Consumption in Mental State Using Brain Computer Interfaces
and Grammatical Evolution. Inventions 2018, 3, 51.

\bibitem[Quinlan(2018)]{australian}J.R. Quinlan, Simplifying Decision
Trees. International Journal of Man-Machine Studies \textbf{27}, pp.
221-234, 1987. 

\bibitem{balance}T. Shultz, D. Mareschal, W. Schmidt, Modeling Cognitive
Development on Balance Scale Phenomena, Machine Learning \textbf{16},
pp. 59-88, 1994.

\bibitem[(2004)]{cleveland1}Z.H. Zhou,Y. Jiang, NeC4.5: neural ensemble
based C4.5,\textquotedbl{} in IEEE Transactions on Knowledge and Data
Engineering \textbf{16}, pp. 770-773, 2004.

\bibitem{cleveland2}R. Setiono , W.K. Leow, FERNN: An Algorithm for
Fast Extraction of Rules from Neural Networks, Applied Intelligence
\textbf{12}, pp. 15-25, 2000.

\bibitem[(1998)]{dermatology}G. Demiroz, H.A. Govenir, N. Ilter,
Learning Differential Diagnosis of Eryhemato-Squamous Diseases using
Voting Feature Intervals, Artificial Intelligence in Medicine. \textbf{13},
pp. 147--165, 1998.

\bibitem[(1996)]{ecoli}P. Horton, K.Nakai, A Probabilistic Classification
System for Predicting the Cellular Localization Sites of Proteins,
In: Proceedings of International Conference on Intelligent Systems
for Molecular Biology \textbf{4}, pp. 109-15, 1996.

\bibitem[(1977)]{hayes-roth}B. Hayes-Roth, B., F. Hayes-Roth. Concept
learning and the recognition and classification of exemplars. Journal
of Verbal Learning and Verbal Behavior \textbf{16}, pp. 321-338, 1977.

\bibitem[(1997)]{heart}I. Kononenko, E. Šimec, M. Robnik-Šikonja,
Overcoming the Myopia of Inductive Learning Algorithms with RELIEFF,
Applied Intelligence \textbf{7}, pp. 39--55, 1997

\bibitem[(2002)]{housevotes}R.M. French, N. Chater, Using noise to
compute error surfaces in connectionist networks: a novel means of
reducing catastrophic forgetting, Neural Comput. \textbf{14}, pp.
1755-1769, 2002.

\bibitem[(2004)]{ion1}J.G. Dy , C.E. Brodley, Feature Selection for
Unsupervised Learning, The Journal of Machine Learning Research \textbf{5},
pp 845--889, 2004.

\bibitem{ion2}S. J. Perantonis, V. Virvilis, Input Feature Extraction
for Multilayered Perceptrons Using Supervised Principal Component
Analysis, Neural Processing Letters \textbf{10}, pp 243--252, 1999.

\bibitem[(2002)]{liver} J. Garcke, M. Griebel, Classification with
sparse grids using simplicial basis functions, Intell. Data Anal.
\textbf{6}, pp. 483-502, 2002.

\bibitem{liver1}J. Mcdermott, R.S. Forsyth, Diagnosing a disorder
in a classification benchmark, Pattern Recognition Letters \textbf{73},
pp. 41-43, 2016.

\bibitem[(2002)]{lymography}G. Cestnik, I. Konenenko, I. Bratko,
Assistant-86: A Knowledge-Elicitation Tool for Sophisticated Users.
In: Bratko, I. and Lavrac, N., Eds., Progress in Machine Learning,
Sigma Press, Wilmslow, pp. 31-45, 1987. 

\bibitem[(2007)]{mammographic}M. Elter, R. Schulz-Wendtland, T. Wittenberg,
The prediction of breast cancer biopsy outcomes using two CAD approaches
that both emphasize an intelligible decision process, Med Phys. \textbf{34},
pp. 4164-72, 2007.

\bibitem[(2007)]{parkinsons1}M.A. Little, P.E. McSharry, S.J Roberts
et al, Exploiting Nonlinear Recurrence and Fractal Scaling Properties
for Voice Disorder Detection. BioMed Eng OnLine \textbf{6}, 23, 2007.

\bibitem{parkinsons2}M.A. Little, P.E. McSharry, E.J. Hunter, J.
Spielman, L.O. Ramig, Suitability of dysphonia measurements for telemonitoring
of Parkinson's disease. IEEE Trans Biomed Eng. \textbf{56}, pp. 1015-1022,
2009.

\bibitem[(2007)]{pima}J.W. Smith, J.E. Everhart, W.C. Dickson, W.C.
Knowler, R.S. Johannes, Using the ADAP learning algorithm to forecast
the onset of diabetes mellitus, In: Proceedings of the Symposium on
Computer Applications and Medical Care IEEE Computer Society Press,
pp.261-265, 1988.

\bibitem[(2007)]{popfailures}D.D. Lucas, R. Klein, J. Tannahill,
D. Ivanova, S. Brandon, D. Domyancic, Y. Zhang, Failure analysis of
parameter-induced simulation crashes in climate models, Geoscientific
Model Development \textbf{6}, pp. 1157-1171, 2013.

\bibitem[(2007)]{regions2}N. Giannakeas, M.G. Tsipouras, A.T. Tzallas,
K. Kyriakidi, Z.E. Tsianou, P. Manousou, A. Hall, E.C. Karvounis,
V. Tsianos, E. Tsianos, A clustering based method for collagen proportional
area extraction in liver biopsy images (2015) Proceedings of the Annual
International Conference of the IEEE Engineering in Medicine and Biology
Society, EMBS, 2015-November, art. no. 7319047, pp. 3097-3100. 

\bibitem[(2007)]{saheart}T. Hastie, R. Tibshirani, Non-parametric
logistic and proportional odds regression, JRSS-C (Applied Statistics)
\textbf{36}, pp. 260--276, 1987.

\bibitem{segment}M. Dash, H. Liu, P. Scheuermann, K. L. Tan, Fast
hierarchical clustering and its validation, Data \& Knowledge Engineering
\textbf{44}, pp 109--138, 2003.

\bibitem[(2007)]{student}P. Cortez, A. M. Gonçalves Silva, Using
data mining to predict secondary school student performance, In Proceedings
of 5th FUture BUsiness TEChnology Conference (FUBUTEC 2008) (pp. 5--12).
EUROSIS-ETI, 2008.

\bibitem[(2007)]{transfusion}I-Cheng Yeh, King-Jang Yang, Tao-Ming
Ting, Knowledge discovery on RFM model using Bernoulli sequence, Expert
Systems with Applications \textbf{36}, pp. 5866-5871, 2009.

\bibitem[(2007)]{wdbc1}Jeyasingh, S., \& Veluchamy, M. (2017). Modified
bat algorithm for feature selection with the Wisconsin diagnosis breast
cancer (WDBC) dataset. Asian Pacific journal of cancer prevention:
APJCP, 18(5), 1257.

\bibitem[(2007)]{wdbc2}Alshayeji, M. H., Ellethy, H., \& Gupta, R.
(2022). Computer-aided detection of breast cancer on the Wisconsin
dataset: An artificial neural networks approach. Biomedical signal
processing and control, 71, 103141.

\bibitem[(2007)]{wine1}M. Raymer, T.E. Doom, L.A. Kuhn, W.F. Punch,
Knowledge discovery in medical and biological datasets using a hybrid
Bayes classifier/evolutionary algorithm. IEEE transactions on systems,
man, and cybernetics. Part B, Cybernetics : a publication of the IEEE
Systems, Man, and Cybernetics Society, \textbf{33} , pp. 802-813,
2003.

\bibitem{wine2}P. Zhong, M. Fukushima, Regularized nonsmooth Newton
method for multi-class support vector machines, Optimization Methods
and Software \textbf{22}, pp. 225-236, 2007.

\bibitem[(2007)]{eeg1}R. G. Andrzejak, K. Lehnertz, F.Mormann, C.
Rieke, P. David, and C. E. Elger, “Indications of nonlinear deterministic
and finite-dimensional structures in time series of brain electrical
activity: dependence on recording region and brain state,” Physical
Review E, vol. 64, no. 6, Article ID 061907, 8 pages, 2001. 

\bibitem{eeg2}A. T. Tzallas, M. G. Tsipouras, and D. I. Fotiadis,
“Automatic Seizure Detection Based on Time-Frequency Analysis and
Artificial Neural Networks,” Computational Intelligence and Neuroscience,
vol. 2007, Article ID 80510, 13 pages, 2007. doi:10.1155/2007/80510

\bibitem[(2007)]{zoo}M. Koivisto, K. Sood, Exact Bayesian Structure
Discovery in Bayesian Networks, The Journal of Machine Learning Research\textbf{
5}, pp. 549--573, 2004.

\bibitem[(2007)]{abalone}Nash, W.J.; Sellers, T.L.; Talbot, S.R.;
Cawthor, A.J.; Ford, W.B. The Population Biology of Abalone (\_Haliotis\_
species) in Tasmania. I. Blacklip Abalone (\_H. rubra\_) from the
North Coast and Islands of Bass Strait, Sea Fisheries Division; Technical
Report No. 48; Department of Primary Industry and Fisheries, Tasmania:
Hobart, Australia, 1994; ISSN 1034-3288

\bibitem[(2007)]{airfoil}Brooks, T.F.; Pope, D.S.; Marcolini, A.M.
Airfoil Self-Noise and Prediction. Technical Report, NASA RP-1218.
July 1989. Available online: https://ntrs.nasa.gov/citations/19890016302
(accessed on 14 November 2024).

\bibitem[(2007)]{concrete}I.Cheng Yeh, Modeling of strength of high
performance concrete using artificial neural networks, Cement and
Concrete Research. \textbf{28}, pp. 1797-1808, 1998. 

\bibitem{friedman}Friedman, J. (1991): Multivariate Adaptative Regression
Splines. Annals of Statistics, 19:1, 1-{}-141. 

\bibitem[(2007)]{housing}D. Harrison and D.L. Rubinfeld, Hedonic
prices and the demand for clean ai, J. Environ. Economics \& Management
\textbf{5}, pp. 81-102, 1978.

\bibitem[(2025)]{optimus}I.G. Tsoulos, V. Charilogis, G. Kyrou, V.N.
Stavrou, A. Tzallas, Journal of Open Source Software \textbf{10},
7584, 2025.

\bibitem{powell}M.J.D Powell, A Tolerant Algorithm for Linearly Constrained
Optimization Calculations, Mathematical Programming \textbf{45}, pp.
547-566, 1989. 

\bibitem[(1991)]{rbf1}J. Park and I. W. Sandberg, Universal Approximation
Using Radial-Basis-Function Networks, Neural Computation \textbf{3},
pp. 246-257, 1991.

\bibitem{rbf2}G.A. Montazer, D. Giveki, M. Karami, H. Rastegar, Radial
basis function neural networks: A review. Comput. Rev. J \textbf{1},
pp. 52-74, 2018.

\bibitem[(2002)]{neat}K. O. Stanley, R. Miikkulainen, Evolving Neural
Networks through Augmenting Topologies, Evolutionary Computation \textbf{10},
pp. 99-127, 2002.

\bibitem[(2002)]{prune}Zhu, V., Lu, Y., \& Li, Q. (2006). MW-OBS:
An improved pruning method for topology design of neural networks.
Tsinghua Science and Technology, 11(4), 307-312.

\bibitem{fcn}Grzegorz Klima, Fast Compressed Neural Networks, available
from \url{http://fcnn.sourceforge.net/}.

\bibitem[(2020)]{adagrad}Ward, R., Wu, X., \& Bottou, L. (2020).
Adagrad stepsizes: Sharp convergence over nonconvex landscapes. Journal
of Machine Learning Research, 21(219), 1-30.

\bibitem[(2024)]{ml_fire}C. Kopitsa, I.G. Tsoulos, V. Charilogis,
A. Stavrakoudis, Predicting the Duration of Forest Fires Using Machine
Learning Methods, Future Internet \textbf{16}, 396, 2024.

\bibitem[(2023)]{pirvision}Emad-Ud-Din, M., \& Wang, Y. (2023). Promoting
occupancy detection accuracy using on-device lifelong learning. IEEE
Sensors Journal, 23(9), 9595-9606.

\bibitem[(1996)]{interval1}M.A. Wolfe, Interval methods for global
optimization, Applied Mathematics and Computation \textbf{75}, pp.
179-206, 1996.

\bibitem{interval2}T. Csendes and D. Ratz, Subdivision Direction
Selection in Interval Methods for Global Optimization, SIAM J. Numer.
Anal. \textbf{34}, pp. 922--938, 1997. 

\end{thebibliography}

\end{adjustwidth}{}
\end{document}
