%% LyX 2.4.3 created this file.  For more info, see https://www.lyx.org/.
%% Do not edit unless you really know what you are doing.
\documentclass[journal,article,submit,pdftex,moreauthors]{Definitions/mdpi}
\usepackage[utf8]{inputenc}
\usepackage{cprotect}
\usepackage{float}
\usepackage{url}
\usepackage{varwidth}
\usepackage{amsmath}

\makeatletter

%%%%%%%%%%%%%%%%%%%%%%%%%%%%%% LyX specific LaTeX commands.

\Title{Next-Gen Evolutionary Algorithms: A Modified Differential Evolution
Approach}

\TitleCitation{Next-Gen Evolutionary Algorithms: A Modified Differential Evolution
Approach}

\Author{Anna Maria Gianni$^{1}$, Ioannis G. Tsoulos$^{2,*}$ , Vasileios
Charilogis$^{3}$ and Glykeria Kyrou$^{4}$}

\AuthorNames{A.M. Gianni, Ioannis G. Tsoulos, Vasileios Charilogis and Glykeria
Kyrou}

\AuthorCitation{Gianni A.M.; Tsoulos, I.G.; Charilogis, V. ;Kyrou, G.}


\address{$^{1}$\quad{}Department of Informatics and Telecommunications,
University of Ioannina, 47150 Kostaki Artas, Greece; am.gianni@uoi.gr\\
$^{2}$\quad{}Department of Informatics and Telecommunications, University
of Ioannina, 47150 Kostaki Artas, Greece; itsoulos@uoi.gr\\
$^{3}$\quad{}Department of Informatics and Telecommunications, University
of Ioannina, 47150 Kostaki Artas, Greece; v.charilog@uoi.gr\\
$^{4}$\quad{}Department of Informatics and Telecommunications, University
of Ioannina, 47150 Kostaki Artas, Greece; g.kyrou@uoi.gr}


\corres{Correspondence: itsoulos@uoi.gr}


\abstract{This paper presents an innovative optimization algorithm based on
Differential Evolution (DE) that combines advanced mutation techniques
with intelligent termination mechanisms. The proposed algorithm is
designed to address the main limitations of classical DE, offering
enhanced performance for complex optimization problems. The core scientific
contribution of this research focuses on three key aspects. First,
we develop a hybrid dual-strategy mutation system where the first
strategy emphasizes exploration of the solution space through monitoring
of the optimal solution, while the second strategy focuses on exploitation
of promising regions using dynamically weighted differential terms.
This dual mechanism ensures a balanced approach between discovering
new solutions and improving existing ones. Second, the algorithm incorporates
a novel Majority Dimension Mechanism that evaluates candidate solutions
through dimension-wise comparison with elite references (best sample
and worst sample). This mechanism dynamically guides the search process
by determining whether to intensify local exploitation or initiate
global exploration based on majority voting across all dimensions.
Third, the work presents numerous new termination rules based on quantitative
evaluation of metric value homogeneity. These rules extend beyond
traditional convergence checks by incorporating multidimensional criteria
that consider both solution distribution and evolutionary dynamics.
This system enables more sophisticated and adaptive decision-making
regarding the optimal stopping point of the optimization process.
The methodology is validated through extensive experimental procedures
covering a wide range of optimization problems. The results demonstrate
significant improvements in both solution quality and computational
efficiency, particularly for high-dimensional problems with numerous
local optima. The algorithm's practical applications span multiple
domains of modern technology and science. In engineering design, it
demonstrates capability for solving complex parametric optimization
problems. For artificial intelligence and machine learning, it provides
a powerful tool for automated hyperparameter tuning. Additionally,
the method finds applications in economic modeling and biological
systems analysis. The study concludes with a critical analysis of
the proposed method's advantages and limitations, along with suggestions
for future research directions. Special emphasis is placed on extending
the algorithm for dynamic optimization problems and integrating deep
learning techniques for automated parameter configuration. The research
findings highlight the proposed algorithm's potential as a high-performance
tool for solving complex optimization challenges in contemporary scientific
and technological contexts.}


\keyword{Optimization; Differential Evolution; Evolutionary Algorithms; Global
Optimization; Adaptive Termination; Mutation Strategies; Metaheuristics;}

\DeclareTextSymbolDefault{\textquotedbl}{T1}
%% Because html converters don't know tabularnewline
\providecommand{\tabularnewline}{\\}
%% Variable width box for table cells
\newenvironment{cellvarwidth}[1][t]
    {\begin{varwidth}[#1]{\linewidth}}
    {\@finalstrut\@arstrutbox\end{varwidth}}
\floatstyle{ruled}
\newfloat{algorithm}{tbp}{loa}
\providecommand{\algorithmname}{Algorithm}
\floatname{algorithm}{\protect\algorithmname}

%%%%%%%%%%%%%%%%%%%%%%%%%%%%%% User specified LaTeX commands.
%  LaTeX support: latex@mdpi.com 
%  For support, please attach all files needed for compiling as well as the log file, and specify your operating system, LaTeX version, and LaTeX editor.

%=================================================================


% For posting an early version of this manuscript as a preprint, you may use "preprints" as the journal and change "submit" to "accept". The document class line would be, e.g., \documentclass[preprints,article,accept,moreauthors,pdftex]{mdpi}. This is especially recommended for submission to arXiv, where line numbers should be removed before posting. For preprints.org, the editorial staff will make this change immediately prior to posting.

%--------------------
% Class Options:
%--------------------
%----------
% journal
%----------
% Choose between the following MDPI journals:
% acoustics, actuators, addictions, admsci, adolescents, aerospace, agriculture, agriengineering, agronomy, ai, algorithms, allergies, alloys, analytica, animals, antibiotics, antibodies, antioxidants, applbiosci, appliedchem, appliedmath, applmech, applmicrobiol, applnano, applsci, aquacj, architecture, arts, asc, asi, astronomy, atmosphere, atoms, audiolres, automation, axioms, bacteria, batteries, bdcc, behavsci, beverages, biochem, bioengineering, biologics, biology, biomass, biomechanics, biomed, biomedicines, biomedinformatics, biomimetics, biomolecules, biophysica, biosensors, biotech, birds, bloods, blsf, brainsci, breath, buildings, businesses, cancers, carbon, cardiogenetics, catalysts, cells, ceramics, challenges, chemengineering, chemistry, chemosensors, chemproc, children, chips, cimb, civileng, cleantechnol, climate, clinpract, clockssleep, cmd, coasts, coatings, colloids, colorants, commodities, compounds, computation, computers, condensedmatter, conservation, constrmater, cosmetics, covid, crops, cryptography, crystals, csmf, ctn, curroncol, currophthalmol, cyber, dairy, data, dentistry, dermato, dermatopathology, designs, diabetology, diagnostics, dietetics, digital, disabilities, diseases, diversity, dna, drones, dynamics, earth, ebj, ecologies, econometrics, economies, education, ejihpe, electricity, electrochem, electronicmat, electronics, encyclopedia, endocrines, energies, eng, engproc, ent, entomology, entropy, environments, environsciproc, epidemiologia, epigenomes, est, fermentation, fibers, fintech, fire, fishes, fluids, foods, forecasting, forensicsci, forests, foundations, fractalfract, fuels, futureinternet, futureparasites, futurepharmacol, futurephys, futuretransp, galaxies, games, gases, gastroent, gastrointestdisord, gels, genealogy, genes, geographies, geohazards, geomatics, geosciences, geotechnics, geriatrics, hazardousmatters, healthcare, hearts, hemato, heritage, highthroughput, histories, horticulturae, humanities, humans, hydrobiology, hydrogen, hydrology, hygiene, idr, ijerph, ijfs, ijgi, ijms, ijns, ijtm, ijtpp, immuno, informatics, information, infrastructures, inorganics, insects, instruments, inventions, iot, j, jal, jcdd, jcm, jcp, jcs, jdb, jeta, jfb, jfmk, jimaging, jintelligence, jlpea, jmmp, jmp, jmse, jne, jnt, jof, joitmc, jor, journalmedia, jox, jpm, jrfm, jsan, jtaer, jzbg, kidney, kidneydial, knowledge, land, languages, laws, life, liquids, literature, livers, logics, logistics, lubricants, lymphatics, machines, macromol, magnetism, magnetochemistry, make, marinedrugs, materials, materproc, mathematics, mca, measurements, medicina, medicines, medsci, membranes, merits, metabolites, metals, meteorology, methane, metrology, micro, microarrays, microbiolres, micromachines, microorganisms, microplastics, minerals, mining, modelling, molbank, molecules, mps, msf, mti, muscles, nanoenergyadv, nanomanufacturing, nanomaterials, ncrna, network, neuroglia, neurolint, neurosci, nitrogen, notspecified, nri, nursrep, nutraceuticals, nutrients, obesities, oceans, ohbm, onco, oncopathology, optics, oral, organics, organoids, osteology, oxygen, parasites, parasitologia, particles, pathogens, pathophysiology, pediatrrep, pharmaceuticals, pharmaceutics, pharmacoepidemiology, pharmacy, philosophies, photochem, photonics, phycology, physchem, physics, physiologia, plants, plasma, pollutants, polymers, polysaccharides, poultry, powders, preprints, proceedings, processes, prosthesis, proteomes, psf, psych, psychiatryint, psychoactives, publications, quantumrep, quaternary, qubs, radiation, reactions, recycling, regeneration, religions, remotesensing, reports, reprodmed, resources, rheumato, risks, robotics, ruminants, safety, sci, scipharm, seeds, sensors, separations, sexes, signals, sinusitis, skins, smartcities, sna, societies, socsci, software, soilsystems, solar, solids, sports, standards, stats, stresses, surfaces, surgeries, suschem, sustainability, symmetry, synbio, systems, taxonomy, technologies, telecom, test, textiles, thalassrep, thermo, tomography, tourismhosp, toxics, toxins, transplantology, transportation, traumacare, traumas, tropicalmed, universe, urbansci, uro, vaccines, vehicles, venereology, vetsci, vibration, viruses, vision, waste, water, wem, wevj, wind, women, world, youth, zoonoticdis 

%---------
% article
%---------
% The default type of manuscript is "article", but can be replaced by: 
% abstract, addendum, article, book, bookreview, briefreport, casereport, comment, commentary, communication, conferenceproceedings, correction, conferencereport, entry, expressionofconcern, extendedabstract, datadescriptor, editorial, essay, erratum, hypothesis, interestingimage, obituary, opinion, projectreport, reply, retraction, review, perspective, protocol, shortnote, studyprotocol, systematicreview, supfile, technicalnote, viewpoint, guidelines, registeredreport, tutorial
% supfile = supplementary materials

%----------
% submit
%----------
% The class option "submit" will be changed to "accept" by the Editorial Office when the paper is accepted. This will only make changes to the frontpage (e.g., the logo of the journal will get visible), the headings, and the copyright information. Also, line numbering will be removed. Journal info and pagination for accepted papers will also be assigned by the Editorial Office.

%------------------
% moreauthors
%------------------
% If there is only one author the class option oneauthor should be used. Otherwise use the class option moreauthors.

%---------
% pdftex
%---------
% The option pdftex is for use with pdfLaTeX. If eps figures are used, remove the option pdftex and use LaTeX and dvi2pdf.

%=================================================================
% MDPI internal commands - do not modify
\firstpage{1} 
 
\setcounter{page}{\@firstpage} 

\pubvolume{1}
\issuenum{1}
\articlenumber{0}
\pubyear{2024}
\copyrightyear{2024}
%\externaleditor{Academic Editor: Firstname Lastname} % For journal Automation, please change Academic Editor to "Communicated by"
\datereceived{}
\daterevised{ } % Comment out if no revised date
\dateaccepted{}
\datepublished{}
%\datecorrected{} % Corrected papers include a "Corrected: XXX" date in the original paper.
%\dateretracted{} % Corrected papers include a "Retracted: XXX" date in the original paper.
\hreflink{https://doi.org/} % If needed use \linebreak
%\doinum{}
%------------------------------------------------------------------
% The following line should be uncommented if the LaTeX file is uploaded to arXiv.org
%\pdfoutput=1

%=================================================================
% Add packages and commands here. The following packages are loaded in our class file: fontenc, inputenc, calc, indentfirst, fancyhdr, graphicx, epstopdf, lastpage, ifthen, lineno, float, amsmath, setspace, enumitem, mathpazo, booktabs, titlesec, etoolbox, tabto, xcolor, soul, multirow, microtype, tikz, totcount, changepage, attrib, upgreek, cleveref, amsthm, hyphenat, natbib, hyperref, footmisc, url, geometry, newfloat, caption

%=================================================================
%% Please use the following mathematics environments: Theorem, Lemma, Corollary, Proposition, Characterization, Property, Problem, Example, ExamplesandDefinitions, Hypothesis, Remark, Definition, Notation, Assumption
%% For proofs, please use the proof environment (the amsthm package is loaded by the MDPI class).

%=================================================================
% The fields PACS, MSC, and JEL may be left empty or commented out if not applicable
%\PACS{J0101}
%\MSC{}
%\JEL{}

%%%%%%%%%%%%%%%%%%%%%%%%%%%%%%%%%%%%%%%%%%
% Only for the journal Diversity
%\LSID{\url{http://}}

%%%%%%%%%%%%%%%%%%%%%%%%%%%%%%%%%%%%%%%%%%
% Only for the journal Applied Sciences:
%\featuredapplication{Authors are encouraged to provide a concise description of the specific application or a potential application of the work. This section is not mandatory.}
%%%%%%%%%%%%%%%%%%%%%%%%%%%%%%%%%%%%%%%%%%

%%%%%%%%%%%%%%%%%%%%%%%%%%%%%%%%%%%%%%%%%%
% Only for the journal Data:
%\dataset{DOI number or link to the deposited data set in cases where the data set is published or set to be published separately. If the data set is submitted and will be published as a supplement to this paper in the journal Data, this field will be filled by the editors of the journal. In this case, please make sure to submit the data set as a supplement when entering your manuscript into our manuscript editorial system.}

%\datasetlicense{license under which the data set is made available (CC0, CC-BY, CC-BY-SA, CC-BY-NC, etc.)}

%%%%%%%%%%%%%%%%%%%%%%%%%%%%%%%%%%%%%%%%%%
% Only for the journal Toxins
%\keycontribution{The breakthroughs or highlights of the manuscript. Authors can write one or two sentences to describe the most important part of the paper.}

%%%%%%%%%%%%%%%%%%%%%%%%%%%%%%%%%%%%%%%%%%
% Only for the journal Encyclopedia
%\encyclopediadef{Instead of the abstract}
%\entrylink{The Link to this entry published on the encyclopedia platform.}
%%%%%%%%%%%%%%%%%%%%%%%%%%%%%%%%%%%%%%%%%%

%%%%%%%%%%%%%%%%%%%%%%%%%%%%%%%%%%%%%%%%%%
% Only for the journal Advances in Respiratory Medicine
%\addhighlights{yes}
%\renewcommand{\addhighlights}{%

%\noindent This is an obligatory section in “Advances in Respiratory Medicine”, whose goal is to increase the discoverability and readability of the article via search engines and other scholars. Highlights should not be a copy of the abstract, but a simple text allowing the reader to quickly and simplified find out what the article is about and what can be cited from it. Each of these parts should be devoted up to 2~bullet points.\vspace{3pt}\\
%\textbf{What are the main findings?}
% \begin{itemize}[labelsep=2.5mm,topsep=-3pt]
% \item First bullet.
% \item Second bullet.
% \end{itemize}\vspace{3pt}
%\textbf{What is the implication of the main finding?}
% \begin{itemize}[labelsep=2.5mm,topsep=-3pt]
% \item First bullet.
% \item Second bullet.
% \end{itemize}
%}
%%%%%%%%%%%%%%%%%%%%%%%%%%%%%%%%%%%%%%%%%%
% Added by lyx2lyx
\usepackage{array}

\makeatother

\begin{document}
\maketitle

\section{Introduction}

Global optimization is concerned with finding the global minimum of
a continuous objective function $f:S\rightarrow R$\textbf{ }where
$S\subseteq R^{n}$ is a compact set. Formally, the problem is defined
as:

\begin{equation}
x^{*}=\mbox{arg}\min_{x\in S}f(x).
\end{equation}
where $x^{*}$is the global minimizer. Typically, is an n-dimensional
hyperrectangle:

\textbf{
\[
S=\left[a_{1},b_{1}\right]\otimes\left[a_{2},b_{2}\right]\otimes\ldots\left[a_{n},b_{n}\right]
\]
}

with, $a_{i}$ and $b_{i}$ representing lower and upper bounds for
each variable $x_{i}$.

This paper focuses on the development of an innovative optimization
algorithm based on DE. DE has emerged as one of the most popular tools
for optimizing complex problems across various fields, such as artificial
intelligence, machine learning, biological systems analysis, and engineering
optimization applications\citep{diffe1,diffe2,key-29,mult_minfinder}.
According to the literature, DE is characterized by its simplicity
and flexibility, enabling it to solve high-complexity problems \citep{genetic3,diffe2,crs2,mult_cfo}.

This novel approach introduces three key scientific contributions.
First, a hybrid dual-strategy mutation system is developed that balances
exploration of the solution space with exploitation of promising regions.
Exploration is achieved through linearly decreasing disturbance coefficients,
while exploitation relies on dynamically weighted differential terms,
as discussed in earlier research \citep{Charilogis,genetic3}. Second,
a system of five pioneering termination rules is introduced, based
on multidimensional criteria that consider the distribution of solutions
and evolutionary dynamics, focusing on stabilization and solution
quality \citep{simann1,fish,key-17,WOA1}.

Additionally, the proposed algorithm includes improvements related
to parameter optimization using techniques such as modified swarm
strategies and hybrid exploration methods \citep{key-18,key-14,key-4}.
This approach has proven highly effective in addressing multimodal
problems with high computational requirements {[}\citep{key-6,key-20}{]}.
The algorithm’s validation is carried out through extensive experimental
trials, utilizing a wide range of benchmark functions to evaluate
its performance and stability {[}\citep{key-8,key-9,key-21}{]}.

The findings of this study highlight the algorithm's applicability
across various fields. Specifically, in engineering technologies,
the algorithm offers solutions for the parameterization of complex
systems {[}\citep{key-33,key-34}{]}. In machine learning, it provides
exceptional accuracy in hyperparameter tuning for neural networks,
while in economic modeling, it contributes to trend forecasting and
resource optimization \citep{key-25,Charilogis,key-26}.

The research concludes with a critical analysis of the method's advantages
and limitations, focusing on opportunities for future improvements.
Specifically, it suggests incorporating dynamic optimization techniques
and leveraging deep learning technologies for automated parameter
configuration \citep{key-28,key-35,key-22}. These prospects underscore
the contribution of this study to the optimization literature, utilizing
hybrid methodologies to solve high-complexity and multimodal problems
\citep{Geetha,testfunc2,pde}.

Furthermore, the study's results demonstrate the proposed algorithm's
capability to solve high-dimensional problems with numerous local
minima, making it valuable for modern scientific and technological
challenges \citep{diffe1,key-29}. This algorithm is successfully
applied in domains such as engineering design, automated hyperparameter
configuration in machine learning, and economic modeling \citep{crs2,fish,key-14}.
Its potential for extension to dynamic problems and integration with
deep learning techniques for automated parameterization presents exciting
prospects for the future \citep{Charilogis,key-28}. This work highlights
the importance of hybrid methodologies in contemporary optimization
research and emphasizes their contribution to solving complex problems
\citep{key-33,testfunc2,pde}.

The rest of the paper is organized as follows:
\begin{itemize}
\item Section 1: Introduction
\item Section 2: Materials and Methods\ref{sec:Materials-and-Methods}
\begin{itemize}
\item Section 2.1: The original DE\ref{subsec:originalDE}
\item Section 2.2: Detailed formulation of the modified DE algorithm\ref{subsec:modifiedDE}
\item Section 2.3: Majority Dimension Mechanism (MDM)
\item Section 2.4: The termination rules\ref{subsec:terminationsRules}
\begin{enumerate}
\item Section 2.4.1: Best solution similarity (BSS)
\item Section 2.4.2: Worst solution similarity (WSS)
\item Section 2.4.3: Top Solutions Similarity (TSS)
\item Section 2.4.4: Bottom Solutions Similarity (BOSS)
\item Section 2.4.5: Solution Range Similarity (SRS)
\item Section 2.4.6: Improvement Rate Similarity (IRS)
\item Section 2.4.7: Termination rule: Double box similarity (Double box)
\item Section 2.4.9: All
\end{enumerate}
\end{itemize}
\item Section 3: Experimental setup and benchmark results.\ref{sec:Results}
\begin{itemize}
\item Section 3.1: Test functions\ref{subsec:testFunctions}
\item Section 3.2: Experimental results.\ref{subsec:experimentalResults}
\end{itemize}
\item Section 4: Discussion of findings and comparative analysis\ref{sec:discussion}
\item Section 5: Conclusions and future research directions\ref{sec:Conclusions}
\end{itemize}

\section{Materials and Methods\label{sec:Materials-and-Methods}}

\subsection{\textbf{The original DE\label{subsec:originalDE}}}

\begin{algorithm}[H]
\begin{enumerate}
\item Initialization:
\begin{enumerate}
\item Set parameters:
\begin{enumerate}
\item Population size $NP$
\item Crossover probability $CR\in[0,1]$
\item Differential weight $F\in[0,2]$
\item Maximum iterations maxIterations
\end{enumerate}
\item Randomly initialize population samples$[1..NP]$ within bounds
\item Evaluate initial fitness $fitnessArray[i]$ for each individual
\item Initialize global best solution $bSample$ and fitness $bF2x$
\end{enumerate}
\item Main Optimization Loop (for iteration = 1 to maxIterations):

For each individual $x_{i}$ in population ($i=1\thinspace to\thinspace NP$):
\begin{enumerate}
\item Parent Selection:
\begin{enumerate}
\item Randomly select 3 distinct indices $indexA,indexB,indexC\neq i$
\end{enumerate}
\item Trial Vector Creation:
\begin{enumerate}
\item Select random dimension $R$
\item For each dimension $d$ (1 to dimension):
\begin{enumerate}
\item $ri$= random number $\in[0,1]$
\item If $ri$ \ensuremath{\le} $CR$ OR $d==R$:
\end{enumerate}
$trialsample[d]=samples[indexA][d]+F\times(samples[indexB][d]-samples[indexC][d])$
\begin{enumerate}
\item Apply boundary correction if needed
\item Else:
\end{enumerate}
$trialsample[d]=samples[i][d]$
\end{enumerate}
\item Selection:
\begin{enumerate}
\item Evaluate trial vector: $minValue=f(trialsample)$
\item If $minValue$ \ensuremath{\le} $fitnessArray[i]$:
\begin{enumerate}
\item Update $samples[i]=trialsample$
\item Update $fitnessArray[i]=minValue$
\item Update global best $bSample$ if improved
\end{enumerate}
\end{enumerate}
\end{enumerate}
\item Local Search Phase (optional):

For each individual $i$:
\begin{enumerate}
\item If $random[0,1]$ \textless{} $localSearchRate$:

$refinedFitness=localSearch(samples[i])$

If $refinedFitness$ \textless{} $fitnessArray[i]$:

Update $samples[i]$ and $fitnessArray[i]$

Update global best $bSample$ if improved
\end{enumerate}
\item Termination:
\begin{enumerate}
\item When max iterations reachedor any of the termination rules is verified\ref{subsec:terminationsRules}
\item Return best solution bSample found
\end{enumerate}
\end{enumerate}
\caption{Pseudocode for Standard DE Algorithm}

\end{algorithm}


\subsection{\textbf{Detailed formulation of the algorithm \label{subsec:modifiedDE} }}

\begin{algorithm}[H]
\begin{enumerate}
\item Initialization Phase:
\begin{enumerate}
\item Set population size $NP$
\item Initialize parameters:
\begin{enumerate}
\item Crossover probability $CR\in[0,1]$
\item Local search probability localSearchRate
\item Problem boundaries $lmargin[],rmargin[]$
\end{enumerate}
\item Randomly initialize population $samples[i],i=1..NP$
\item Evaluate initial fitness $fitnessArray[i]$ for all individuals
\item Identify global best solution$bSample$ and fitness $bF2x$
\end{enumerate}
\item Main Optimization Loop (until max iterations reached):

For each individual $x_{i}$ in population ($i=1toNP$):
\begin{enumerate}
\item Strategy Selection:
\begin{enumerate}
\item Generate random number $randStrategy\in[0,1]$
\item If $randStrategy$ \textless{} 0.2: Use Strategy 1
\item Else: Use Strategy 2
\end{enumerate}
\item Mutation Phase:

Strategy 1 (Exploration):
\begin{enumerate}
\item For each dimension j:

Generate random numbers $r=\in[0,1]$

$distance=\mid samples[i][j]-bestSample[j]\mid$ 

$trialsample[j]=bSamply[j]-r\times distance$

Apply boundary repair if needed
\end{enumerate}
Strategy 2 (Exploitation):
\begin{enumerate}
\item Select $dimension+1$ distinct random indices from population ($\neq i$)
\item Choose random dimension $R$
\item For each dimension $d$:
\begin{enumerate}
\item $r_{i}=random[0,1]$
\item If $r_{i}$ \textgreater$CR$ or $d==R$:

Generate random number$R_{a}=\in[-0.0,1.0]$

$differentialWeight=\frac{1}{2}+2\times R_{a}$\citep{Charilogis}

$trialsample[d]=samples[index_{1}][d]+differentialWeight\times(samples[index_{2}][d]-samples[otherindices_{i}][d])$

Apply boundary repair
\item Else:

$trialsample[d]=x_{i}[d]$
\end{enumerate}
\end{enumerate}
\item Selection:
\begin{enumerate}
\item Evaluate trial vector: $minValue=f(trialsample)$
\item If $minValue$ \textless{} $fitnessArray[i]$:

$samples[i]=trialsample$

$fitnessArray[i]=minValue$

If $minValue$ \textless{} $bF2x$:

Update global best $bSamply=trialsample$

$bF2x=minValue$
\end{enumerate}
\end{enumerate}
\item Local Search Phase (optional):

For each individual $i$:
\begin{enumerate}
\item If $random[0,1]$ \textless{} $localSearchRate$:

$refinedFitness=localSearch(samples[i])$

If $refinedFitness$ \textless{} $fitnessArray[i]$:

Update $samples[i]$ and $fitnessArray[i]$

Update global best if improved
\end{enumerate}
\item Termination:
\begin{enumerate}
\item When max iterations reached or any of the termination rules is verified\ref{subsec:terminationsRules}
\item Return global best solution $bSample$
\end{enumerate}
\end{enumerate}
\caption{Modified Differential Evolution with Dual Mutation Strategies}

\end{algorithm}


\subsection{\textbf{Majority Dimension Mechanism}}

The Majority Dimension Mechanism plays a significant role in optimization
algorithms, where it serves as a guiding tool for the search process.
It operates by comparing a current sample $s$ against two critical
reference points: an optimal sample $b$ representing a known good
solution or minimum, and a worse sample $w$ representing less desirable
solutions. The mechanism's operation begins with a dimension-by-dimension
analysis of the sample. For each vector element, it calculates the
absolute differences between the sample and the corresponding elements
of both reference vectors. The mechanism then determines which reference
vector the current sample aligns with more closely in each individual
dimension. When the function returns true, indicating the sample is
closer to the optimal sample in most dimensions, this signifies the
current position is near a previously discovered minimum. In this
case, the optimization algorithm may make decisions such as changing
direction, increasing the search step size, or applying randomized
variations, aiming to avoid local minima traps and explore new regions
of the solution space. Conversely, when the function returns false,
indicating the sample is closer to the worse sample, this suggests
the current position warrants further exploration. Here, the algorithm
may focus its search on the current region, hoping for further solution
improvement. In a Euclidean parameter space, this mechanism enables
dynamic search adaptation. When leading to true, it encourages the
algorithm to make larger jumps toward new directions or employ local
minima escape methods. When leading to false, it concentrates the
search around the current area for more detailed exploration. The
key advantages of this mechanism include its ability to prevent stagnation
at suboptimal solutions, its operational flexibility by focusing on
individual dimensions rather than aggregate measures, and its broad
applicability to various optimization algorithms - from gradient-based
to evolutionary approaches. In summary, the Majority Dimension Mechanism
functions as an intelligent director in optimal solution searches.
It offers a balanced approach between exploring new regions and exploiting
known good solutions, enhancing both the efficiency and reliability
of the optimization process without being constrained by local optima.

The formula (function) of Majority Dimension Mechanism :\medskip{}

$\text{checkSample}(s,b,w)=\begin{cases}
\text{true}, & \text{if }\sum_{d=1}^{dimension}\mathbb{I}\left(|s_{d}-b_{d}|<|s_{d}-w_{d}|\right)>\sum_{d=1}^{dimension}\mathbb{I}\left(|s_{d}-w_{d}|<|s_{d}-b_{d}|\right)\\
\text{false}, & \text{else}
\end{cases}$

\medskip{}

Where:

$s=[s_{1},s_{2},\ldots,s_{d}]$ : The sample being evaluated

$b=[b_{1},b_{2},\ldots,b_{d}]$ : The best sample 

$w=[w_{1},w_{2},\ldots,w_{d}]$ : The worst sample

\subsection{\textbf{The termination rules\label{subsec:terminationsRules}}}

The algorithm's termination mechanism evaluates similarity between
specific values in the fitness array across consecutive iterations.
Each rule focuses on different aspects of population evolution by
monitoring changes in critical metrics from the fitness array. For
the best solution, the stability of its value is checked across iterations.
If the difference remains negligible for a specified number of iterations,
similarity is considered achieved. Similarly, an equivalent check
is applied to the population's worst solution, monitoring the stability
of its value. Furthermore, the mechanism evaluates similarity among
solution groups. For top solutions, it examines the sum of a percentage
of the best fitness values per iteration, while for worst solutions
it performs an analogous check on the sum of corresponding values.
An additional rule measures similarity in solution range by comparing
the difference between best and worst values across iterations. Finally,
a rule checks similarity in improvement rate by comparing how much
the best and worst solutions have improved between iterations. All
these rules aim to identify moments when fitness array values show
significant similarity for sufficient duration, indicating the population
has reached stability. This logic ensures the algorithm terminates
only when adequate result stabilization occurs, optimizing runtime
without compromising solution quality. Using multiple similarity rules
provides a comprehensive convergence picture, covering various aspects
of population evolution.

General parameters:
\begin{itemize}
\item $e$ controls the precision requirement (typically set to 1e-6)
\item $sim$ determines the required stability duration (commonly 5-10 iterations)
\end{itemize}

\subsubsection{Best solution similarity}

The termination mechanism is based on a simple criterion that evaluates
the similarity of values in the fitness array. Specifically, at each
iteration $iter$, the difference

\begin{equation}
\delta_{sim}^{(iter)}=\left|f_{sim,min}^{(iter)}-f_{sim,min}^{(iter-1)}\right|,
\end{equation}
is calculated, where $f_{sim,min}^{(iter)}$ represents the best fitness
value found up to iteration $iter$. If the difference $\delta_{sim}^{(iter)}$
remains less than or equal to a predefined accuracy threshold $e$
for at least $sim$ consecutive iterations, then the population is
considered to have converged and the process terminates. This criterion
ensures that the algorithm will stop only when minimal progress occurs
in the best solution for a sufficient time period, thereby optimizing
computational time.

\subsubsection{Worst solution similarity}

The termination mechanism is based on the difference

\begin{equation}
\delta_{sim}^{(iter)}=\left|f_{sim,max}^{(iter)}-f_{sim,max}^{(iter-1)}\right|,
\end{equation}
where$f_{sim,max}^{(iter)}$ represents the worst fitness value at
iteration iter. If $\delta_{sim}^{(iter)}\le e$ for $sim$ consecutive
iterations, the algorithm terminates, indicating the population has
stabilized. This guarantees termination only occurs when the worst
solution stops improving significantly. This logic provides an efficient
way to monitor population stability, optimizing computational time
without compromising solution quality. The mechanism is particularly
useful in applications where fitness value variance across the population
is a critical factor for determining convergence.

\subsubsection{Top Solutions Similarity}

The termination mechanism evaluates the difference
\begin{equation}
\delta_{best}=\left|\sum_{i=NP-K}^{NP}f_{sorted}^{(iter)}[i]-\sum_{i=NP-K}^{NP}f_{sorted}^{(iter-1)}[i]\right|,
\end{equation}

when $K=\lceil\text{population}\times\text{sumRate}\rceil$,

where $f_{sorted}^{(t)}[i]$ represents the i-th best fitness value
in the sorted population at iteration $iter$, $NP$ is the population
size, and $K$ is an integer smaller than $NP$. The process terminates
when the difference $\delta_{best}$ remains below a specified threshold
$e$ for a predetermined number of iterations, indicating stabilization
of the overall performance of the population's top solutions. This
criterion focuses on monitoring the collective trend of the K best
solutions, providing a robust and reliable convergence measure. It
proves particularly valuable in optimization problems where the stability
of the population's elite solutions serves as a critical indicator
of the algorithm's final convergence, ensuring the process completes
only when sufficient stability is achieved among the top-performing
solutions.

\subsubsection{Bottom Solutions similarity}

The termination mechanism evaluates the difference

\begin{equation}
\delta_{worst}=\left|\sum_{i=N-K}^{N}f_{sorted}^{(iter)}[i]-\sum_{i=N-K}^{N}f_{sorted}^{(iter-1)}[i]\right|,
\end{equation}

when $K=\lceil\text{population}\times\text{sumRate}\rceil$, where
$f_{sorted}^{(iter)}[i]$ represents the i-th worst fitness value
in the sorted population at iteration$iter$, with $NP$ denoting
the total population size and $K$ being an integer smaller than $NP$.
The process terminates when the difference $\delta_{worst}$ remains
below a predefined accuracy threshold $e$ for a specified number
of consecutive iterations, indicating that the population's worst
solutions have stabilized. This approach focuses on monitoring the
collective behavior of the $K$ worst solutions, providing an additional
convergence metric that complements classical termination criteria.
It proves particularly valuable in optimization problems where the
evolution of low-quality solutions serves as a critical factor for
determining the algorithm's overall convergence. The mechanism ensures
the optimization process completes only when stability is achieved
in both the best and worst solutions of the population, offering a
more comprehensive convergence assessment. The $\delta_{worst}$ metric
serves as an effective early-warning system for population stagnation,
particularly useful in multimodal optimization where poor solutions
may indicate unexplored regions of the search space. By incorporating
both solution quality extremes in termination decisions, the algorithm
achieves more robust performance across diverse problem landscapes.

\subsubsection{Solution Range Similarity}

The termination mechanism calculates the range difference

\begin{equation}
\delta_{range}=|(f_{worst}^{(iter)}-f_{best}^{(iter)})-(f_{worst}^{(iter-1)}-f_{best}^{(iter-1)})|,
\end{equation}

when $\delta_{range}\leq\epsilon\quad\text{for }sim\text{ iterations}$,
where $f_{worst}^{(iter)}$ and $f_{best}^{(iter)}$ represent the
worst and best fitness values respectively in the current iteration.
The process terminates when $\delta_{range}$ remains below a specified
threshold $e$ for a predetermined number of iterations, indicating
stabilization of the solution range within the population. This criterion
measures the variation in fitness value range between consecutive
iterations, providing a comprehensive perspective on convergence.
It proves particularly effective in problems where the contraction
of solution diversity serves as a critical convergence indicator.
The mechanism enhances termination reliability by simultaneously monitoring
both optimal and worst solutions, ensuring the algorithm completes
only when the entire population reaches equilibrium. The $\delta_{range}$
metric offers unique advantages in multimodal optimization by capturing
global population dynamics rather than just elite solution behavior.
When maintained below threshold $e$, it indicates the population
has sufficiently explored the search space and is concentrating around
promising regions. This criterion works synergistically with other
termination rules to provide robust convergence detection across various
problem types and population sizes.

\subsubsection{Improvement Rate Similarity}

\begin{equation}
\delta_{imp}=|(f_{worst}^{(iter-1)}-f_{worst}^{(iter)})-(f_{best}^{(iter-1)}-f_{best}^{(iter)})|,
\end{equation}

$\delta_{imp}\leq\epsilon\quad\text{for }sim\text{ iterations}$

\subsubsection{Termination rule: double box}

The parameter $NP$ represents the number of agents participating
in the algorithm. The termination rule is defined as follows: the
process terminates if the value$\delta^{(iter)}$ is less than or
equal to $e$ for a predefined number of iterations $sim$. The termination
criterion is the so-called DoubleBox rule, which was first proposed
in the work of Tsoulos {[}xxx{]}. According to this criterion, the
search process terminates when sufficient coverage of the search space
has been achieved. The coverage estimation is based on the asymptotic
approximation of the relative proportion of points leading to local
minima. Since the exact coverage cannot be directly calculated, sampling
is performed over a wider region. The search is interrupted when the
variance of the sample distribution falls below a predefined threshold,
which is adjusted based on the most recent discovery of a new local
minimum. According to this criterion, the algorithm terminates when
the following condition is met: when the relative variance $\sigma^{(k)}$
falls below half of the variance $\sigma^{(k_{last})}$ from the last
iteration $k_{last}$ where a new optimal function value was found,
that is when

\begin{equation}
\sigma^{(iter)}\leq\frac{\sigma^{(klast)}}{2}
\end{equation}

This logic ensures the algorithm will terminate either when the maximum
allowed number of iterations is exhausted, or when the variance of
results indicates that sufficient exploration of the solution space
has been achieved. The combination of these criteria provides a balanced
termination system that considers both computational cost and solution
quality.

\subsubsection{Termination rule: all}

The work of Charilogis et al. {[}xxx{]} introduces an innovative combination
of termination criteria, which is incorporated and expanded in the
present article through the \textquotedbl All\textquotedbl{} criterion.
This criterion triggers the optimization termination when any of the
individual criteria is satisfied, ensuring an optimal balance between
convergence speed and solution accuracy. Its dynamic adaptability
allows optimal adjustment of the process according to the specific
characteristics of each optimization problem. The \textquotedbl All\textquotedbl{}
criterion operates as a logical disjunction (OR) of all basic termination
rules, offering exceptional flexibility for both serial and parallel
applications. In environments where different algorithms or computing
units may converge at heterogeneous rates, this criterion ensures
immediate termination once satisfactory convergence is achieved, avoiding
unnecessary computations without compromising solution quality. The
main advantages of the \textquotedbl All\textquotedbl{} criterion
include enhanced efficiency with significant reduction in objective
function evaluations, broad adaptability to various problem types
(from smooth to highly multimodal functions), and easy integration
with existing optimization methods. This integration ease makes it
an ideal choice for both academic research and practical applications.
In conclusion, the \textquotedbl All\textquotedbl{} criterion represents
an optimal synthesis of multiple termination mechanisms, offering
an intelligent solution for managing the trade-off between accuracy
and computational cost. The success of this approach highlights the
importance of hybrid methodologies in modern optimization research,
paving the way for further refinements and applications to complex
real-world problems.

\section{Experimental setup and benchmark results\label{sec:Results}}

\paragraph{This section begins with a description of the functions that will
be used in the experiments and then presents in detail the experiments
that were performed, in which the parameters available in the proposed
algorithm were studied, in order to study its reliability and adequacy.
The following is the table \ref{tab:settings} with the relevant parameter
settings of the method.}

\cprotect\paragraph{
\begin{table}[H]
\protect\caption{Parameters and settings\label{tab:settings}}

\protect\centering{}%
\begin{tabular}{|c|c|c|}
\hline 
PARAMETER & VALUE & EXPLANATION\tabularnewline
\hline 
\hline 
 &  & \tabularnewline
\hline 
 &  & \tabularnewline
\hline 
 &  & \tabularnewline
\hline 
 &  & \tabularnewline
\hline 
 &  & \tabularnewline
\hline 
 &  & \tabularnewline
\hline 
\begin{description}
\item 
\end{description}
 &  & \tabularnewline
\hline 
\begin{description}
\item 
\end{description}
 &  & \tabularnewline
\hline 
\begin{description}
\item 
\end{description}
 &  & \tabularnewline
\hline 
\begin{description}
\item 
\end{description}
 &  & \tabularnewline
\hline 
\end{tabular}\protect
\end{table}
}

\subsection{\textbf{Test Functions\label{subsec:testFunctions}}}

The experiments were conducted on a wide range of test functions\citep{testfunc2,testfunc2-1,testfunc4}
as shown in Table \ref{tab:benchmarkFunctions}.

\begin{table}[H]
\caption{The benchmark functions used in the conducted experiments.\label{tab:benchmarkFunctions}}

\centering{}%
\begin{tabular}{|c|c|c|}
\hline 
{\footnotesize NAME} & {\scriptsize FORMULA} & {\scriptsize DIMENSION}\tabularnewline
\hline 
\hline 
{\footnotesize ACKLEY} & {\scriptsize$f(x)=-a\exp\left(-b\sqrt{\frac{1}{n}\sum_{i=1}^{n}x_{i}^{2}}\right)-\exp\left(\frac{1}{n}\sum_{i=1}^{n}\cos\left(cx_{i}\right)\right)+a+\exp(1)\quad a=20.0$} & {\scriptsize 2}\tabularnewline
\hline 
{\footnotesize BF1} & {\scriptsize$f(x)=x_{1}^{2}+2x_{2}^{2}-\frac{3}{10}\cos\left(3\pi x_{1}\right)-\frac{4}{10}\cos\left(4\pi x_{2}\right)+\frac{7}{10}$} & {\scriptsize 2}\tabularnewline
\hline 
{\footnotesize BF2} & {\scriptsize$f(x)=x_{1}^{2}+2x_{2}^{2}-\frac{3}{10}\cos\left(3\pi x_{1}\right)\cos\left(4\pi x_{2}\right)+\frac{3}{10}$} & {\scriptsize 2}\tabularnewline
\hline 
{\footnotesize CAMEL} & {\scriptsize$f(x)=4x_{1}^{2}-2.1x_{1}^{4}+\frac{1}{3}x_{1}^{6}+x_{1}x_{2}-4x_{2}^{2}+4x_{2}^{4},\quad x\in[-5,5]^{2}$} & {\scriptsize 2}\tabularnewline
\hline 
{\footnotesize CM} & {\scriptsize$f(x)=\sum_{i=1}^{n}x_{i}^{2}-\frac{1}{10}\sum_{i=1}^{n}\cos\left(5\pi x_{i}\right)$} & {\scriptsize 4, 10}\tabularnewline
\hline 
{\footnotesize DIFFPOWER} & {\scriptsize$f(x)=\sum_{i=1}^{n}|x_{i}-y_{i}|^{p}$} & {\scriptsize$n=2\ p=2,5,10$}\tabularnewline
\hline 
{\footnotesize BRANIN} & \begin{cellvarwidth}[t]
\centering
{\scriptsize$f(x)=\left(x_{2}-\frac{5.1}{4\pi^{2}}x_{1}^{2}+\frac{5}{\pi}x_{1}-6\right)^{2}+10\left(1-\frac{1}{8\pi}\right)\cos(x_{1})+10$}{\scriptsize\par}

{\scriptsize$-5\le x_{1}\le10,\ 0\le x_{2}\le15$}
\end{cellvarwidth} & {\scriptsize 2}\tabularnewline
\hline 
{\footnotesize DISCUS} & {\scriptsize$f(x)=10^{6}x_{1}^{2}+\sum_{i=2}^{n}x_{i}^{2}$} & {\scriptsize 10}\tabularnewline
\hline 
{\footnotesize EASOM} & {\scriptsize$f(x)=-\cos\left(x_{1}\right)\cos\left(x_{2}\right)\exp\left(\left(x_{2}-\pi\right)^{2}-\left(x_{1}-\pi\right)^{2}\right)$} & {\scriptsize 2}\tabularnewline
\hline 
{\footnotesize ELP} & {\scriptsize$f(x)=\sum_{i=1}^{n}\left(10^{6}\right)^{\frac{i-1}{n-1}}x_{i}^{2}$} & {\scriptsize$n=10$}\tabularnewline
\hline 
{\footnotesize EXP} & {\scriptsize$f(x)=-\exp\left(-0.5\sum_{i=1}^{n}x_{i}^{2}\right),\quad-1\le x_{i}\le1$} & {\scriptsize$n=16$}\tabularnewline
\hline 
{\footnotesize F5} & {\scriptsize$f(x)=\left(\left(4.0-2.1x_{1}^{2}+\frac{x_{1}^{4}}{3.0}\right)x_{1}^{2}\right)+\left(x_{1}x_{2}\right)+\left(\left(4.0x_{2}^{2}-4.0\right)x_{2}^{2}\right)\quad-5\le x_{i}\le5$} & {\scriptsize 2}\tabularnewline
\hline 
{\footnotesize F9} & {\scriptsize$f(x)=-\exp\left(-0.5\sum_{i=1}^{n}x_{i}^{2}\right),\quad x\in[0,1]^{n}$} & {\scriptsize 2}\tabularnewline
\hline 
{\footnotesize F12} & \begin{cellvarwidth}[t]
\centering
{\scriptsize$f(x)=\frac{\pi}{n}\left(10\sin\left(\pi y_{1}\right)+\sum_{i=1}^{n-1}\left(\left(y_{i}-1\right)^{2}\left(1+10\sin^{2}\left(\pi y_{i+1}\right)\right)\right)+\left(y_{n}-1\right)^{2}\right)$}{\scriptsize\par}

{\scriptsize$+\sum_{i=1}^{n}u\left(x_{i},10,100,4\right)$}
\end{cellvarwidth} & {\scriptsize 2}\tabularnewline
\hline 
{\footnotesize F13} & {\scriptsize$f(x)=\sum_{i=1}^{n}\frac{x_{i}^{2}}{4000}-\prod_{i=1}^{n}\cos\left(\frac{x_{i}}{\sqrt{i}}\right)+1$} & {\scriptsize 2}\tabularnewline
\hline 
{\footnotesize F14} & {\scriptsize$f(x)=\left(\frac{1}{500}+\sum_{j=1}^{25}\frac{1}{j+\sum_{i=1}^{n}\left(x_{i}-a_{ij}\right)^{6}}\right)^{-1}$} & {\scriptsize 2}\tabularnewline
\hline 
{\footnotesize F15} & {\scriptsize$f(x)=\sum_{i=1}^{11}\left(a_{i}-\frac{x_{1}\left(b_{i}+b_{i}x_{2}\right)}{b_{i}^{2}+b_{i}x_{3}+x_{4}}\right)^{2}$} & {\scriptsize 2}\tabularnewline
\hline 
{\footnotesize F18} & {\scriptsize$f(x)=-\sum_{i=1}^{4}c_{1}\exp\left(-\sum_{j=1}^{n}a_{ij}\left(x_{j}-p_{ij}\right)^{2}\right)\quad c_{1}=0.965$} & {\scriptsize 2}\tabularnewline
\hline 
{\footnotesize F19} & {\scriptsize$f(x)=-\sum_{i=1}^{4}c_{1}\exp\left(-\sum_{j=1}^{n}a_{ij}\left(x_{j}-p_{ij}\right)^{2}\right)\quad c_{1}=0.83$} & {\scriptsize 2}\tabularnewline
\hline 
{\footnotesize GKLS\citep{Gaviano}} & {\scriptsize$f(x)=\mbox{Gkls}(x,n,w)$} & {\scriptsize$n=2,3\ w=50,100$}\tabularnewline
\hline 
{\footnotesize GRIEWANK2} & {\scriptsize$f(x)=1+\frac{1}{200}\sum_{i=1}^{2}x_{i}^{2}-\prod_{i=1}^{2}\frac{\cos(x_{i})}{\sqrt{(i)}}$} & {\scriptsize 2}\tabularnewline
\hline 
{\footnotesize GRIEWANK10} & {\scriptsize f$(x)=1+\frac{1}{200}\sum_{i=1}^{10}x_{i}^{2}-\prod_{i=1}^{10}\frac{\cos(x_{i})}{\sqrt{(i)}}$} & {\scriptsize 10}\tabularnewline
\hline 
{\footnotesize HANSEN} & {\scriptsize$f(x)=\sum_{i=1}^{5}i\cos\left[(i-1)x_{1}+i\right]\sum_{j=1}^{5}j\cos\left[(j+1)x_{2}+j\right]$} & {\scriptsize 2}\tabularnewline
\hline 
{\footnotesize HARTMAN3} & {\scriptsize$f(x)=-\sum_{i=1}^{4}c_{i}\exp\left(-\sum_{j=1}^{3}a_{ij}\left(x_{j}-p_{ij}\right)^{2}\right)$} & {\scriptsize 3}\tabularnewline
\hline 
{\footnotesize HARTAMN6} & {\scriptsize$f(x)=-\sum_{i=1}^{4}c_{i}\exp\left(-\sum_{j=1}^{6}a_{ij}\left(x_{j}-p_{ij}\right)^{2}\right)$} & {\scriptsize 6}\tabularnewline
\hline 
{\footnotesize POTENTIAL\citep{Lennard}} & {\scriptsize$V_{LJ}(r)=4\epsilon\left[\left(\frac{\sigma}{r}\right)^{12}-\left(\frac{\sigma}{r}\right)^{6}\right]$} & {\scriptsize$n=9,15,21,30$}\tabularnewline
\hline 
{\footnotesize RARSTIGIN} & {\scriptsize$f(x)=x_{1}^{2}+x_{2}^{2}-\cos(18x_{1})-\cos(18x_{2})$} & {\scriptsize 2}\tabularnewline
\hline 
{\footnotesize ROSENBROCK} & {\tiny$f(x)=\sum_{i=1}^{n-1}\left(100\left(x_{i+1}-x_{i}^{2}\right)^{2}+\left(x_{i}-1\right)^{2}\right),\quad-30\le x_{i}\le30$} & {\scriptsize$n=4,8$}\tabularnewline
\hline 
{\footnotesize SCHWEFEL} & {\tiny$f(x)=\sum_{i=1}^{n}\left(\sum_{j=1}^{i}x_{j}\right)^{2}$} & {\scriptsize 2}\tabularnewline
\hline 
{\footnotesize SCHWEFEL221} & {\tiny$f(x)=418.9829n+\sum_{i=1}^{n}-x_{i}\sin\left(\sqrt{\left|x_{i}\right|}\right)$} & {\scriptsize 2}\tabularnewline
\hline 
{\footnotesize SCHWEFEL222} & {\tiny$f(x)=\sum_{i=1}^{n-1}\left(100\left(x_{i+1}-x_{i}^{2}\right)^{2}+\left(x_{i}-1\right)^{2}\right),\quad-30\le x_{i}\le30$} & {\scriptsize 2}\tabularnewline
\hline 
{\footnotesize Shekel5} & {\scriptsize$f(x)=-\sum_{i=1}^{5}\frac{1}{(x-a_{i})(x-a_{i})^{T}+c_{i}}$} & {\scriptsize 4}\tabularnewline
\hline 
{\footnotesize Shekel7} & {\scriptsize$f(x)=-\sum_{i=1}^{7}\frac{1}{(x-a_{i})(x-a_{i})^{T}+c_{i}}$} & {\scriptsize 4}\tabularnewline
\hline 
{\footnotesize Shekel10} & {\scriptsize$f(x)=-\sum_{i=1}^{10}\frac{1}{(x-a_{i})(x-a_{i})^{T}+c_{i}}$} & {\scriptsize 4}\tabularnewline
\hline 
{\footnotesize Sinusoidal\citep{Zabinsky}} & {\scriptsize$f(x)=-\left(2.5\prod_{i=1}^{n}\sin\left(x_{i}-z\right)+\prod_{i=1}^{n}\sin\left(5\left(x_{i}-z\right)\right)\right),\quad0\le x_{i}\le\pi$} & {\scriptsize$n=4,8$}\tabularnewline
\hline 
{\footnotesize Test2N} & {\scriptsize$f(x)=\frac{1}{2}\sum_{i=1}^{n}x_{i}^{4}-16x_{i}^{2}+5x_{i}$} & {\scriptsize$n=4,5,6$}\tabularnewline
\hline 
{\footnotesize Test30N} & {\scriptsize$\frac{1}{10}\sin^{2}\left(3\pi x_{1}\right)\sum_{i=2}^{n-1}\left(\left(x_{i}-1\right)^{2}\left(1+\sin^{2}\left(3\pi x_{i+1}\right)\right)\right)+\left(x_{n}-1\right)^{2}\left(1+\sin^{2}\left(2\pi x_{n}\right)\right)$} & {\scriptsize$n=3,4$}\tabularnewline
\hline 
\end{tabular}
\end{table}


\subsection{\textbf{Experimental results\label{subsec:experimentalResults}} }

For the aforementioned functions, a series of tests was conducted
on a computer equipped with an AMD Ryzen 5950X processor and 128GB
of RAM, running Debian Linux. Each test was repeated 30 times with
new random values in each iteration, and the average results were
recorded. The tool used was developed in ANSI C++ using the GLOBALOPTIMUS
platform, which is open-source and available at \url{https://github.com/itsoulos/GLOBALOPTIMUS}
(last accessed: 8/4/2025). The parameter settings of the method are
shown in Table \ref{tab:settings}.

In the following experimental results, the values in the cells correspond
to the average number of function calls over 30 repetitions. The numbers
in parentheses indicate the percentage of cases where the method successfully
found the global minimum. If no parentheses are present, it means
the method was 100\% successful in all tests.

\begin{table}[H]

\caption{Comparison of classic DE with new DE and new DE with majority dimension
mechanism}

\begin{centering}
\begin{tabular}{|l|c|c|c|}
\hline 
FUNCTION & DE & NEWDE & NEWDE+MDM\tabularnewline
\hline 
\hline 
ACKLAY & 19688 & 15089 & 7249\tabularnewline
\hline 
BF1 & 10683 & 8537 & 5204\tabularnewline
\hline 
BF2 & 10419 & 8416 & 4775\tabularnewline
\hline 
BF3 & 10013 & 7572 & 4285\tabularnewline
\hline 
BRANIN & 5670 & 5852 & 2762\tabularnewline
\hline 
CAMEL & 8116 & 9206 & 5358\tabularnewline
\hline 
DIFFPOWER2 & 14547 & 11793 & 10734\tabularnewline
\hline 
DIFFPOWER5 & 31332 & 33168 & 25736\tabularnewline
\hline 
DIFFPOWER10 & 39317 & 38228 & 40297\tabularnewline
\hline 
EASOM & 4662 & 4589 & 3664\tabularnewline
\hline 
ELP10 & 7834 & 7471 & 6053\tabularnewline
\hline 
ELP20 & 10648 & 10180 & 8809\tabularnewline
\hline 
ELP30 & 13003 & 12573 & 10957\tabularnewline
\hline 
EXP4 & 6986 & 7134 & 3824\tabularnewline
\hline 
EXP8 & 7326 & 7141 & 4329\tabularnewline
\hline 
GKLS250 & 7238 & 9351 & 3181\tabularnewline
\hline 
GKLS350 & 8107 & 8314 & 1811(0.96)\tabularnewline
\hline 
GOLDSTEIN & 9013 & 7751 & 5026\tabularnewline
\hline 
GRIEWANK2 & 12785 & 10309 & 4397(0.66)\tabularnewline
\hline 
GRIEWANK10 & 16527 & 16607 & 9553\tabularnewline
\hline 
HANSEN & 7748 & 7768 & 6432\tabularnewline
\hline 
HARTMAN3 & 6434 & 6450 & 3179\tabularnewline
\hline 
HARTMAN6 & 7181 & 6827 & 4790\tabularnewline
\hline 
POTENTIAL3 & 7871 & 7677 & 5994\tabularnewline
\hline 
POTENTIAL5 & 11961 & 11698 & 10653\tabularnewline
\hline 
POTENTIAL6 & 15460(0.56) & 15449(0.7) & 12775(0.86)\tabularnewline
\hline 
POTENTIAL10 & 22602 & 21695 & 20237\tabularnewline
\hline 
RASTRIGIN & 10597 & 9556 & 4639(0.93)\tabularnewline
\hline 
ROSENBROCK4 & 10336 & 10513 & 8729\tabularnewline
\hline 
ROSENBROCK8 & 12909 & 12799 & 11259\tabularnewline
\hline 
ROSENBROCK16 & 16527 & 16920 & 15377\tabularnewline
\hline 
SHEKEL5 & 7772 & 7533 & 5306\tabularnewline
\hline 
SHEKEL7 & 8013 & 7586 & 5052\tabularnewline
\hline 
SHEKEL10 & 8306 & 7527 & 5120\tabularnewline
\hline 
SINU4 & 8904 & 7515 & 6723\tabularnewline
\hline 
SINU8 & 8968 & 7411 & 6795\tabularnewline
\hline 
SINU16 & 12241 & 10472 & 8577\tabularnewline
\hline 
TEST2N4 & 7874 & 8107 & 4292\tabularnewline
\hline 
TEST2N5 & 9347 & 9109 & 3907(0.96)\tabularnewline
\hline 
TEST2N7 & 12006 & 11153(0.9) & 4565(0.7)\tabularnewline
\hline 
TEST30N3 & 7296 & 7572 & 4558\tabularnewline
\hline 
TEST30N4 & 9103 & 8804 & 5958\tabularnewline
\hline 
TOTAL & 483370(0.98) & 459422(0.99) & 332921(0.97)\tabularnewline
\hline 
\end{tabular}
\par\end{centering}
\end{table}

\begin{table}[H]

\caption{Comparison new DE with majority dimension mechanism method versus
others}

\centering{}%
\begin{tabular}{|c|c|c|c|c|c|}
\hline 
FUNCTION & GA & WOA & PSO & ACO & NEWDE+MDM\tabularnewline
\hline 
\hline 
ACKLAY & 10436 & 33824 & 9279 & 9091 & 7249\tabularnewline
\hline 
BF1 & 6265 & 13475 & 6311 & 7047(0.66) & 5204\tabularnewline
\hline 
BF2 & 5995 & 13693 & 5884 & 6908(0.76) & 4775\tabularnewline
\hline 
BF3 & 5559 & 21432 & 5509 & 6450(0.83) & 4285\tabularnewline
\hline 
BRANIN & 4536 & 8619 & 4667 & 5944 & 2762\tabularnewline
\hline 
CAMEL & 5017 & 8902 & 5050 & 4757 & 5358\tabularnewline
\hline 
DIFFPOWER2 & 8704 & 13806 & 10988 & 11556 & 10734\tabularnewline
\hline 
DIFFPOWER5 & 23774 & 41028 & 27029 & 55183 & 25736\tabularnewline
\hline 
DIFFPOWER10 & 25511 & 58060 & 34319 & 84710 & 40297\tabularnewline
\hline 
EASOM & 4080 & 5437 & 4134 & 4203 & 3664\tabularnewline
\hline 
ELP10 & 5663 & 22804 & 6588 & 4637 & 6053\tabularnewline
\hline 
ELP20 & 8800 & 36649 & 8953 & 5065 & 8809\tabularnewline
\hline 
ELP30 & 12757 & 42506 & 11075 & 5416 & 10957\tabularnewline
\hline 
EXP4 & 5163 & 8397 & 5163 & 5935 & 3824\tabularnewline
\hline 
EXP8 & 5318 & 11478 & 5440 & 6197 & 4329\tabularnewline
\hline 
GKLS250 & 4575 & 7612 & 4628 & 4459 & 3181\tabularnewline
\hline 
GKLS350 & 5184 & 10088 & 4769 & 4614 & 1811(0.96)\tabularnewline
\hline 
GOLDSTEIN & 5932 & 11804 & 6051 & 7248 & 5026\tabularnewline
\hline 
GRIEWANK2 & 7485 & 11135(0.83) & 5317 & 6430(0.43) & 4397(0.63)\tabularnewline
\hline 
GRIEWANK10 & 9393 & 51041 & 10239 & 7848 & 9553\tabularnewline
\hline 
HANSEN & 6025 & 15091 & 5033 & 5091(0.66) & 6432\tabularnewline
\hline 
HARTMAN3 & 4936 & 10911 & 5102 & 5408 & 3179\tabularnewline
\hline 
HARTMAN6 & 5419 & 21302 & 5825 & 6154(0.7) & 4790\tabularnewline
\hline 
POTENTIAL3 & 6455 & 12705 & 6998 & 6854 & 5994\tabularnewline
\hline 
POTENTIAL5 & 9878 & 66605 & 12339 & 9702 & 10653\tabularnewline
\hline 
POTENTIAL6 & 13891(0.76) & 10648(0.93) & 13945(0.46) & 11166(0.06) & 12775(0.86)\tabularnewline
\hline 
POTENTIAL10 & 16834 & 197044 & 17948 & 12540(0.3) & 20237\tabularnewline
\hline 
RASTRIGIN & 6868 & 10530 & 5756 & 5346(0.4) & 4639(0.93)\tabularnewline
\hline 
ROSENBROCK4 & 6414 & 18576 & 7611 & 4848 & 8729\tabularnewline
\hline 
ROSENBROCK8 & 8128 & 25777 & 10198 & 5374 & 11259\tabularnewline
\hline 
ROSENBROCK16 & 11678 & 37759 & 13529 & 5987 & 15377\tabularnewline
\hline 
SHEKEL5 & 5705 & 22886 & 5915 & 6815(0.56) & 5306\tabularnewline
\hline 
SHEKEL7 & 5741 & 26964 & 5938 & 6777(0.63) & 5052\tabularnewline
\hline 
SHEKEL10 & 5829 & 20334 & 5915 & 6670(0.46) & 5120\tabularnewline
\hline 
SINU4 & 5334 & 13266 & 5355 & 5687(0.73) & 6723\tabularnewline
\hline 
SINU8 & 5839 & 21358 & 6188 & 6472(0.9) & 6795\tabularnewline
\hline 
SINU16 & 7278 & 47713 & 6745 & 8465(0.73) & 8577\tabularnewline
\hline 
TEST2N4 & 5813 & 16104 & 5339 & 5752(0.56) & 4292\tabularnewline
\hline 
TEST2N5 & 6516 & 18131 & 5644 & 5893(0.36) & 3907(0.76)\tabularnewline
\hline 
TEST2N7 & 8205(0.96) & 23489(0.63) & 6057(0.93) & 6199(0.03) & 4565(0.7)\tabularnewline
\hline 
TEST30N3 & 5635 & 11307 & 5634 & 6591 & 4558\tabularnewline
\hline 
TEST30N4 & 6594 & 18229 & 6464 & 8813 & 5958\tabularnewline
\hline 
TOTAL & 335162(0.99) & 1098519(0.98) & 350871(0.98) & 406302(0.8) & 332921(0.97)\tabularnewline
\hline 
\end{tabular}
\end{table}

\begin{table}[H]

\caption{The proposed method with new different stopping rules}

\begin{centering}
\begin{tabular}{|c|c|c|c|c|c|c|c|c|}
\hline 
FUNCTION & BSS & WSS & TSS & BOSS & SRS & IRS & doublebox & All\tabularnewline
\hline 
\hline 
ACKLEY & 7249 & 8063 & 21423 & 45754 & 8664 & 6187 & 8601 & 6187\tabularnewline
\hline 
BF1 & 5204 & 5770 & 6559 & 41602 & 6002 & 4539 & 5564 & 4539\tabularnewline
\hline 
BF2 & 4775 & 5297 & 5701 & 34369 & 5522 & 4077 & 5647 & 4077\tabularnewline
\hline 
BF3 & 4285 & 4482 & 5261 & 28391 & 4694 & 3728 & 4999 & 3728\tabularnewline
\hline 
BRANIN & 2762 & 3639 & 3324 & 15776 & 3696 & 2603 & 2772 & 2603\tabularnewline
\hline 
CAMEL & 5358 & 4386 & 5811 & 22948 & 5843 & 3497 & 10667 & 3497\tabularnewline
\hline 
DIFFPOWER2 & 10734 & 12276 & 9767 & 28349 & 12398 & 9767 & 11172 & 9767\tabularnewline
\hline 
DIFFPOWER5 & 25736 & 32486 & 24771 & 92201 & 34880 & 24771 & 26377 & 24771\tabularnewline
\hline 
DIFFPOWER10 & 40297 & 52420 & 31056 & 91219 & 58127 & 31056 & 55305 & 31056\tabularnewline
\hline 
EASOM & 3664 & 4894 & 4714 & 2789 & 4944 & 3242 & 3800 & 2789\tabularnewline
\hline 
ELP10 & 6053 & 5133 & 5930 & 21980 & 6459 & 4307 & 12313 & 4307\tabularnewline
\hline 
ELP20 & 8809 & 8781 & 9387 & 29239 & 10094 & 6587 & 18033 & 6587\tabularnewline
\hline 
ELP30 & 10957 & 10871 & 12290 & 23745 & 12533 & 8672 & 22989 & 8672\tabularnewline
\hline 
EXP4 & 3824 & 4408 & 4948 & 3092 & 4593 & 3575 & 4270 & 3092\tabularnewline
\hline 
EXP8 & 4329 & 4878 & 4944 & 3052 & 5457 & 3522 & 7516 & 3052\tabularnewline
\hline 
GKLS250 & 3181 & 3245 & 9199 & 7690 & 3245 & 2919 & 3977 & 2919\tabularnewline
\hline 
GKLS350 & 1811(0.96) & 2142(0.96) & 3268(0.96) & 10756(0.96) & 2142(0.96) & 1811(0.96) & 1811(0.96) & 1811(0.96)\tabularnewline
\hline 
GOLDSTEIN & 5026 & 5576 & 7682 & 38776 & 5867 & 4364 & 5169 & 4364\tabularnewline
\hline 
GRIEWANK2 & 4397(0.63) & 4761(0.63) & 8722(0.80) & 29118(0.96) & 5212(0.66) & 3559(0.53) & 4898(0.63) & 3599(0.53)\tabularnewline
\hline 
GRIEWANK10 & 9553 & 10698 & 11061 & 62990 & 12085 & 7866 & 16496 & 7866\tabularnewline
\hline 
HANSEN & 6432 & 8143 & 14493 & 88770 & 9146 & 4238 & 13116 & 4238\tabularnewline
\hline 
HARTMAN3 & 3179 & 3898 & 3073 & 2651 & 4121 & 3042 & 3643 & 2651\tabularnewline
\hline 
HARTMAN6 & 4790 & 4903 & 5811 & 3400 & 5521 & 3981 & 7438 & 3400\tabularnewline
\hline 
POTENTIAL3 & 5994 & 7416 & 6592 & 20224 & 7691 & 4683 & 11358 & 4683\tabularnewline
\hline 
POTENTIAL5 & 10653 & 12221 & 11679 & 58988 & 13108 & 8224 & 30085 & 8224\tabularnewline
\hline 
POTENTIAL6 & 12775(0.86) & 13968(0.86) & 13720(0.90) & 74701 & 15690(0.93) & 9616(0.7) & 35090(0.96) & 9616(0.7)\tabularnewline
\hline 
POTENTIAL10 & 20237 & 17453 & 46851(0.93) & 101111 & 23157 & 12650 & 56930 & 12650\tabularnewline
\hline 
RASTRIGIN & 4639(0.93) & 5063(0.96) & 6900 & 15210 & 5545(0.96) & 3548(0.93) & 6473(0.93) & 3548(0.93)\tabularnewline
\hline 
ROSENBROCK4 & 8729 & 8284 & 9075 & 34803 & 9794 & 6583 & 13961 & 6583\tabularnewline
\hline 
ROSENBROCK8 & 11259 & 10337 & 11512 & 51878 & 12605 & 8104 & 22692 & 8104\tabularnewline
\hline 
ROSENBROCK16 & 15377 & 17661 & 16644 & 64581 & 20007 & 11826 & 25043 & 11826\tabularnewline
\hline 
SHEKEL5 & 5306 & 6204 & 6925 & 3672 & 6452 & 4262 & 15952 & 3672\tabularnewline
\hline 
SHEKEL7 & 5052 & 6224 & 6907 & 3528 & 6463 & 4081 & 16879 & 3528\tabularnewline
\hline 
SHEKEL10 & 5120 & 5813 & 8051 & 3516 & 6232 & 4055 & 14980 & 3516\tabularnewline
\hline 
SINU4 & 6723 & 6508 & 11594 & 6730 & 8163 & 4310 & 20602 & 4310\tabularnewline
\hline 
SINU8 & 6795 & 8776 & 13952 & 4309 & 9631 & 4971 & 46245 & 4293\tabularnewline
\hline 
SINU16 & 8577 & 13879 & 30354 & 5761 & 14369 & 6785 & 11051 & 5761\tabularnewline
\hline 
TEST2N4 & 4292 & 4992 & 12290 & 26112 & 5427 & 3290 & 5930 & 3290\tabularnewline
\hline 
TEST2N5 & 3907(0.96) & 5199(0.96) & 10493 & 21617 & 5406(0.96) & 3226(0.93) & 4791 & 3226(0.93)\tabularnewline
\hline 
TEST2N7 & 4565(0.7) & 5020(0.7) & 12859(0.86) & 19407(0.86) & 5709(0.76) & 3510(0.56) & 6396(0.76) & 3510(0.56)\tabularnewline
\hline 
TEST30N3 & 4558 & 5402 & 3806 & 25347 & 6063 & 3384 & 8178 & 3384\tabularnewline
\hline 
TEST30N4 & 5958 & 6551 & 5880 & 48097 & 7383 & 4564 & 14847 & 4564\tabularnewline
\hline 
TOTAL & 332921(0.97) & 378121(0.97) & 465279(0.98) & 1318249(0.99) & 420140(0.98) & 263582(0.96) & 624056(0.98) & 257860(0.96)\tabularnewline
\hline 
\end{tabular}
\par\end{centering}
\end{table}


\section{Discussion of findings and comparative analysis\label{sec:discussion}}

\section{Conclusions and future research directions\label{sec:Conclusions}}

\institutionalreview{Not Applicable.}

\informedconsent{Not applicable.}

\acknowledgments{This research has been financed by the European Union: Next Generation
EU through the Program Greece 2.0 National Recovery and Resilience
Plan, under the call RESEARCH--CREATE--INNOVATE, project name “iCREW:
Intelligent small craft simulator for advanced crew training using
Virtual Reality techniques” (project code: TAEDK-06195).}

\conflictsofinterest{The authors declare no conflicts of interest.}

\appendixtitles{no}

\begin{adjustwidth}{-\extralength}{0cm}{}

\begin{thebibliography}{999}
\bibitem{go_math1}Carrizosa, E., Molero-Río, C., \& Romero Morales,
D. (2021). Mathematical optimization in classification and regression
trees. Top, 29(1), 5-33.

\bibitem{go_math3}Legat, B., Dowson, O., Garcia, J. D., \& Lubin,
M. (2022). MathOptInterface: a data structure for mathematical optimization
problems. INFORMS Journal on Computing, 34(2), 672-689.

\bibitem{go_physics2}Su, H., Zhao, D., Heidari, A. A., Liu, L., Zhang,
X., Mafarja, M., \& Chen, H. (2023). RIME: A physics-based optimization.
Neurocomputing, 532, 183-214.

\bibitem{go_physics3}Stilck França, D., \& Garcia-Patron, R. (2021).
Limitations of optimization algorithms on noisy quantum devices. Nature
Physics, 17(11), 1221-1227.

\bibitem{go_chem1}Zhang, J., \& Glezakou, V. A. (2021). Global optimization
of chemical cluster structures: Methods, applications, and challenges.
International Journal of Quantum Chemistry, 121(7), e26553.

\bibitem{go_chem2}Hu, Y., Zang, Z., Chen, D., Ma, X., Liang, Y.,
You, W., \& Zhang, Z. (2022). Optimization and evaluation of SO2 emissions
based on WRF-Chem and 3DVAR data assimilation. Remote Sensing, 14(1),
220.

\bibitem{go_med2}Kaur, P., \& Singh, R. K. (2023). A review on optimization
techniques for medical image analysis. Concurrency and Computation:
Practice and Experience, 35(1), e7443.

\bibitem{medicine}Houssein, E. H., Hosney, M. E., Mohamed, W. M.,
Ali, A. A., \& Younis, E. M. (2023). Fuzzy-based hunger games search
algorithm for global optimization and feature selection using medical
data. Neural Computing and Applications, 35(7), 5251-5275.

\bibitem{go_bio1}Wang, L., Cao, Q., Zhang, Z., Mirjalili, S., \&
Zhao, W. (2022). Artificial rabbits optimization: A new bio-inspired
meta-heuristic algorithm for solving engineering optimization problems.
Engineering Applications of Artificial Intelligence, 114, 105082.

\bibitem{go_bio2}Hesami, M., \& Jones, A. M. P. (2020). Application
of artificial intelligence models and optimization algorithms in plant
cell and tissue culture. Applied Microbiology and Biotechnology, 104(22),
9449-9485.

\bibitem{go_agri1}Filip, M., Zoubek, T., Bumbalek, R., Cerny, P.,
Batista, C. E., Olsan, P., ... \& Findura, P. (2020). Advanced computational
methods for agriculture machinery movement optimization with applications
in sugarcane production. Agriculture, 10(10), 434.

\bibitem{go_agri2}Akintuyi, O. B. (2024). Adaptive AI in precision
agriculture: a review: investigating the use of self-learning algorithms
in optimizing farm operations based on real-time data. Research Journal
of Multidisciplinary Studies, 7(02), 016-030.

\bibitem{go_econ1}Wang, Y., Ma, Y., Song, F., Ma, Y., Qi, C., Huang,
F., ... \& Zhang, F. (2020). Economic and efficient multi-objective
operation optimization of integrated energy system considering electro-thermal
demand response. Energy, 205, 118022.

\bibitem{go_econ2}Alirahmi, S. M., Mousavi, S. B., Razmi, A. R.,
\& Ahmadi, P. (2021). A comprehensive techno-economic analysis and
multi-criteria optimization of a compressed air energy storage (CAES)
hybridized with solar and desalination units. Energy Conversion and
Management, 236, 114053.

\bibitem{interval1}Wolfe, M. A. (1996). Interval methods for global
optimization. Applied Mathematics and Computation, 75(2-3), 179-206.

\bibitem{interval2}Csendes, T., \& Ratz, D. (1997). Subdivision direction
selection in interval methods for global optimization. SIAM Journal
on Numerical Analysis, 34(3), 922-938.

\bibitem{go_determ1}Shezan, S. A., Ishraque, M. F., Shafiullah, G.
M., Kamwa, I., Paul, L. C., Muyeen, S. M., ... \& Kumar, P. P. (2023).
Optimization and control of solar-wind islanded hybrid microgrid by
using heuristic and deterministic optimization algorithms and fuzzy
logic controller. Energy reports, 10, 3272-3288.

\bibitem{go_determ3}Xu, Z., Zhao, Z., \& Liu, J. (2024). Deterministic
Multi-Objective Optimization of Analog Circuits. Electronics, 13(13),
2510.

\bibitem{stohastic}Hsieh, Y. P., Karimi Jaghargh, M. R., Krause,
A., \& Mertikopoulos, P. (2024). Riemannian stochastic optimization
methods avoid strict saddle points. Advances in Neural Information
Processing Systems, 36.

\bibitem{stohastic1}Tyurin, A., \& Richtárik, P. (2024). Optimal
time complexities of parallel stochastic optimization methods under
a fixed computation model. Advances in Neural Information Processing
Systems, 36.

\bibitem{key-2}Sohail, A. (2023). Genetic algorithms in the fields
of artificial intelligence and data sciences. Annals of Data Science,
10(4), 1007-1018.

\bibitem{genetic3}Charilogis, V., Tsoulos, I. G., \& Stavrou, V.
N. (2023). An Intelligent Technique for Initial Distribution of Genetic
Algorithms. Axioms, 12(10), 980.

\bibitem{diffe1}Deng, W., Shang, S., Cai, X., Zhao, H., Song, Y.,
\& Xu, J. (2021). An improved differential evolution algorithm and
its application in optimization problem. Soft Computing, 25, 5277-5298.

\bibitem{diffe2}Pant, M., Zaheer, H., Garcia-Hernandez, L., \& Abraham,
A. (2020). Differential Evolution: A review of more than two decades
of research. Engineering Applications of Artificial Intelligence,
90, 103479.

\bibitem{key-29}Price, W. (1983). Global optimization by controlled
random search. Journal of optimization theory and applications, 40,
333-348.

\bibitem{crs2}Křivý, I., \& Tvrdik, J. (1995). The controlled random
search algorithm in optimizing regression models. Computational statistics
\& data analysis, 20(2), 229-234.

\bibitem{simann1}Ingber, L. (1989). Very fast simulated re-annealing.
Mathematical and computer modelling, 12(8), 967-973.

\bibitem{simann2}Eglese, R. W. (1990). Simulated annealing: a tool
for operational research. European journal of operational research,
46(3), 271-281.

\bibitem{mult_minfinder}Tsoulos, I. G., \& Lagaris, I. E. (2006).
MinFinder: Locating all the local minima of a function. Computer Physics
Communications, 174(2), 166-179.

\bibitem{mult_cfo}Liu, Y., \& Tian, P. (2015). A multi-start central
force optimization for global optimization. Applied Soft Computing,
27, 92-98.

\bibitem{pso_major}Shami, T. M., El-Saleh, A. A., Alswaitti, M.,
Al-Tashi, Q., Summakieh, M. A., \& Mirjalili, S. (2022). Particle
swarm optimization: A comprehensive survey. Ieee Access, 10, 10031-10061.

\bibitem{pso1}Gad, A. G. (2022). Particle swarm optimization algorithm
and its applications: a systematic review. Archives of computational
methods in engineering, 29(5), 2531-2561. 

\bibitem{fish}Pourpanah, F., Wang, R., Lim, C. P., Wang, X. Z., \&
Yazdani, D. (2023). A review of artificial fish swarm algorithms:
Recent advances and applications. Artificial Intelligence Review,
56(3), 1867-1903.

\bibitem{key-18}Zhang, C., Zhang, F. M., Li, F., \& Wu, H. S. (2014,
June). Improved artificial fish swarm algorithm. In 2014 9th IEEE
conference on industrial electronics and applications (pp. 748-753).
IEEE.

\bibitem{dolphin}Kareem, S. W., Mohammed, A. S., \& Khoshabaa, F.
S. (2023). Novel nature-inspired meta-heuristic optimization algorithm
based on hybrid dolphin and sparrow optimization. International Journal
of Nonlinear Analysis and Applications, 14(1), 355-373. 

\bibitem{key-17}Wu, T. Q., Yao, M., \& Yang, J. H. (2016). Dolphin
swarm algorithm. Frontiers of Information Technology \& Electronic
Engineering, 17(8), 717-729.

\bibitem{WOA}Nadimi-Shahraki, M. H., Zamani, H., Asghari Varzaneh,
Z., \& Mirjalili, S. (2023). A systematic review of the whale optimization
algorithm: theoretical foundation, improvements, and hybridizations.
Archives of Computational Methods in Engineering, 30(7), 4113-4159.

\bibitem{WOA1}Brodzicki, A., Piekarski, M., \& Jaworek-Korjakowska,
J. (2021). The whale optimization algorithm approach for deep neural
networks. Sensors, 21(23), 8003.

\bibitem{key-14}Rokbani, N., Kumar, R., Abraham, A., Alimi, A. M.,
Long, H. V., Priyadarshini, I., \& Son, L. H. (2021). Bi-heuristic
ant colony optimization-based approaches for traveling salesman problem.
Soft Computing, 25, 3775-3794.

\bibitem{aco2}Wu, L., Huang, X., Cui, J., Liu, C., \& Xiao, W. (2023).
Modified adaptive ant colony optimization algorithm and its application
for solving path planning of mobile robot. Expert Systems with Applications,
215, 119410.

\bibitem{key-4}Abualigah, L., Yousri, D., Abd Elaziz, M., Ewees,
A. A., Al-Qaness, M. A., \& Gandomi, A. H. (2021). Aquila optimizer:
a novel meta-heuristic optimization algorithm. Computers \& Industrial
Engineering, 157, 107250.

\bibitem{key-5}Zhao, J., Gao, Z. M., \& Chen, H. F. (2022). The simplified
aquila optimization algorithm. IEEE Access, 10, 22487-22515.

\bibitem{key-19}Abualigah, L., Sbenaty, B., Ikotun, A. M., Zitar,
R. A., Alsoud, A. R., Khodadadi, N., ... \& Jia, H. (2024). Aquila
optimizer: review, results and applications. Metaheuristic optimization
algorithms, 89-103.

\bibitem{key-6}Abualigah, L., Diabat, A., Mirjalili, S., Abd Elaziz,
M., \& Gandomi, A. H. (2021). The arithmetic optimization algorithm.
Computer methods in applied mechanics and engineering, 376, 113609.

\bibitem{key-7}Kaveh, A., \& Hamedani, K. B. (2022, January). Improved
arithmetic optimization algorithm and its application to discrete
structural optimization. In Structures (Vol. 35, pp. 748-764). Elsevier.

\bibitem{key-20}Zhang, J., Zhang, G., Huang, Y., \& Kong, M. (2022).
A novel enhanced arithmetic optimization algorithm for global optimization.
IEEE Access, 10, 75040-75062.

\bibitem{key-8}Salawudeen, A. T., Mu’azu, M. B., Yusuf, A., \& Adedokun,
A. E. (2021). A Novel Smell Agent Optimization (SAO): An extensive
CEC study and engineering application. Knowledge-Based Systems, 232,
107486.

\bibitem{key-9}Salawudeen, A. T., Mu’azu, M. B., Sha’aban, Y. A.,
\& Adedokun, E. A. (2018, September). On the development of a novel
smell agent optimization (SAO) for optimization problems. In 2nd International
Conference on Information and Communication Technology and its Applications
(ICTA 2018), Minna.

\bibitem{key-21}Salawudeen, A. T., Mu’azu, M. B., Yusuf, A., \& Adedokun,
E. A. (2018). From smell phenomenon to smell agent optimization (SAO):
a feasibility study. Proceedings of ICGET.

\bibitem{key-33}Chakraborty, A., \& Kar, A. K. (2017). Swarm intelligence:
A review of algorithms. Nature-inspired computing and optimization:
Theory and applications, 475-494.

\bibitem{key-34}Brezočnik, L., Fister Jr, I., \& Podgorelec, V. (2018).
Swarm intelligence algorithms for feature selection: a review. Applied
Sciences, 8(9), 1521.

\bibitem{key-25}Mohammed, S., Sha’aban, Y. A., Umoh, I. J., Salawudeen,
A. T., \& Ibn Shamsah, S. M. (2023). A hybrid smell agent symbiosis
organism search algorithm for optimal control of microgrid operations.
Plos one, 18(6), e0286695.

\bibitem[(2022)]{Charilogis}Charilogis, V., Tsoulos, I.G.,Tzallas,
A.,Karvounis, E. (2022). Modifications for the Differential Evolution
Algorithm. Symmetry ,2022,14,447. Doi: https://doi.org/10.3390/sym14030447

\bibitem{key-26}Meadows, O. A., Mu’Azu, M. B., \& Salawudeen, A.
T. (2022, April). A smell agent optimization approach to capacitated
vehicle routing problem for solid waste collection. In 2022 IEEE Nigeria
4th International Conference on Disruptive Technologies for Sustainable
Development (NIGERCON) (pp. 1-5). IEEE.

\bibitem{key-27}Kotte, S., Injeti, S. K., Thunuguntla, V. K., Kumar,
P. P., Nuvvula, R. S., Dhanamjayulu, C., ... \& Khan, B. (2024). Energy
curve based enhanced smell agent optimizer for optimal multilevel
threshold selection of thermographic breast image segmentation. Scientific
Reports, 14(1), 21833.

\bibitem{key-28}Arumugam, M., Thiyagarajan, A., Adhi, L., \& Alagar,
S. (2024). Crossover smell agent optimized multilayer perceptron for
precise brain tumor classification on MRI images. Expert Systems with
Applications, 238, 121453.

\bibitem{key-35}Charilogis, V., \& Tsoulos, I. G. (2022). Toward
an ideal particle swarm optimizer for multidimensional functions.
Information, 13(5), 217.

\bibitem{key-22}Powell, M. J. D. (1989). A tolerant algorithm for
linearly constrained optimization calculations. Mathematical Programming,
45, 547-566.

\bibitem[(2023)]{Eltamaly}Eltamaly, AM, Rabie, AH. A Novel Musical
Chairs Optimization Algorithm. Arab J Sci Eng 48 , 10371--10403 (2023).
https://doi.org/10.1007/s13369-023-07610-5

\bibitem[(2013)]{Geetha}Geetha, S., Poonthalir, G. \& Vanathi, P.T.
(2013) .Nested particle swarm optimisation for multi-depot vehicle
routing problem. International Journal of Operational Research 16(3):329
- 348 http://dx.doi.org/10.1504/IJOR.2013.052336

\bibitem{key-10}Ali, M. M., \& Kaelo, P. (2008). Improved particle
swarm algorithms for global optimization. Applied mathematics and
computation, 196(2), 578-593.

\bibitem{testfunc2}Koyuncu, H., \& Ceylan, R. (2019). A PSO based
approach: Scout particle swarm algorithm for continuous global optimization
problems. Journal of Computational Design and Engineering, 6(2), 129-142.

\bibitem{testfunc2-1}Siarry, P., Berthiau, G., Durdin, F., \& Haussy,
J. (1997). Enhanced simulated annealing for globally minimizing functions
of many-continuous variables. ACM Transactions on Mathematical Software
(TOMS), 23(2), 209-228.

\bibitem{testfunc4}LaTorre, A., Molina, D., Osaba, E., Poyatos, J.,
Del Ser, J., \& Herrera, F. (2021). A prescription of methodological
guidelines for comparing bio-inspired optimization algorithms. Swarm
and Evolutionary Computation, 67, 100973.

\bibitem[(2023)]{pde}Charilogis, V.; Tsoulos, I.G. A Parallel Implementation
of the Differential Evolution Method. Analytics 2023, 2, 17--30. 

\bibitem{Gaviano} M. Gaviano, D.E. Ksasov, D. Lera, Y.D. Sergeyev,
Software for generation of classes of test functions with known local
and global minima for global optimization, ACM Trans. Math. Softw.
\textbf{29}, pp. 469-480, 2003.

\bibitem{Lennard} J.E. Lennard-Jones, On the Determination of Molecular
Fields, Proc. R. Soc. Lond. A \textbf{ 106}, pp. 463--477, 1924.

\bibitem{Zabinsky} Z.B. Zabinsky, D.L. Graesser, M.E. Tuttle, G.I.
Kim, Global optimization of composite laminates using improving hit
and run, In: Recent advances in global optimization, pp. 343-368,
1992.

\end{thebibliography}

\end{adjustwidth}{}
\end{document}
