%% LyX 2.4.3 created this file.  For more info, see https://www.lyx.org/.
%% Do not edit unless you really know what you are doing.
\documentclass[journal,article,submit,pdftex,moreauthors]{Definitions/mdpi}
\usepackage{textcomp}
\usepackage[utf8]{inputenc}
\usepackage{float}
\usepackage{url}
\usepackage{bm}
\usepackage{varwidth}
\usepackage{amsmath}
\usepackage{graphicx}

\makeatletter

%%%%%%%%%%%%%%%%%%%%%%%%%%%%%% LyX specific LaTeX commands.

\Title{A novel method that is based on Differential Evolution suitable for
large scale optimization problems}

\Author{Glykeria Kyrou$^{1,*}$, Vasileios Charilogis$^{2}$ and Ioannis G.
Tsoulos$^{3}$}


\address{$^{1}$\quad{}Department of Informatics and Telecommunications,
University of Ioannina, 47150 Kostaki Artas, Greece; g.kyrou@uoi.gr\\
$^{2}$\quad{}Department of Informatics and Telecommunications, University
of Ioannina, Greece; v.charilog@uoi.gr\\
$^{3}\quad$Department of Informatics and Telecommunications, University
of Ioannina, 47150 Kostaki Artas, Greece;itsoulos@uoi.gr}


\corres{Correspondence: g.kyrou@uoi.gr}


\abstract{Global optimization represents a fundamental challenge in computer
science and engineering, as it aims to identify high-quality solutions
to problems spanning from moderate to extremely high dimensionality.
The Differential Evolution algorithm (DE) is a population-based algorithm
like genetic algorithms and uses similar operators such as: crossover,
mutation and selection. The proposed method introduces a set of methodological
enhancements designed to increase both the robustness and the computational
efficiency of the classical The DE framework. Specifically, an adaptive
termination criterion is incorporated, enabling early stopping based
on statistical measures of convergence and population stagnation.
Furthermore, a population sampling strategy based on k-means clustering
is employed to enhance exploration and improve the redistribution
of individuals in high-dimensional search spaces. This mechanism enables
structured population renewal and effectively mitigates premature
convergence. The enhanced algorithm was evaluated on standard large-scale
numerical optimization benchmarks and compared with established global
optimization methods. The experimental results indicate substantial
improvements in convergence speed, scalability, and solution stability.}


\keyword{Optimization; Differential Evolution algorithm; Evolutionary techniques;
Stochastic methods; Large-scale problems}

\newcommand*{\LyXFourPerEmSpace}{\hskip0.25em\relax}
\newcommand*\LyXHairSpace{\hspace{1pt}}
\newcommand*\LyXZeroWidthSpace{\hspace{0pt}}
%% Because html converters don't know tabularnewline
\providecommand{\tabularnewline}{\\}
%% Variable width box for table cells
\newenvironment{cellvarwidth}[1][t]
    {\begin{varwidth}[#1]{\linewidth}}
    {\@finalstrut\@arstrutbox\end{varwidth}}
\floatstyle{ruled}
\newfloat{algorithm}{tbp}{loa}
\providecommand{\algorithmname}{Algorithm}
\floatname{algorithm}{\protect\algorithmname}

%%%%%%%%%%%%%%%%%%%%%%%%%%%%%% User specified LaTeX commands.
%  LaTeX support: latex@mdpi.com 
%  For support, please attach all files needed for compiling as well as the log file, and specify your operating system, LaTeX version, and LaTeX editor.

%=================================================================
%\documentclass[preprints,article,submit,pdftex,moreauthors]{Definitions/mdpi} 
% For posting an early version of this manuscript as a preprint, you may use "preprints" as the journal. Changing "submit" to "accept" before posting will remove line numbers.

% Below journals will use APA reference format:
% admsci, aichem, behavsci, businesses, econometrics, economies, education, ejihpe, famsci, games, humans, ijcs, ijfs, journalmedia, jrfm, languages, psycholint, publications, tourismhosp, youth

% Below journals will use Chicago reference format:
% arts, genealogy, histories, humanities, jintelligence, laws, literature, religions, risks, socsci

%--------------------
% Class Options:
%--------------------
%----------
% journal
%----------
% Choose between the following MDPI journals:
% accountaudit, acoustics, actuators, addictions, adhesives, admsci, adolescents, aerobiology, aerospace, agriculture, agriengineering, agrochemicals, agronomy, ai, air, algorithms, allergies, alloys, amh, analytica, analytics, anatomia, anesthres, animals, antibiotics, antibodies, antioxidants, applbiosci, appliedchem, appliedmath, appliedphys, applmech, applmicrobiol, applnano, applsci, aquacj, architecture, arm, arthropoda, arts, asc, asi, astronomy, atmosphere, atoms, audiolres, automation, axioms, bacteria, batteries, bdcc, behavsci, beverages, biochem, bioengineering, biologics, biology, biomass, biomechanics, biomed, biomedicines, biomedinformatics, biomimetics, biomolecules, biophysica, biosensors, biosphere, biotech, birds, blockchains, bloods, blsf, brainsci, breath, buildings, businesses, cancers, carbon, cardiogenetics, catalysts, cells, ceramics, challenges, chemengineering, chemistry, chemosensors, chemproc, children, chips, cimb, civileng, cleantechnol, climate, clinbioenerg, clinpract, clockssleep, cmd, cmtr, coasts, coatings, colloids, colorants, commodities, complications, compounds, computation, computers, condensedmatter, conservation, constrmater, cosmetics, covid, crops, cryo, cryptography, crystals, csmf, ctn, curroncol, cyber, dairy, data, ddc, dentistry, dermato, dermatopathology, designs, devices, diabetology, diagnostics, dietetics, digital, disabilities, diseases, diversity, dna, drones, dynamics, earth, ebj, ecm, ecologies, econometrics, economies, education, eesp, ejihpe, electricity, electrochem, electronicmat, electronics, encyclopedia, endocrines, energies, eng, engproc, ent, entomology, entropy, environments, epidemiologia, epigenomes, esa, est, famsci, fermentation, fibers, fintech, fire, fishes, fluids, foods, forecasting, forensicsci, forests, fossstud, foundations, fractalfract, fuels, future, futureinternet, futureparasites, futurepharmacol, futurephys, futuretransp, galaxies, games, gases, gastroent, gastrointestdisord, gastronomy, gels, genealogy, genes, geographies, geohazards, geomatics, geometry, geosciences, geotechnics, geriatrics, glacies, grasses, greenhealth, gucdd, hardware, hazardousmatters, healthcare, hearts, hemato, hematolrep, heritage, higheredu, highthroughput, histories, horticulturae, hospitals, humanities, humans, hydrobiology, hydrogen, hydrology, hygiene, idr, iic, ijerph, ijfs, ijgi, ijmd, ijms, ijns, ijpb, ijt, ijtm, ijtpp, ime, immuno, informatics, information, infrastructures, inorganics, insects, instruments, inventions, iot, j, jal, jcdd, jcm, jcp, jcs, jcto, jdad, jdb, jeta, jfb, jfmk, jimaging, jintelligence, jlpea, jmahp, jmmp, jmms, jmp, jmse, jne, jnt, jof, joitmc, joma, jop, jor, journalmedia, jox, jpbi, jpm, jrfm, jsan, jtaer, jvd, jzbg, kidney, kidneydial, kinasesphosphatases, knowledge, labmed, laboratories, land, languages, laws, life, lights, limnolrev, lipidology, liquids, literature, livers, logics, logistics, lubricants, lymphatics, machines, macromol, magnetism, magnetochemistry, make, marinedrugs, materials, materproc, mathematics, mca, measurements, medicina, medicines, medsci, membranes, merits, metabolites, metals, meteorology, methane, metrics, metrology, micro, microarrays, microbiolres, microelectronics, micromachines, microorganisms, microplastics, microwave, minerals, mining, mmphys, modelling, molbank, molecules, mps, msf, mti, multimedia, muscles, nanoenergyadv, nanomanufacturing, nanomaterials, ncrna, ndt, network, neuroglia, neurolint, neurosci, nitrogen, notspecified, nursrep, nutraceuticals, nutrients, obesities, oceans, ohbm, onco, oncopathology, optics, oral, organics, organoids, osteology, oxygen, parasites, parasitologia, particles, pathogens, pathophysiology, pediatrrep, pets, pharmaceuticals, pharmaceutics, pharmacoepidemiology, pharmacy, philosophies, photochem, photonics, phycology, physchem, physics, physiologia, plants, plasma, platforms, pollutants, polymers, polysaccharides, populations, poultry, powders, preprints, proceedings, processes, prosthesis, proteomes, psf, psych, psychiatryint, psychoactives, psycholint, publications, purification, quantumrep, quaternary, qubs, radiation, reactions, realestate, receptors, recycling, regeneration, religions, remotesensing, reports, reprodmed, resources, rheumato, risks, robotics, rsee, ruminants, safety, sci, scipharm, sclerosis, seeds, sensors, separations, sexes, signals, sinusitis, siuj, skins, smartcities, sna, societies, socsci, software, soilsystems, solar, solids, spectroscj, sports, standards, stats, std, stresses, surfaces, surgeries, suschem, sustainability, symmetry, synbio, systems, tae, targets, taxonomy, technologies, telecom, test, textiles, thalassrep, therapeutics, thermo, timespace, tomography, tourismhosp, toxics, toxins, transplantology, transportation, traumacare, traumas, tropicalmed, universe, urbansci, uro, vaccines, vehicles, venereology, vetsci, vibration, virtualworlds, viruses, vision, waste, water, wem, wevj, wild, wind, women, world, youth, zoonoticdis

%---------
% article
%---------
% The default type of manuscript is "article", but can be replaced by: 
% abstract, addendum, article, book, bookreview, briefreport, casereport, comment, commentary, communication, conferenceproceedings, correction, conferencereport, entry, expressionofconcern, extendedabstract, datadescriptor, editorial, essay, erratum, hypothesis, interestingimage, obituary, opinion, projectreport, reply, retraction, review, perspective, protocol, shortnote, studyprotocol, systematicreview, supfile, technicalnote, viewpoint, guidelines, registeredreport, tutorial
% supfile = supplementary materials

%----------
% submit
%----------
% The class option "submit" will be changed to "accept" by the Editorial Office when the paper is accepted. This will only make changes to the frontpage (e.g., the logo of the journal will get visible), the headings, and the copyright information. Also, line numbering will be removed. Journal info and pagination for accepted papers will also be assigned by the Editorial Office.

%------------------
% moreauthors
%------------------
% If there is only one author the class option oneauthor should be used. Otherwise use the class option moreauthors.

%---------
% pdftex
%---------
% The option pdftex is for use with pdfLaTeX. If eps figures are used, remove the option pdftex and use LaTeX and dvi2pdf.

%=================================================================
% MDPI internal commands - do not modify
\firstpage{1} 
\setcounter{page}{\@firstpage}
\pubvolume{1}
\issuenum{1}
\articlenumber{0}
\pubyear{2025}
\copyrightyear{2025}
%\externaleditor{Firstname Lastname} % More than 1 editor, please add `` and '' before the last editor name
\datereceived{}
\daterevised{ } % Comment out if no revised date
\dateaccepted{}
\datepublished{}
%\datecorrected{} % For corrected papers include a "Corrected: XXX" date in the original paper.
%\dateretracted{} % For retracted papers include a "RETRACTED: XXX" date in the original paper.
\hreflink{https://doi.org/} % If needed use \linebreak
%\doinum{}
%\pdfoutput=1 % Uncommented for upload to arXiv.org
%\CorrStatement{yes}  % For updates
%\longauthorlist{yes} % For many authors that exceed the left citation part

%=================================================================
% Add packages and commands here. The following packages are loaded in our class file: fontenc, inputenc, calc, indentfirst, fancyhdr, graphicx, epstopdf, lastpage, ifthen, lineno, float, amsmath, setspace, enumitem, mathpazo, booktabs, titlesec, etoolbox, tabto, xcolor, soul, multirow, microtype, tikz, totcount, changepage, attrib, upgreek, cleveref, amsthm, hyphenat, natbib, hyperref, footmisc, url, geometry, newfloat, caption

%=================================================================
%% Please use the following mathematics environments: Theorem, Lemma, Corollary, Proposition, Characterization, Property, Problem, Example, ExamplesandDefinitions, Hypothesis, Remark, Definition, Notation, Assumption
%% For proofs, please use the proof environment (the amsthm package is loaded by the MDPI class).

%=================================================================
% The fields PACS, MSC, and JEL may be left empty or commented out if not applicable
%\PACS{J0101}
%\MSC{}
%\JEL{}

%%%%%%%%%%%%%%%%%%%%%%%%%%%%%%%%%%%%%%%%%%
% Only for the journal Diversity
%\LSID{\url{http://}}

%%%%%%%%%%%%%%%%%%%%%%%%%%%%%%%%%%%%%%%%%%
% Only for the journal Applied Sciences:
%\featuredapplication{Authors are encouraged to provide a concise description of the specific application or a potential application of the work. This section is not mandatory.}
%%%%%%%%%%%%%%%%%%%%%%%%%%%%%%%%%%%%%%%%%%

%%%%%%%%%%%%%%%%%%%%%%%%%%%%%%%%%%%%%%%%%%
% Only for the journal Data:
%\dataset{DOI number or link to the deposited data set in cases where the data set is published or set to be published separately. If the data set is submitted and will be published as a supplement to this paper in the journal Data, this field will be filled by the editors of the journal. In this case, please make sure to submit the data set as a supplement when entering your manuscript into our manuscript editorial system.}

%\datasetlicense{license under which the data set is made available (CC0, CC-BY, CC-BY-SA, CC-BY-NC, etc.)}

%%%%%%%%%%%%%%%%%%%%%%%%%%%%%%%%%%%%%%%%%%
% Only for the journal BioTech, Fishes, Neuroimaging and Toxins
%\keycontribution{The breakthroughs or highlights of the manuscript. Authors can write one or two sentences to describe the most important part of the paper.}

%%%%%%%%%%%%%%%%%%%%%%%%%%%%%%%%%%%%%%%%%%
% Only for the journal Encyclopedia
%\encyclopediadef{Instead of the abstract}
%\entrylink{The Link to this entry published on the encyclopedia platform.}
%%%%%%%%%%%%%%%%%%%%%%%%%%%%%%%%%%%%%%%%%%

%%%%%%%%%%%%%%%%%%%%%%%%%%%%%%%%%%%%%%%%%%
% Different journals have different requirements. Please check the specific journal guidelines in the "Instructions for Authors" on the journal's official website.
 
%\addhighlights{yes}
%\renewcommand{\addhighlights}{%
%
%\noindent The goal is to increase the discoverability and readability of the article via search engines and other scholars. Highlights should not be a copy of the abstract, but a simple text allowing the reader to quickly and simplified find out what the article is about and what can be cited from it. Each of these parts should be devoted up to 2 bullet points.\vspace{3pt}\\
%\textbf{What are the main findings?}
% \begin{itemize}[labelsep=2.5mm,topsep=-3pt]
% \item First bullet.
% \item Second bullet.
% \end{itemize}\vspace{3pt}
%\textbf{What is the implication of the main finding?}
% \begin{itemize}[labelsep=2.5mm,topsep=-3pt]
% \item First bullet.
% \item Second bullet.
% \end{itemize}
%}
%%%%%%%%%%%%%%%%%%%%%%%%%%%%%%%%%%%%%%%%%%

\makeatother

\begin{document}
\maketitle

\section{Introduction}

The\textbf{ }basic goal of global optimization is to find the global
minimum of a continuous multidimensional function and is defined as:
\begin{equation}
x^{*}=\mbox{arg}\min_{x\in S}f(x)\label{eq:eq1}
\end{equation}
with $S$: 
\[
S=\left[a_{1},b_{1}\right]\times\left[a_{2},b_{2}\right]\times\ldots\left[a_{n},b_{n}\right]
\]

with, $a_{i}$ and $b_{i}$ representing lower and upper bounds for
each variable $x_{i}$.

In recent years many researchers have published important reviews
on global optimization. Such methods find application in a wide range
of scientific fields, such as mathematics \citep{go_math1,go_math3},
physics \citep{go_physics2,go_physics3}, chemistry \citep{go_chem1,go_chem2},
biology \citep{go_bio1,go_bio2}, medicine \citep{go_med2,medicine},
agriculture \citep{go_agri1,go_agri2} and economics \citep{go_econ1,go_econ2}.\textbf{
}A particular challenge is the large-scale global optimization (LSGO)
problem, where the complexity increases significantly with increasing
problem dimensions. Finding efficient and computationally feasible
solutions has become particularly difficult, which has led the research
community to focus on the development of innovative algorithms. LSGO
problems are encountered in a wide range of applications, while their
importance is also reflected in the organization of the first global
large-scale optimization competition within the framework of the CEC
in 2008. Other competitions followed in 2010 \citep{key-23}, 2013
\citep{key-24} and 2015 \citep{key-25}, attracting the intense interest
of the academic community.

To address these challenges, various heuristic and meta-heuristic
approaches have been developed. Evolutionary Algorithms (EA) \citep{key-6,key-7}
are one of the most effective categories, as they mimic natural selection
and genetic evolution to search for best solutions. Due to their adaptability
and robustness, EAs can solve difficult optimization problems. Some
of the most well-known EAs are DE \citep{diffe1,diffe2}, Genetic
Algorithms \citep{genetic2,genetic3}, Evolutionary Strategies \citep{key-8,key-11},
Evolutionary Programming\citep{key-12,key-13}, Multimodal Optimization
Algorithms \citep{key-14,key-15}. Also, methods inspired by Swarm
Intelligence \citep{key-27,key-28} such as: Particle Swarm Optimization
(PSO) \citep{pso_major,pso1}, Ant Colony Optimization (ACO) \citep{aco1,aco2},
Artificial Bee Colony Optimization (ABC) \citep{key-18,key-17}, Firefly
Algorithm (FA) \citep{key-19,key-20} , Bat Algorithm \citep{key-50,key-51}
are strong alternatives.

DE is one of the most widely used optimization techniques, as it offers
high robustness, simplicity and fast convergence. DE is a highly efficient
evolutionary algorithm that has gained significant recognition since
the late 1990s. DE, originally introduced in 1995 by Storn and Price
\citep{key-1,key-2-1}, has proven to be a versatile optimization
tool that can be applied in various scientific and engineering fields.
It is particularly effective for symmetric optimization problems,
as well as for dealing with discontinuous, noisy, and dynamic challenges.
In physics, it has been used in energy-related problems, including
wind power optimization\citep{key-1-1}. In chemistry, it has contributed
to advances in atmospheric chemistry and the development of high-performance
chemical reactors \citep{key-2}\citep{key-3}. DE has also had significant
implications in health-related areas, such as breast cancer research
and medical diagnostics\citep{key-4}. Despite its effectiveness,
classical DE may face additional challenges when applied to high-dimensional
and large-scale optimization problems. In such settings, the adaptation
of control parameters becomes more demanding, while exploitation efficiency
during later stages of the search process may be reduced. Moreover,
as problem dimensionality increases, maintaining fast convergence
and high solution accuracy can become more difficult. These observations
have motivated extensive research efforts toward the development of
enhanced DE variants, aiming to improve adaptability, scalability,
and the balance between exploration and exploitation in large-scale
optimization problems.

Recent studies have proposed various strategies to address large-scale
optimization challenges, including cooperative coevolution \citep{large_co},
Particle Swarm Optimization \citep{large_pso}, a DE approach \citep{large_memetic},
a self-adaptive Fast Fireworks Algorithm\citep{key-40},\textbf{ }swarm-based
methods with learning mechanisms\citep{key-60,key-61} and advanced
decomposition techniques such as dual Differential Grouping \citep{key-43}.
These approaches highlight the ongoing research interest in designing
efficient and scalable optimization frameworks for large-scale problems.

In this work, a unified and modular DE framework is proposed to improve
efficiency, robustness, and convergence behavior in large-scale optimization
problems. Instead of relying on a single evolutionary strategy or
fixed parameter settings, the proposed approach integrates multiple
enhancement mechanisms within a single DE framework. These mechanisms
are designed to be independently configurable, allowing the algorithm
to adapt more effectively to diverse problem characteristics while
preserving the fundamental structure of classical DE.

The main contributions of this paper are summarized as follows:
\begin{itemize}
\item A unified and modular DE framework is introduced, integrating multiple
control mechanisms within a single optimization scheme, including
mutation weighting, parent selection, local refinement activation,
and termination criteria.
\item A k-means--based population sampling strategy is incorporated to
preserve population structure and improve sampling efficiency in high-dimensional
search spaces.
\item Alternative mechanisms for computing the differential weight parameter
are proposed, incorporating Number-based, Random, and Migrant strategies
to enhance adaptability in large-scale optimization problems.
\item An optional tournament-based parent selection strategy is employed
to improve selection pressure while maintaining population diversity
and robustness.
\item A periodic local optimization refinement using deterministic local
optimizers, such as BFGS, is integrated to enhance solution accuracy
and accelerate convergence without compromising global exploration.
\item A population-based termination criterion is introduced to enable early
stopping when convergence stagnation is detected, significantly reducing
unnecessary objective function evaluations.
\item The proposed framework is specifically designed for large-scale global
optimization problems and aims to achieve a more effective balance
between exploration and exploitation compared to classical DE variants.
\end{itemize}
Although the individual enhancement mechanisms employed in the proposed
framework, including adaptive termination strategies, clustering-based
population initialization, adaptive differential weight schemes, and
local optimization techniques, have been previously studied in the
literature, their combined integration within a single DE framework
is not straightforward. Each mechanism operates at a distinct stage
of the evolutionary process and affects different aspects of the search
dynamics, such as diversity preservation, convergence behavior, and
computational efficiency.

The contribution of the present work lies in the systematic and modular
coordination of these mechanisms within a unified optimization framework.
By enabling controlled interaction between global exploration, adaptive
parameter control, local exploitation, and termination mechanisms,
the proposed approach facilitates synergistic effects that cannot
be readily achieved through the isolated application of individual
enhancements. This structured integration provides a principled framework
for improving robustness and scalability in large-scale global optimization
problems.

Importantly, the novelty of the proposed approach lies not in the
introduction of new standalone operators, but in the principled way
these mechanisms are coordinated within a single framework. Each component
is explicitly designed to operate in synergy with the others, rather
than being applied independently. In particular, structured sampling
and adaptive differential weighting jointly influence population diversity
and step-size control, while tournament selection ensures that informative
population feedback is preserved for adaptive mechanisms to exploit.
This coordinated interaction provides a theoretical justification
for the combined framework and explains why its behavior cannot be
reduced to a simple aggregation of existing techniques.

The remains of this paper are divided as follows: in section \ref{sec:DE}
the original DE algorithm, the proposed method as well as the flowchart
with detailed description are presented, in section \ref{sec:Results}
of the test functions used in the experiments as well as the related
experiments are presented. In the \ref{sec:Discussion} section, there
is a brief discussion of the results obtained from the experiments.
In section \ref{sec:Conclusions} some conclusions and directions
for future improvements are discussed.

\section{DE Algorithm\label{sec:DE}}

\subsubsection*{2.1 The original DE method}

DE is a population-based evolutionary algorithm that has been widely
used for continuous optimization problems. The method maintains a
population of candidate solutions, which are iteratively evolved through
the application of mutation, crossover, and selection operators. At
each iteration, new candidate solutions are generated by combining
information from multiple population members, while selection is performed
based on objective function comparisons. The DE procedure begins by
defining the population size $NP$. In order to ensure the feasibility
of the classical DE mutation operator, which requires three distinct
population members in addition to the target vector, a minimum population
size of $NP\geq4$ is required. In practice, the population size is
often related to the dimensionality of the optimization problem. A
commonly adopted guideline in the DE literature is to set $NP=10n$,
where $n$ denotes the problem dimension, as this choice has been
shown to provide robust performance across a wide range of problems
without extensive parameter tuning. We note that alternative formulations
relating the population size to the problem dimension have also been
proposed in the literature; however, such choices are problem-dependent
and do not affect the general applicability of the DE framework. The
initial population is generated randomly within the search space and
evaluated using the objective function. During each iteration, for
every target vector $\mathbf{x}_{i}$, three distinct population members
are randomly selected to construct a mutant vector through a differential
mutation operation. This mutant vector is then combined with the target
vector using a binomial crossover mechanism, producing a trial vector.
If the trial vector achieves an objective function value that is not
worse than that of the target vector, it replaces the target vector
in the population. The evolutionary process continues until a termination
criterion is satisfied, such as reaching a maximum number of iterations
or meeting a convergence condition. The algorithm returns the best
solution found during the search process. Regarding parameter settings,
the crossover probability is set to $CR=0.9$ and the differential
weight to $F=0.8$, following values commonly adopted in the DE literature\citep{key-2-1,key-22}.
These parameter values have been empirically shown to provide stable
performance across a broad range of optimization problems without
requiring problem-specific tuning. In this study, all parameters of
the original DE algorithm are kept fixed throughout the experimental
evaluation in order to ensure a fair and unbiased comparison with
the proposed method. For clarity, the full steps of the original DE
algorithm are summarized in Algorithm 1.

\begin{algorithm}
{\scriptsize\textbf{INPUT}}{\scriptsize\par}

{\scriptsize - $f$: objective function}{\scriptsize\par}

{\scriptsize - $NP$: population size}{\scriptsize\par}

{\scriptsize - $CR$: Crossover rate}{\scriptsize\par}

{\scriptsize - $F$: Differential weight}{\scriptsize\par}

{\scriptsize - $n$: Problem dimension}{\scriptsize\par}

{\scriptsize\textbf{OUTPUT}}{\scriptsize\par}

{\scriptsize -$x_{best}$}{\scriptsize\par}

{\scriptsize\textbf{INITIALIZATION}}{\scriptsize\par}

{\scriptsize -Generate an initial population of $NP$ candidate solutions$x_{i},\LyXFourPerEmSpace\LyXHairSpace i=1,\ldots,NP$,
uniformly at random within the search bounds.}{\scriptsize\par}

{\scriptsize -Evaluate the objective function $f(x_{i})$ for all individuals.}{\scriptsize\par}

{\scriptsize -Set $x_{best}$\LyXZeroWidthSpace{} as the individual
with the best objective value.}{\scriptsize\par}

{\scriptsize\textbf{main pseudocode}}{\scriptsize\par}

{\scriptsize 01}{\scriptsize\textbf{ while}}{\scriptsize{} }{\scriptsize\textbf{stopping
criterion is not met do }}{\scriptsize\par}

{\scriptsize 02 \hspace{0.5cm}}{\scriptsize\textbf{for}}{\scriptsize{}
each individual $i,$ $i\in$ \{1..$NP$\} }{\scriptsize\textbf{do}}{\scriptsize\par}

{\scriptsize 03 \hspace{0.5cm}\hspace{0.5cm}Select randomly three
agents$\,a,b,c$$\,\in$ \{1..$NP$\}}{\scriptsize\par}

{\scriptsize 04\hspace{0.5cm}\hspace{0.5cm}Generate mutant vector$\,\,u=a_{j}+F\times(b_{j}-c_{j})$}{\scriptsize\par}

{\scriptsize 05\hspace{0.5cm}\hspace{0.5cm}Select a random index
$R$$\,\in$ \{1..$n$\}}{\scriptsize\par}

{\scriptsize 06\hspace{0.5cm}\hspace{0.5cm}Compute the trial vector
$y=[y_{1},y_{2},....y_{n}]$as follows.}{\scriptsize\par}

{\scriptsize 07 \hspace{0.5cm}\hspace{0.5cm}f}{\scriptsize\textbf{or
}}{\scriptsize each dimension$j=1$ to n }{\scriptsize\textbf{do}}{\scriptsize\par}

{\scriptsize 08 \hspace{0.5cm}\hspace{0.5cm}\hspace{0.5cm} Generate
a random number $r_{i}\in[0,1]$}{\scriptsize\par}

{\scriptsize 09 \hspace{0.5cm}\hspace{0.5cm}\hspace{0.5cm}}{\scriptsize\textbf{if}}{\scriptsize{}
$r_{j}<CR\,\,or\,\,j=R$}{\scriptsize\textbf{ then }}{\scriptsize\par}

{\scriptsize 10 \hspace{0.5cm}\hspace{0.5cm} \hspace{0.5cm}\hspace{0.5cm}Set
$y_{j}=u_{j}.$}{\scriptsize\par}

{\scriptsize 11 \hspace{0.5cm}\hspace{0.5cm}\hspace{0.5cm}}{\scriptsize\textbf{else }}{\scriptsize\par}

{\scriptsize 12\hspace{0.5cm}\hspace{0.5cm} \hspace{0.5cm}\hspace{0.5cm}Set
$y_{j}=x_{ij}.$}{\scriptsize\par}

{\scriptsize 13 \hspace{0.5cm}\hspace{0.5cm}\hspace{0.5cm}}{\scriptsize\textbf{endif}}{\scriptsize\par}

{\scriptsize 14 \hspace{0.5cm}\hspace{0.5cm}}{\scriptsize\textbf{end}}{\scriptsize f}{\scriptsize\textbf{or}}{\scriptsize\par}

{\scriptsize 15 \hspace{0.5cm}\hspace{0.5cm}}{\scriptsize\textbf{if
}}{\scriptsize$f(y)\leq f(x_{i})$ }{\scriptsize\textbf{then }}{\scriptsize\par}

{\scriptsize 16 \hspace{0.5cm}\hspace{0.5cm}\hspace{0.5cm}Replace
$x_{i}\,with\,y$}{\scriptsize\par}

{\scriptsize 17 \hspace{0.5cm}\hspace{0.5cm}}{\scriptsize\textbf{endif}}{\scriptsize\par}

{\scriptsize 18 \hspace{0.5cm}}{\scriptsize\textbf{endfor}}{\scriptsize\par}

{\scriptsize 19 }{\scriptsize\textbf{endwhile}}{\scriptsize\par}

{\scriptsize 20 }{\scriptsize\textbf{return}}{\scriptsize{} $x_{best}$}{\scriptsize\par}

\caption{Original DE Algorithm}

\end{algorithm}

\begin{itemize}
\item \medskip{}
\end{itemize}

\subsubsection*{2.2 The proposed DE method}

The main contribution of the proposed approach is the formulation
of a unified and modular DE framework that systematically integrates
multiple control mechanisms within a single optimization scheme. Unlike
most existing DE variants, which typically modify a single algorithmic
component (e.g., mutation strategy or parameter adaptation), the proposed
method allows the independent configuration and combined use of several
algorithmic mechanisms while preserving the fundamental evolutionary
structure of the original DE algorithm. The proposed method extends
the classical DE framework by introducing a modular design that incorporates
additional control components related to differential weight computation,
parent selection, local refinement, and termination criteria. This
design enables both the independent analysis of each component and
their joint exploitation within a single, coherent optimization process,
facilitating a systematic investigation of robustness and parameter
sensitivity. The main methodological contributions of the proposed
framework can be summarized as follows:
\begin{itemize}
\item Alternative mechanisms for differential weight computation
\item A k-means--based population sampling strategy is incorporated to
preserve population structure and improve sampling efficiency in high-dimensional
search spaces
\item Optional tournament-based parent selection strategies,
\item Periodic local refinement using a deterministic local optimizer
\item A population-based termination criterion
\end{itemize}
The algorithm starts with an initialization phase in which a population
of $NP$ agents is randomly generated and evaluated using the objective
function. Additional control parameters are also initialized at this
stage, including the local search rate $p_{l}$, which determines
the frequency of local refinement, the tournament size $N_{t}$, which
controls selection pressure, the maximum number of generations $N_{g}$,
the termination criteria $N_{I}$, and the iteration counter $k$.
During the evolutionary process, different strategies for computing
the differential weight $F$ can be employed. These include a constant
value, a random mechanism defined as $F=-0.5+2r$, where $r\in[0,1]$,
as well as a migrant-based strategy. For each agent, candidate solutions
are generated through mutation and crossover operations, followed
by a selection step based on objective function comparisons. In addition
to the evolutionary operators, a deterministic local search procedure
based on the BFGS method\citep{powell} may be periodically applied
to refine promising solutions. The optimization process proceeds iteratively
until a termination condition is satisfied, which may be defined either
by a maximum number of generations or by a population-based convergence
criterion. By jointly integrating stochastic variation, adaptive differential
weight mechanisms, and deterministic local refinement within a unified
framework, the proposed method aims to improve search efficiency and
to achieve a more effective balance between exploration and exploitation.
Moreover, the modular structure of the algorithm allows the systematic
evaluation of the effect of individual algorithmic components, as
demonstrated in the sensitivity analysis presented in Section 3. For
clarity, the complete steps of the proposed DE algorithm are summarized
in Algorithm 2.

\begin{algorithm}
{\scriptsize\textbf{INPUT}}{\scriptsize\par}

{\scriptsize - $f$: objective function}{\scriptsize\par}

{\scriptsize - $NP$: population size}{\scriptsize\par}

{\scriptsize - $N_{t}$: tournament size}{\scriptsize\par}

{\scriptsize - $N_{g}$: maximum number of iterations}{\scriptsize\par}

{\scriptsize - $N_{I}$: termination criteria}{\scriptsize\par}

{\scriptsize - $CR$: Crossover rate}{\scriptsize\par}

{\scriptsize - $F$: Differential weight}{\scriptsize\par}

{\scriptsize - $n$: Problem dimension}{\scriptsize\par}

{\scriptsize - $k$: iteration counter}{\scriptsize\par}

{\scriptsize\textbf{OUTPUT}}{\scriptsize\par}

{\scriptsize -$x_{best}$}{\scriptsize\par}

{\scriptsize\textbf{INITIALIZATION}}{\scriptsize\par}

{\scriptsize -Set as $NP$ the population size (number of agents)}{\scriptsize\par}

{\scriptsize -Create randomly NP agents $x_{i},\LyXFourPerEmSpace\LyXHairSpace i=1,\ldots,NP$}{\scriptsize\par}

{\scriptsize -Compute the fitness value$f_{i}$= $f(x_{i})$ for each
agent}{\scriptsize\par}

{\scriptsize -Set as $p_{l}$ the local search rate}{\scriptsize\par}

{\scriptsize -Set as $N_{g}$ maximum number of iterations}{\scriptsize\par}

{\scriptsize -Set as $N_{I}$termination criteria}{\scriptsize\par}

{\scriptsize -Set as $N_{t}$tournament size}{\scriptsize\par}

{\scriptsize -Set $k\leftarrow0$ as the iteration counter}{\scriptsize\par}

{\scriptsize -Set the parameter $CR$, with $CR\leq1$}{\scriptsize\par}

{\scriptsize -Select the differential weight method F: }{\scriptsize\par}

{\scriptsize (a) Number : $F$ is constant value.}{\scriptsize\par}

{\scriptsize (b) Random :$F=-0.5+2r,$ $r\in[0,1]$ poposed by Charilogis
et al.\citep{de_char}.}{\scriptsize\par}

{\scriptsize (c) Migrant : migrant-based michanism\citep{de_migrant}.}{\scriptsize\par}

{\scriptsize\textbf{main pseudocode}}{\scriptsize\par}

{\scriptsize 01}{\scriptsize\textbf{ while}}{\scriptsize{} }{\scriptsize\textbf{stopping
criterion is not met do }}{\scriptsize\par}

{\scriptsize 02 \hspace{0.5cm}}{\scriptsize\textbf{for}}{\scriptsize{}
each individual $i,$ $i\in$ \{1..$NP$\} }{\scriptsize\textbf{do}}{\scriptsize\par}

{\scriptsize 03 \hspace{0.5cm}\hspace{0.5cm}Select the agent $x_{i}$}{\scriptsize\par}

{\scriptsize 04 \hspace{0.5cm}\hspace{0.5cm}Select randomly three
distinct agents $x_{a},x_{b},x_{c}$}{\scriptsize\par}

{\scriptsize 05 \hspace{0.5cm}\hspace{0.5cm}Choose a random integer
$R$$\,\in$ $[1,n]$}{\scriptsize\par}

{\scriptsize 06 \hspace{0.5cm}\hspace{0.5cm}Create a trial point
$x_{t}$}{\scriptsize\par}

{\scriptsize 07 \hspace{0.5cm}\hspace{0.5cm}}{\scriptsize\textbf{for}}{\scriptsize{}
$j\in$ \{1..$n$\} }{\scriptsize\textbf{do}}{\scriptsize\par}

{\scriptsize 08 \hspace{0.5cm}\hspace{0.5cm}\hspace{0.5cm} Select
a random number $r\in[0,1]$}{\scriptsize\par}

{\scriptsize 09 \hspace{0.5cm}\hspace{0.5cm}\hspace{0.5cm}}{\scriptsize\textbf{if}}{\scriptsize{}
$r<CR\,\,or\,\,j=R$}{\scriptsize\textbf{ then }}{\scriptsize\par}

{\scriptsize 10 \hspace{0.5cm}\hspace{0.5cm} \hspace{0.5cm}\hspace{0.5cm}Set
$x_{t,j}=x_{a,j}$$+F\times(x_{b,j}-x_{c,j})$}{\scriptsize\par}

{\scriptsize 11 \hspace{0.5cm}\hspace{0.5cm}\hspace{0.5cm}}{\scriptsize\textbf{else }}{\scriptsize\par}

{\scriptsize 12\hspace{0.5cm}\hspace{0.5cm} \hspace{0.5cm}\hspace{0.5cm}Set
$x_{t,j}=x_{ij}$}{\scriptsize\par}

{\scriptsize 13 \hspace{0.5cm}\hspace{0.5cm}\hspace{0.5cm}}{\scriptsize\textbf{endif}}{\scriptsize\par}

{\scriptsize 14 \hspace{0.5cm}}{\scriptsize\textbf{endfor}}{\scriptsize\par}

{\scriptsize 15 \hspace{0.5cm}Set $y_{t}=f(x_{t})$}{\scriptsize\par}

{\scriptsize 16 \hspace{0.5cm}i}{\scriptsize\textbf{f }}{\scriptsize$(y_{t})\preceq f(x_{i})$
}{\scriptsize\textbf{then }}{\scriptsize\par}

{\scriptsize 17 \hspace{0.5cm}\hspace{0.5cm}Replace $x_{i}\,with\,x_{t}$}{\scriptsize\par}

{\scriptsize 18 \hspace{0.5cm}}{\scriptsize\textbf{endif}}{\scriptsize\par}

{\scriptsize 19 \hspace{0.5cm}Select a random number $r\in[0,1]$}{\scriptsize\par}

{\scriptsize 20 \hspace{0.5cm}}{\scriptsize\textbf{if}}{\scriptsize{}
$r\leq p_{l}$}{\scriptsize\textbf{then }}{\scriptsize\par}

{\scriptsize 21 \hspace{0.5cm}\hspace{0.5cm}Apply local search $x_{i}=LS(x_{i})$\citep{powell}}{\scriptsize\par}

{\scriptsize 22 \hspace{0.5cm}}{\scriptsize\textbf{endif}}{\scriptsize\par}

{\scriptsize 23 \hspace{0.5cm}\hspace{0.5cm}}{\scriptsize\textbf{endfor}}{\scriptsize\par}

{\scriptsize 24 \hspace{0.5cm}\hspace{0.5cm}$Set\,\,k\leftarrow k+1$}{\scriptsize\par}

{\scriptsize\textbf{TERMINATION CHECK }}{\scriptsize\par}

{\scriptsize 25\hspace{0.5cm}}{\scriptsize\textbf{if}}{\scriptsize{}
$k\geq N_{g}$ }{\scriptsize\textbf{then}}{\scriptsize{} terminate}{\scriptsize\par}

{\scriptsize 26\hspace{0.5cm}Compute $\delta^{(k)}$$=$$\left|\sum_{i=1}{\displaystyle ^{NP}}\mid f_{i}^{(k)}\mid-\sum_{i=1}{\displaystyle ^{NP}}\mid f_{i}^{(k-1)}\mid\right|$}{\scriptsize\par}

{\scriptsize 27\hspace{0.5cm}}{\scriptsize\textbf{if}}{\scriptsize{}
$\delta^{(k)}$$\leq\varepsilon$ for $N_{I}$ iterations }{\scriptsize\textbf{then}}{\scriptsize{}
terminate.}{\scriptsize\par}

{\scriptsize 28}{\scriptsize\textbf{ endwhile}}{\scriptsize\par}

{\scriptsize 29}{\scriptsize\textbf{ return}}{\scriptsize{} $x_{best}$}\caption{Proposed Algorithm}
\end{algorithm}

\medskip{}

The steps of the proposed DE algorithm can be described as follows:

\begin{figure}
\centering{}\includegraphics{diagramma}\caption{The steps of the proposed DE algorithm.}
\end{figure}


\section{Experiments\label{sec:Results}}

This section begins with a description of the functions that will
be used in the experiments and then presents in detail the experiments
that were performed, in which the parameters available in the proposed
algorithm were studied, in order to study their reliability and adequacy. 

\subsection{Test Functions}

A variety of benchmark test functions were used in the conducted experiments.
These functions have been widely adopted in previous studies \citep{testfunc1,testfunc2,testfunc2-1,testfunc4}.
In the present work, the test functions are evaluated using dimensionalities
ranging from 25 to 150, where the constant n denotes the dimension
of the objective function. The benchmark test functions used in the
experimental study are summarized in Table 1, including their mathematical
formulation, dimensionality, and global optimum values.
\begin{center}
{\tiny{}
\begin{table}
\centering{}\caption{Benchmark test functions used in experimental study.}
{\tiny{}%
\begin{tabular}{|c|c|c|c|}
\hline 
{\tiny\textbf{NAME}} & {\tiny\textbf{FORMULA}} & {\tiny\textbf{DIM}} & {\tiny$G_{min}$}\tabularnewline
\hline 
\hline 
\begin{cellvarwidth}[t]
\centering
{\tiny ATTRACTIVE}{\tiny\par}

{\tiny SECTOR}
\end{cellvarwidth} & {\tiny$f(x)=\left(\sum_{i=1}^{n}(s_{i}x_{i})^{2}\right)^{0.9}$} & {\tiny 2} & {\tiny 0}\tabularnewline
\hline 
{\tiny BUCHE RASTRIGIN} & {\tiny$f(\mathbf{x})=\sum_{i=1}^{n}\left[z_{i}\cdot\left(1+0.1\cdot\sin(10\pi z_{i})\right)\right]$} & {\tiny n} & {\tiny 0}\tabularnewline
\hline 
{\tiny DIFFERENT POWERS} & {\tiny$f(\mathbf{x})=\sqrt{\sum_{i=1}^{n}|x_{i}|^{2+4\frac{i-1}{n-1}}}$} & {\tiny n} & {\tiny 0}\tabularnewline
\hline 
{\tiny DISCUS} & {\tiny$f(x)=10^{6}x_{1}^{2}+\sum_{i=2}^{n}x_{i}^{2}$} & {\tiny n} & {\tiny 0}\tabularnewline
\hline 
{\tiny ELLIPSOIDAL} & {\tiny$f(x)=\sum_{i=1}^{n}\left(10^{6}\right)^{\frac{i-1}{n-1}}x_{i}^{2}$} & {\tiny n} & {\tiny 0}\tabularnewline
\hline 
{\tiny GALLAGHER101} & {\tiny$f(\mathbf{x})=\max_{i=1}^{101}\left[h_{i}-w_{i}\sqrt{\sum_{j=1}^{n}(x_{j}-c_{ij})^{2}}\right]$$\ min:100+1$} & {\tiny n} & {\tiny 0}\tabularnewline
\hline 
{\tiny GALLAGHER21} & {\tiny$f(\mathbf{x})=\max_{i=1}^{21}\left[h_{i}-w_{i}\sqrt{\sum_{j=1}^{n}(x_{j}-c_{ij})^{2}}\right]$$\ min:10+10+1$} & {\tiny n} & {\tiny 0}\tabularnewline
\hline 
\begin{cellvarwidth}[t]
\centering
{\tiny GRIEWANK}{\tiny\par}

{\tiny ROSENBROCK}
\end{cellvarwidth} & {\tiny$f(\mathbf{x})=\underbrace{\left(\frac{\|\mathbf{x}\|^{2}}{4000}-\prod_{i=1}^{n}\cos\left(\frac{x_{i}}{\sqrt{i}}\right)+1\right)}_{\text{Griewank}}\cdot\underbrace{\left(\frac{1}{10}\sum_{i=1}^{n-1}\left[100(x_{i+1}-x_{i}^{2})^{2}+(1-x_{i})^{2}\right]\right)}_{\text{Rosenbrock}}$} & {\tiny n} & {\tiny 0}\tabularnewline
\hline 
{\tiny GRIEWANK} & {\tiny$f(x)=1+\frac{1}{200}\sum_{i=1}^{n}x_{i}^{2}-\prod_{i=1}^{n}\frac{\cos(x_{i})}{\sqrt{(i)}}$} & {\tiny n} & {\tiny 0}\tabularnewline
\hline 
{\tiny RARSTIGIN} & {\tiny$f(\mathbf{x})=An+\sum_{i=1}^{n}\left[x_{i}^{2}-A\cos(2\pi x_{i})\right]$$\ A=10$} & {\tiny n} & {\tiny 0}\tabularnewline
\hline 
{\tiny ROSENBROCK} & {\tiny$f(x)=\sum_{i=1}^{n-1}\left(100\left(x_{i+1}-x_{i}^{2}\right)^{2}+\left(x_{i}-1\right)^{2}\right),\quad-30\le x_{i}\le30$} & {\tiny n} & {\tiny 0}\tabularnewline
\hline 
{\tiny SHARP RIDGE} & {\tiny$f(\mathbf{x})=x_{1}^{2}+\alpha\sum_{i=2}^{n}x_{i}^{2},\ a>1$} & {\tiny n} & {\tiny 0}\tabularnewline
\hline 
{\tiny SPHERE} & {\tiny$f(\mathbf{x})=\sum_{i=1}^{n}x_{i}^{2}$} & {\tiny n} & {\tiny 0}\tabularnewline
\hline 
{\tiny STEP ELLIPSOIDAL} & {\tiny$f(\mathbf{x})=\sum_{i=1}^{n}\left\lfloor x_{i}+0.5\right\rfloor ^{2}+\alpha\sum_{i=1}^{n}\left(10^{6}\cdot\frac{i-1}{n-1}\right)x_{i}^{2},\ a=1$} & {\tiny n} & {\tiny 0}\tabularnewline
\hline 
{\tiny ZAKHAROV} & {\tiny$f(\mathbf{x})=\sum_{i=1}^{n}x_{i}^{2}+\left(\sum_{i=1}^{n}\frac{i}{2}x_{i}\right)^{2}+\left(\sum_{i=1}^{n}\frac{i}{2}x_{i}\right)^{4}$} & {\tiny n} & {\tiny 0}\tabularnewline
\hline 
\end{tabular}}
\end{table}
}{\tiny\par}
\par\end{center}

\subsection{Experimental results }

A series of experiments were carried out for the previously mentioned
functions and these experiments were executed on an AMD RYZEN 5950X
with 128GB RAM. The operating system of the running machine was Debian
Linux. Each experiment was conducted 30 times, with different random
numbers each time, and the averages were recorded. The software used
in the experiments was coded in ANSI C++ using the freely available
optimization environment of OPTIMUS\citep{OPTIMUS}, which can be
downloaded from\textbf{ }\url{https://github.com/itsoulos/OPTIMUS}.
In addition to the proposed DE framework, a Genetic Algorithm (GA)
is employed as a baseline evolutionary method for comparative evaluation.
The inclusion of Genetic Algorithm provides a well-established reference
approach in global optimization, allowing a clearer assessment of
the performance of the proposed method. All comparative methods were
independently implemented and evaluated within the same experimental
framework and were not directly adopted from the corresponding literature.
All algorithms were implemented in ANSI C++ and executed under identical
hardware and software conditions to ensure a fair and consistent comparison.
All algorithms were run using the same termination criterion, in order
to ensure a fair and reproducible comparison. To guarantee comparability
across all methods, common control parameters were fixed, as summarized
in Table 2. Specifically, the number of agents (population size) was
set to 200 for all algorithms, the maximum number of iterations was
fixed at 200, and the local search rate was uniformly set to 0.05.
These shared settings ensure that performance differences arise from
algorithmic design rather than differences in experimental configuration.
Algorithm-specific parameters were selected according to standard
practices reported in the literature and were kept constant throughout
all experiments. No problem-specific parameter tuning was applied
in order to avoid bias and to maintain methodological consistency.
The parameter settings for both the proposed method and the GA are
summarized in the Table \ref{tab:expSettings}

\begin{table}[H]
\caption{The values of the parameters of the proposed method.\label{tab:expSettings}}

\centering{}%
\begin{tabular}{|c|c|c|}
\hline 
PARAMETER & MEANING & VALUE\tabularnewline
\hline 
\hline 
$NP$ & Number of agents for all methods & 200\tabularnewline
\hline 
$n$ & Maximum number of allowed iterations for all methods & 200\tabularnewline
\hline 
$p_{l}$ & Local search rate, & 0.05\tabularnewline
\hline 
$F$ & Differential weight for classic DE & $F\in[0,1.0]$\tabularnewline
\hline 
$F$ & Differential weight for PROPOSED & 0.8\tabularnewline
\hline 
$CR$ & Crossover probability & 0.9\tabularnewline
\hline 
$N_{I}$ & Number of iterations used in the termination rule & 8\tabularnewline
\hline 
- & Mutation rate for GA & 0.05 (5\%)\tabularnewline
\hline 
- & Selection Rate for GA & 0.05 (5\%)\tabularnewline
\hline 
- & Selection method for GA & Roulette\tabularnewline
\hline 
\end{tabular}
\end{table}

The parameter values selected for the Genetic Algorithm were chosen
based on standard practices in evolutionary computation and preliminary
empirical evaluation. All Genetic Algorithm parameters were kept fixed
throughout the experimental study to avoid problem-specific tuning
and to ensure a fair and consistent comparison with the proposed method.

\subsection{The effect of differential weight mechanism}

Table \ref{tab:Experiments-using-different} presents the impact of
the three differential weight strategies NUMBER(T), RANDOM(T), and
MIGRANT(T) on the performance of the algorithm across a broad set
of benchmark functions, where (T) denotes tournament--based selection.
The results clearly show that MIGRANT(T) is by far the most efficient
method.. The MIGRANT(T) strategy consistently achieves the best outcomes,
requiring the fewest objective function evaluations overall (387,335)
compared with NUMBER(T) (527,444) and RANDOM(T) (543,201). This improvement
is particularly noteworthy given that all three strategies achieve
the same overall success rate (0.85), indicating that the performance
advantage arises purely from efficiency rather than reliability differences.
This trend is visible across multiple test functions. In the Attractive
Sector family (25--150 dimensions), MIGRANT(T) demonstrates uniformly
superior performance. For example, in Attractive Sector\_25, it requires
1697 calls, compared with 1743 for NUMBER(T) and 1756 for RANDOM(T).
As dimensionality increases, this advantage becomes even more pronounced.
The performance gap becomes even more substantial in multimodal landscapes
such as Buche--Rastrigin. In Buche Rastrigin\_25, MIGRANT(T) needs
5893 function calls (success rate 0.90), significantly fewer than
NUMBER(T) (12,243) and RANDOM(T) (12,035). The difference becomes
overwhelming in the highest-dimensional case (Buche Rastrigin\_150):
MIGRANT(T) completes the optimization with 23,466 calls, while NUMBER(T)
requires 40,240, and RANDOM(T) needs 39,263. These results underline
MIGRANT(T)’s superior adaptability in sharply multimodal and high-variance
landscapes. For example, Ellipsoidal\_150 is solved in 16,930 calls
by MIGRANT(T), compared with 19,311 for NUMBER(T) and 19,940 for RANDOM(T).
This difference becomes particularly important for large-scale smooth
problems, where maintaining efficiency is critical.

Overall, the evidence strongly indicates that MIGRANT(T) is the most
effective differential weight mechanism among the tested variants.
It consistently reduces the number of function evaluations across
a wide variety of functions both unimodal and multimodal while preserving
identical success rates. This combination of efficiency, robustness,
and stability makes MIGRANT(T) a particularly advantageous choice
for enhancing the performance of DE in high-dimensional and challenging
optimization scenarios.

{\scriptsize{}
\begin{table}[H]
{\scriptsize\caption{Experiments using different weight selection for the proposed method.\label{tab:Experiments-using-different}}
}{\scriptsize\par}

{\tiny{}%
\begin{tabular}{|c|c|c|c|}
\hline 
{\tiny\textbf{FUNCTION}} & {\tiny\textbf{MIGRANT (T)}} & {\tiny\textbf{NUMBER (T)}} & {\tiny\textbf{RANDOM (T)}}\tabularnewline
\hline 
\hline 
\begin{cellvarwidth}[t]
\centering
{\tiny ATTRACTIVE}{\tiny\par}

{\tiny SECTOR\_25}
\end{cellvarwidth} & 1697 & 1743 & 1756\tabularnewline
\hline 
\begin{cellvarwidth}[t]
\centering
{\tiny ATTRACTIVE}{\tiny\par}

{\tiny SECTOR\_50}
\end{cellvarwidth} & 1761 & 1823 & 1828\tabularnewline
\hline 
\begin{cellvarwidth}[t]
\centering
{\tiny ATTRACTIVE}{\tiny\par}

{\tiny SECTOR\_100}
\end{cellvarwidth} & 1832 & 1879 & 1880\tabularnewline
\hline 
\begin{cellvarwidth}[t]
\centering
{\tiny ATTRACTIVE}{\tiny\par}

{\tiny SECTOR\_150}
\end{cellvarwidth} & 1867 & 1900 & 1920\tabularnewline
\hline 
{\tiny BUCHE RASTRIGIN\_25} & 5893(0.90) & 12243(0.90) & 12035(0.90)\tabularnewline
\hline 
{\tiny BUCHE RASTRIGIN\_50} & 12585(0.50) & 19529(0.50) & 20457(0.50)\tabularnewline
\hline 
{\tiny BUCHE RASTRIGIN\_100} & 16490(0.53) & 30055(0.53) & 31465(0.53)\tabularnewline
\hline 
{\tiny BUCHE RASTRIGIN\_150} & 23466(0.27) & 40240(0.27) & 39263(0.27)\tabularnewline
\hline 
{\tiny DISCUS\_25} & 1992 & 1857 & 1896\tabularnewline
\hline 
{\tiny DISCUS\_50} & 2060 & 1926 & 1971\tabularnewline
\hline 
{\tiny DISCUS\_100} & 2104 & 1978 & 1989\tabularnewline
\hline 
{\tiny DISCUS\_150} & 2144 & 2006 & 2040\tabularnewline
\hline 
{\tiny DIFFERENTPOWERS\_25} & 6478 & 11422 & 11629\tabularnewline
\hline 
{\tiny DIFFERENTPOWERS\_50} & 11183 & 15258 & 15179\tabularnewline
\hline 
{\tiny DIFFERENTPOWERS\_100} & 16225 & 21451 & 20659\tabularnewline
\hline 
{\tiny DIFFERENTPOWERS\_150} & 21495 & 24429 & 24670\tabularnewline
\hline 
{\tiny ELLIPSOIDAL\_25} & 3590 & 3751 & 3958\tabularnewline
\hline 
{\tiny ELLIPSOIDAL\_50} & 6424 & 6864 & 7184\tabularnewline
\hline 
{\tiny ELLIPSOIDAL\_100} & 11549 & 13756 & 13890\tabularnewline
\hline 
{\tiny ELLIPSOIDAL\_150} & 16930 & 19311 & 19940\tabularnewline
\hline 
{\tiny GALLAGHER21\_25} & 2261(0.90) & 3815(0.90) & 6364(0.90)\tabularnewline
\hline 
{\tiny GALLAGHER21\_50} & 4503(0.50) & 5277(0.50) & 9643(0.50)\tabularnewline
\hline 
{\tiny GALLAGHER21\_100} & 1756(0.53) & 1523(0.53) & 1521(0.53)\tabularnewline
\hline 
{\tiny GALLAGHER21\_150} & 1662(0.27) & 1521(0.27) & 1526(0.27)\tabularnewline
\hline 
{\tiny GALLAGHER101\_25} & 2769(0.90) & 3472(0.90) & 5657(0.90)\tabularnewline
\hline 
{\tiny GALLAGHER101\_50} & 4890(0.50) & 6950(0.50) & 7454(0.50)\tabularnewline
\hline 
{\tiny GALLAGHER101\_100} & 5886(0.53) & 6846(0.53) & 9505(0.53)\tabularnewline
\hline 
{\tiny GALLAGHER101\_150} & 8646(0.27) & 7701(0.27) & 12352(0.27)\tabularnewline
\hline 
{\tiny GRIEWANK \_25} & 4084 & 5276 & 5145\tabularnewline
\hline 
{\tiny GRIEWANK \_50} & 5039 & 5138 & 5729\tabularnewline
\hline 
{\tiny GRIEWANK \_100} & 6460 & 5726 & 6002\tabularnewline
\hline 
{\tiny GRIEWANK \_150} & 6542 & 5870 & 6164\tabularnewline
\hline 
{\tiny GRIEWANK\_ROSENBROCK\_25} & 4466 & 7458 & 6939\tabularnewline
\hline 
{\tiny GRIEWANK\_ROSENBROCK\_50} & 5325 & 9766 & 9255\tabularnewline
\hline 
{\tiny GRIEWANK\_ROSENBROCK\_100} & 6465 & 11776 & 11001\tabularnewline
\hline 
{\tiny GRIEWANK\_ROSENBROCK\_150} & 7272 & 13482 & 12543\tabularnewline
\hline 
{\tiny ROSENBROCK\_25} & 5950 & 7824 & 7955\tabularnewline
\hline 
{\tiny ROSENBROCK\_50} & 8963 & 13970 & 13057\tabularnewline
\hline 
{\tiny ROSENBROCK\_100} & 15930 & 23402 & 22348\tabularnewline
\hline 
{\tiny ROSENBROCK\_150} & 22135 & 32850 & 31562\tabularnewline
\hline 
{\tiny RARSTIGIN\_25} & 4577(0.90) & 9691(0.90) & 10242(0.90)\tabularnewline
\hline 
{\tiny RARSTIGIN\_50} & 7746(0.50) & 13134(0.50) & 12740(0.50)\tabularnewline
\hline 
{\tiny RARSTIGIN\_100} & 9147(0.53) & 13128(0.53) & 13184(0.53)\tabularnewline
\hline 
{\tiny RARSTIGIN\_150} & 11620(0.27) & 15105(0.27) & 15602(0.27)\tabularnewline
\hline 
{\tiny SPHERE\_25} & 1481 & 1507 & 1512\tabularnewline
\hline 
{\tiny SPHERE\_50} & 1509 & 1534 & 1539\tabularnewline
\hline 
{\tiny SPHERE\_100} & 1524 & 1555 & 1556\tabularnewline
\hline 
{\tiny SPHERE\_150} & 1535 & 1568 & 1567\tabularnewline
\hline 
{\tiny STEP ELLIPSOIDAL\_25} & 1625(0.90) & 1642(0.90) & 2090(0.90)\tabularnewline
\hline 
{\tiny STEP ELLIPSOIDAL\_50} & 2300(0.50) & 2774(0.50) & 4021(0.50)\tabularnewline
\hline 
{\tiny STEP ELLIPSOIDAL\_100} & 2465(0.53) & 1598(0.53) & 1571(0.53)\tabularnewline
\hline 
{\tiny STEP ELLIPSOIDAL\_150} & 3143(0.27) & 1531(0.27) & 1521(0.27)\tabularnewline
\hline 
{\tiny SHARP RIDGE\_25} & 5104 & 6215 & 6026\tabularnewline
\hline 
{\tiny SHARP RIDGE\_50} & 5226 & 6850 & 7123\tabularnewline
\hline 
{\tiny SHARP RIDGE\_100} & 5995 & 7782 & 7649\tabularnewline
\hline 
{\tiny SHARP RIDGE\_150} & 6481 & 8112 & 8237\tabularnewline
\hline 
{\tiny ZAKHAROV\_25} & 2185 & 2752 & 2639\tabularnewline
\hline 
{\tiny ZAKHAROV\_50} & 3027 & 4063 & 3864\tabularnewline
\hline 
{\tiny ZAKHAROV\_100} & 5572 & 6265 & 5634\tabularnewline
\hline 
{\tiny ZAKHAROV\_150} & 6304 & 7574 & 7553\tabularnewline
\hline 
 & 387335(0.85) & 527444(0.85) & 543201(0.85)\tabularnewline
\hline 
\end{tabular}}{\tiny\par}
\end{table}
}{\scriptsize\par}

Figure 2 presents a pairwise statistical comparison of the three migration
strategies MIGRANT (T), NUMBER (T), and RANDOM (T)with respect to
the required number of function evaluations. The Kruskal--Wallis
test (p = 0.3) does not indicate the presence of statistically significant
overall differences among the three strategies, suggesting comparable
distributions of computational cost. This outcome is consistent with
the results of the pairwise t-tests, for which none of the comparisons
reach statistical significance (p \textgreater{} 0.05). Accordingly,
the observed differences in medians and dispersion across the strategies
are not supported as statistically significant. The MIGRANT (T) strategy
exhibits a slightly lower average number of function evaluations;
however, this difference is not accompanied by statistical significance
and remains within the range of stochastic variability. Overall, the
results indicate similar behavior of the three migration strategies
in terms of computational cost. 

\begin{figure}
\centering{}\includegraphics[scale=0.5]{tournament_selection}\caption{A statistical comparison of the proposed with different weight selection.}
\end{figure}


\subsection{The effect of selection mechanism}

Table \ref{tab:Effect-of-Random}compares the four selection strategies
RANDOM(R), RANDOM(T), MIGRANT(R), and MIGRANT(T), where (R) denotes
purely random selection and (T) denotes tournament--based selection.
The results clearly show that MIGRANT(T) is by far the most efficient
method. It achieves the lowest total number of objective function
evaluations (387,335), significantly outperforming MIGRANT(R) (962,599),
RANDOM(T) (543,201), and RANDOM(R) (767,225). Since all methods achieve
the same success rate (0.85), the performance differences are due
solely to efficiency, demonstrating the importance of the selection
mechanism. This advantage becomes evident across nearly all tested
functions. For the Attractive Sector family (25--150 dimensions),
MIGRANT(T) consistently requires the fewest evaluations. For example,
in Attractive Sector\_25, it needs only 1697 calls, compared to 2174
MIGRANT(R), 1756 for RANDOM(T) and as many as 2162 for RANDOM(R).
The differences are even more striking for multimodal benchmarks such
as Buche--Rastrigin. In the 25-dimensional case, MIGRANT(T) achieves
5893 calls (0.90 success), while MIGRANT(R) needs 15,894, RANDOM(T)
needs 12,035, and RANDOM(R) 11,921. At the 150-dimensional level,
the gap widens dramatically: MIGRANT(T) requires 23,466 calls, whereas
MIGRANT(R) rises to 77,590, RANDOM(T) 39,263 and RANDOM(R) to 54,663.
These results highlight the strong stabilizing effect that tournament
selection has on the MIGRANT mechanism. The superiority of MIGRANT(T)
is even more pronounced in the Sharp Ridge functions. In Sharp Ridge\_150,
MIGRANT(T) completes the optimization with 6481 calls, while MIGRANT(R)
requires 12,053, RANDOM(T) 8237, and RANDOM(R) 12,395. Finally, in
the Zakharov functions, MIGRANT(T) again shows consistently superior
performance. In Zakharov\_150, MIGRANT(T) needs just 6304 calls, compared
to 25,370 for MIGRANT(R), 7553 RANDOM(T) and 16,240 for RANDOM(R).
Even in the easier 25-dimensional case, MIGRANT(T) requires 2185 calls,
whereas RANDOM(R) requires over twice as many (4605).

Overall, these results highlight the strong interaction between the
MIGRANT mechanism and tournament selection. Tournament selection dramatically
enhances the performance of MIGRANT, reducing the computational cost
by large margins across all functions while preserving identical success
rates. As a result, MIGRANT(T) emerges as the most balanced, stable,
and efficient strategy, making it highly suitable for optimization
scenarios where minimizing objective function evaluations is essential.

{\scriptsize{}
\begin{table}[H]
{\scriptsize\caption{Effect of Random and Tournament Selection Strategies on Optimization
Performance\label{tab:Effect-of-Random}}
}{\scriptsize\par}

{\tiny{}%
\begin{tabular}{|c|c|c|c|c|}
\hline 
{\tiny\textbf{FUNCTION}} & {\tiny\textbf{MIGRANT (T)}} & {\tiny\textbf{MIGRANT (R)}} & {\tiny\textbf{RANDOM (T)}} & {\tiny\textbf{RANDOM (R)}}\tabularnewline
\hline 
\hline 
\begin{cellvarwidth}[t]
\centering
{\tiny ATTRACTIVE}{\tiny\par}

{\tiny SECTOR\_25}
\end{cellvarwidth} & 1697 & 2174 & 1756 & 2162\tabularnewline
\hline 
\begin{cellvarwidth}[t]
\centering
{\tiny ATTRACTIVE}{\tiny\par}

{\tiny SECTOR\_50}
\end{cellvarwidth} & 1761 & 2212 & 1828 & 2162\tabularnewline
\hline 
\begin{cellvarwidth}[t]
\centering
{\tiny ATTRACTIVE}{\tiny\par}

{\tiny SECTOR\_100}
\end{cellvarwidth} & 1832 & 2177 & 1880 & 2192\tabularnewline
\hline 
\begin{cellvarwidth}[t]
\centering
{\tiny ATTRACTIVE}{\tiny\par}

{\tiny SECTOR\_150}
\end{cellvarwidth} & 1867 & 2206 & 1920 & 2174\tabularnewline
\hline 
{\tiny BUCHE RASTRIGIN\_25} & 5893(0.90) & 15894(0.90) & 12035(0.90) & 11921(0.90)\tabularnewline
\hline 
{\tiny BUCHE RASTRIGIN\_50} & 12585(0.50) & 50438(0.50) & 20457(0.50) & 20542(0.50)\tabularnewline
\hline 
{\tiny BUCHE RASTRIGIN\_100} & 16490(0.53) & 59214(0.53) & 31465(0.53) & 34570(0.53)\tabularnewline
\hline 
{\tiny BUCHE RASTRIGIN\_150} & 23466(0.27) & 77590(0.27) & 39263(0.27) & 54663(0.27)\tabularnewline
\hline 
{\tiny DISCUS\_25} & 1992 & 2588 & 1896 & 2542\tabularnewline
\hline 
{\tiny DISCUS\_50} & 2060 & 2601 & 1971 & 2552\tabularnewline
\hline 
{\tiny DISCUS\_100} & 2104 & 2553 & 1989 & 2617\tabularnewline
\hline 
{\tiny DISCUS\_150} & 2144 & 2608 & 2040 & 2591\tabularnewline
\hline 
{\tiny DIFFERENTPOWERS\_25} & 6478 & 13918 & 11629 & 14477\tabularnewline
\hline 
{\tiny DIFFERENTPOWERS\_50} & 11183 & 20100 & 15179 & 20064\tabularnewline
\hline 
{\tiny DIFFERENTPOWERS\_100} & 16225 & 27396 & 20659 & 29408\tabularnewline
\hline 
{\tiny DIFFERENTPOWERS\_150} & 21495 & 35710 & 24670 & 35070\tabularnewline
\hline 
{\tiny ELLIPSOIDAL\_25} & 3590 & 6424 & 3958 & 5932\tabularnewline
\hline 
{\tiny ELLIPSOIDAL\_50} & 6424 & 11704 & 7184 & 10585\tabularnewline
\hline 
{\tiny ELLIPSOIDAL\_100} & 11549 & 20736 & 13890 & 20887\tabularnewline
\hline 
{\tiny ELLIPSOIDAL\_150} & 16930 & 29835 & 19940 & 28265\tabularnewline
\hline 
{\tiny GALLAGHER21\_25} & 2261(0.90) & 5412(0.90) & 6364(0.90) & 2891(0.90)\tabularnewline
\hline 
{\tiny GALLAGHER21\_50} & 4503(0.50) & 11988(0.50) & 9643(0.50) & 3311(0.50)\tabularnewline
\hline 
{\tiny GALLAGHER21\_100} & 1756(0.53) & 1565(0.53) & 1521(0.53) & 1524(0.53)\tabularnewline
\hline 
{\tiny GALLAGHER21\_150} & 1662(0.27) & 1490(0.27) & 1526(0.27) & 1520(0.27)\tabularnewline
\hline 
{\tiny GALLAGHER101\_25} & 2769(0.90) & 5180(0.90) & 5657(0.90) & 3016(0.90)\tabularnewline
\hline 
{\tiny GALLAGHER101\_50} & 4890(0.50) & 21179(0.50) & 7454(0.50) & 4878(0.50)\tabularnewline
\hline 
{\tiny GALLAGHER101\_100} & 5886(0.53) & 26739(0.53) & 9505(0.53) & 4507(0.53)\tabularnewline
\hline 
{\tiny GALLAGHER101\_150} & 8646(0.27) & 46866(0.27) & 12352(0.27) & 3400(0.27)\tabularnewline
\hline 
{\tiny GRIEWANK \_25} & 4084 & 8148 & 5145 & 9902\tabularnewline
\hline 
{\tiny GRIEWANK \_50} & 5039 & 7894 & 5729 & 5203\tabularnewline
\hline 
{\tiny GRIEWANK \_100} & 6460 & 9083 & 6002 & 4145\tabularnewline
\hline 
{\tiny GRIEWANK \_150} & 6542 & 9154 & 6164 & 4075\tabularnewline
\hline 
{\tiny GRIEWANK\_ROSENBROCK\_25} & 4466 & 11510 & 6939 & 17429\tabularnewline
\hline 
{\tiny GRIEWANK\_ROSENBROCK\_50} & 5325 & 14658 & 9255 & 24666\tabularnewline
\hline 
{\tiny GRIEWANK\_ROSENBROCK\_100} & 6465 & 15890 & 11001 & 34019\tabularnewline
\hline 
{\tiny GRIEWANK\_ROSENBROCK\_150} & 7272 & 17910 & 12543 & 39208\tabularnewline
\hline 
{\tiny ROSENBROCK\_25} & 5950 & 13718 & 7955 & 15591\tabularnewline
\hline 
{\tiny ROSENBROCK\_50} & 8963 & 21827 & 13057 & 23980\tabularnewline
\hline 
{\tiny ROSENBROCK\_100} & 15930 & 34948 & 22348 & 40245\tabularnewline
\hline 
{\tiny ROSENBROCK\_150} & 22135 & 49061 & 31562 & 53073\tabularnewline
\hline 
{\tiny RARSTIGIN\_25} & 4577(0.90) & 11276(0.90) & 10242(0.90) & 9910(0.90)\tabularnewline
\hline 
{\tiny RARSTIGIN\_50} & 7746(0.50) & 26967(0.50) & 12740(0.50) & 14234(0.50)\tabularnewline
\hline 
{\tiny RARSTIGIN\_100} & 9147(0.53) & 27639(0.53) & 13184(0.53) & 16666(0.53)\tabularnewline
\hline 
{\tiny RARSTIGIN\_150} & 11620(0.27) & 34865(0.27) & 15602(0.27) & 19135(0.27)\tabularnewline
\hline 
{\tiny SPHERE\_25} & 1481 & 1620 & 1512 & 1627\tabularnewline
\hline 
{\tiny SPHERE\_50} & 1509 & 1641 & 1539 & 1634\tabularnewline
\hline 
{\tiny SPHERE\_100} & 1524 & 1635 & 1556 & 1644\tabularnewline
\hline 
{\tiny SPHERE\_150} & 1535 & 1644 & 1567 & 1639\tabularnewline
\hline 
{\tiny STEP ELLIPSOIDAL\_25} & 1625(0.90) & 2073(0.90) & 2090(0.90) & 1750(0.90)\tabularnewline
\hline 
{\tiny STEP ELLIPSOIDAL\_50} & 2300(0.50) & 5937(0.50) & 4021(0.50) & 1664(0.50)\tabularnewline
\hline 
{\tiny STEP ELLIPSOIDAL\_100} & 2465(0.53) & 6546(0.53) & 1571(0.53) & 1523(0.53)\tabularnewline
\hline 
{\tiny STEP ELLIPSOIDAL\_150} & 3143(0.27) & 11487(0.27) & 1521(0.27) & 1520(0.27)\tabularnewline
\hline 
{\tiny SHARP RIDGE\_25} & 5104 & 10153 & 6026 & 11776\tabularnewline
\hline 
{\tiny SHARP RIDGE\_50} & 5226 & 11108 & 7123 & 12123\tabularnewline
\hline 
{\tiny SHARP RIDGE\_100} & 5995 & 11592 & 7649 & 12704\tabularnewline
\hline 
{\tiny SHARP RIDGE\_150} & 6481 & 12053 & 8237 & 12395\tabularnewline
\hline 
{\tiny ZAKHAROV\_25} & 2185 & 3941 & 2639 & 4605\tabularnewline
\hline 
{\tiny ZAKHAROV\_50} & 3027 & 8972 & 3864 & 7963\tabularnewline
\hline 
{\tiny ZAKHAROV\_100} & 5572 & 23782 & 5634 & 14514\tabularnewline
\hline 
{\tiny ZAKHAROV\_150} & 6304 & 25370 & 7553 & 16240\tabularnewline
\hline 
 & 387335(0.85) & 962599(0.85) & 543201(0.85) & 767225(0.85)\tabularnewline
\hline 
\end{tabular}}{\tiny\par}
\end{table}
}{\scriptsize\par}

Figure 3 presents a pairwise statistical comparison of the four strategies
MIGRANT (T), MIGRANT (R), RANDOM (T) and RANDOM (R) based on their
distributions of function calls. The Kruskal--Wallis test indicates
a statistically significant overall difference among the groups (p
= 0.0019), suggesting that at least one strategy differs from the
others in terms of computational cost. To further examine these differences,
pairwise t-tests were conducted, with p-values annotated using standard
significance notation (ns: p \textgreater{} 0.05, : p \textless{}
0.05, {*}: p \textless{} 0.01, {*}{*}{*}: p \textless{} 0.001, {*}{*}{*}{*}:
p \textless{} 0.0001) The pairwise analysis shows that MIGRANT (T)
differs significantly from MIGRANT (R) and RANDOM (R), with the corresponding
comparisons reaching higher levels of statistical significance. The
strategies MIGRANT (R) and RANDOM (T) exhibit intermediate behavior,
with several pairwise comparisons indicating statistically significant
differences at moderate significance levels. Comparisons labeled as
“ns” indicate pairs for which no statistically significant differences
are detected. Overall, the distributions of function calls indicate
lower evaluation counts for the MIGRANT-based strategies, particularly
MIGRANT (T), relative to the RANDOM-based variants. These results
describe differences in computational cost among the examined strategies
under the considered experimental conditions.

\begin{figure}
\centering{}\includegraphics[scale=0.5]{comparison_tournament_random_selection}\caption{Statistical Comparison of Random and Tournament Selection Strategies
on Optimization Performance.}
\end{figure}


\subsection{The effect of sampling method}

In Table \ref{tab:Experiments-on-the} tournament selection is used
to choose the samples that participate in the core operator of DE.
Four strategies for computing the differential weight are evaluated:
random weight with uniform sampling (Random(U)), random weight with
k-means sampling\citep{kmeans1,kmeans2} (Random(K)), MIGRANT weight
with uniform sampling (Migrant(U)), and MIGRANT weight with k-means
sampling (Migrant(K)). The k-means method, originally proposed by
MacQueen\citep{MacQueen} and used extensively in later work {[}65,
66{]}, is employed not only to determine cluster centers but also
as a structured sampling mechanism. Across all test functions, MIGRANT(K)
consistently achieves the lowest total number of function calls (387,335)
with a success rate of 0.85, outperforming all other sampling strategies.
This advantage becomes clear when examining individual benchmarks.
For the Attractive Sector family (dimensions 25--150), MIGRANT(K)
systematically requires fewer evaluations than MIGRANT(U) and both
Random methods. For example, in Attractive Sector\_25, MIGRANT(K)
uses only 1697 evaluations compared to 1738 for MIGRANT(U), while
Random(K) and Random(U) require 1756 and 1792 respectively. This pattern
holds across all dimensionalities, showing the benefit of structured
sampling in unimodal landscapes. The effect becomes far more pronounced
in multimodal functions such as Buche Rastrigin. In Buche Rastrigin\_25,
MIGRANT(K) needs 5893 function calls with a success rate of 0.90,
in contrast to MIGRANT(U)'s 12,818 calls (0.03). Random(K) performs
similarly to MIGRANT(K) in success rate but requires more evaluations
(12,035), while Random(U) is by far the least efficient (28,865 calls).
The difference becomes dramatic in higher dimensions: in Buche Rastrigin\_150,
MIGRANT(K) performs the task in 23,466 calls (0.27), whereas Random(U)
escalates to 90,211, showing the instability of uniform sampling in
complex landscapes. Although differences here are smaller due to the
problem’s structure, MIGRANT(K) preserves its advantage in stability.
The superiority of MIGRANT(K) is again evident in the Step Ellipsoidal
group. In Step Ellipsoidal\_25, MIGRANT(K) requires only 5104 calls,
remarkably lower than Random(U) (6699), Random (K) (6026) and still
better than MIGRANT(U) (5014, but with lower success). Finally, for
the Zakharov functions, MIGRANT(K) again shows the best balance between
evaluation cost and success rate. In Zakharov\_25, MIGRANT(K) requires
only 2185 evaluations, beating Migrant(U) (2283), Random(K) (2639)
and Random(U) (2797). 

Overall, integrating k-means sampling into the MIGRANT strategy leads
to substantial improvements in both efficiency and reliability. MIGRANT(K)
not only requires the fewest total function evaluations but also maintains
high success rates across diverse problem categories, making it the
most effective approach for the benchmark set. In contrast, Random(U)
repeatedly demonstrates the lowest efficiency, highlighting the advantage
of structured sampling over uniform dispersion in high-dimensional
optimization.

{\scriptsize{}
\begin{table}[H]
{\scriptsize\caption{Experiments on the performance of DE using sampling methods\label{tab:Experiments-on-the}}
}{\scriptsize\par}

{\tiny{}%
\begin{tabular}{|c|c|c|c|c|}
\hline 
{\tiny\textbf{FUNCTION}} & {\tiny\textbf{MIGRANT (K)}} & {\tiny\textbf{MIGRANT (U)}} & {\tiny\textbf{RANDOM (K)}} & {\tiny\textbf{RANDOM (U)}}\tabularnewline
\hline 
\hline 
\begin{cellvarwidth}[t]
\centering
{\tiny ATTRACTIVE}{\tiny\par}

{\tiny SECTOR\_25}
\end{cellvarwidth} & 1697 & 1738 & 1756 & 1792\tabularnewline
\hline 
\begin{cellvarwidth}[t]
\centering
{\tiny ATTRACTIVE}{\tiny\par}

{\tiny SECTOR\_50}
\end{cellvarwidth} & 1761 & 1792 & 1828 & 1832\tabularnewline
\hline 
\begin{cellvarwidth}[t]
\centering
{\tiny ATTRACTIVE}{\tiny\par}

{\tiny SECTOR\_100}
\end{cellvarwidth} & 1832 & 1866 & 1880 & 1891\tabularnewline
\hline 
\begin{cellvarwidth}[t]
\centering
{\tiny ATTRACTIVE}{\tiny\par}

{\tiny SECTOR\_150}
\end{cellvarwidth} & 1867 & 1890 & 1920 & 1920\tabularnewline
\hline 
{\tiny BUCHE RASTRIGIN\_25} & 5893(0.90) & 12818(0.03) & 12035(0.90) & 28865(0.03)\tabularnewline
\hline 
{\tiny BUCHE RASTRIGIN\_50} & 12585(0.50) & 23622(0.03) & 20457(0.50) & 34379(0.03)\tabularnewline
\hline 
{\tiny BUCHE RASTRIGIN\_100} & 16490(0.53) & 41526(0.03) & 31465(0.53) & 70319(0.03)\tabularnewline
\hline 
{\tiny BUCHE RASTRIGIN\_150} & 23466(0.27) & 55612(0.03) & 39263(0.27) & 90211(0.03)\tabularnewline
\hline 
{\tiny DIFFERENT POWERS\_25} & 1992 & 2016 & 1896 & 1936\tabularnewline
\hline 
{\tiny DIFFERENT POWERS\_50} & 2060 & 2077 & 1971 & 1989\tabularnewline
\hline 
{\tiny DIFFERENT POWERS\_100} & 2104 & 2114 & 1989 & 2026\tabularnewline
\hline 
{\tiny DIFFERENT POWERS\_150} & 2144 & 2150 & 2040 & 2058\tabularnewline
\hline 
{\tiny DISCUS\_25} & 6478 & 7368 & 11629 & 11484\tabularnewline
\hline 
{\tiny DISCUS\_50} & 11183 & 11666 & 15179 & 15789\tabularnewline
\hline 
{\tiny DISCUS\_100} & 16225 & 17566 & 20659 & 21459\tabularnewline
\hline 
{\tiny DISCUS\_150} & 21495 & 22526 & 24670 & 24485\tabularnewline
\hline 
{\tiny ELLIPSOIDAL\_25} & 3590 & 3640 & 3958 & 3873\tabularnewline
\hline 
{\tiny ELLIPSOIDAL\_50} & 6424 & 6399 & 7184 & 7022\tabularnewline
\hline 
{\tiny ELLIPSOIDAL\_100} & 11549 & 12161 & 13890 & 13610\tabularnewline
\hline 
{\tiny ELLIPSOIDAL\_150} & 16930 & 17905 & 19940 & 19576\tabularnewline
\hline 
{\tiny GALLAGHER21\_25} & 2261(0.90) & 6920(0.03) & 6364(0.90) & 34112(0.03)\tabularnewline
\hline 
{\tiny GALLAGHER21\_50} & 4503(0.50) & 7904(0.03) & 9643(0.50) & 17404(0.03)\tabularnewline
\hline 
{\tiny GALLAGHER21\_100} & 1756(0.53) & 1463 & 1521(0.53) & 1524(0.53)\tabularnewline
\hline 
{\tiny GALLAGHER21\_150} & 1662(0.27) & 1463 & 1526(0.27) & 1522(0.27)\tabularnewline
\hline 
{\tiny GALLAGHER101\_25} & 2769(0.90) & 6395(0.03) & 5657(0.90) & 27324(0.03)\tabularnewline
\hline 
{\tiny GALLAGHER101\_50} & 4890(0.50) & 8204(0.03) & 7454(0.50) & 17075(0.03)\tabularnewline
\hline 
{\tiny GALLAGHER101\_100} & 5886(0.53) & 10816(0.03) & 9505(0.53) & 18232(0.03)\tabularnewline
\hline 
{\tiny GALLAGHER101\_150} & 8646(0.27) & 12129(0.03) & 12352(0.27) & 17231(0.03)\tabularnewline
\hline 
\begin{cellvarwidth}[t]
\centering
{\tiny GRIEWANK}{\tiny\par}

{\tiny ROSENBROCK\_25}
\end{cellvarwidth} & 4084 & 4353 & 5145 & 5434\tabularnewline
\hline 
\begin{cellvarwidth}[t]
\centering
{\tiny GRIEWANK}{\tiny\par}

{\tiny ROSENBROCK\_50}
\end{cellvarwidth} & 5039 & 5290 & 5729 & 5631\tabularnewline
\hline 
\begin{cellvarwidth}[t]
\centering
{\tiny GRIEWANK}{\tiny\par}

{\tiny ROSENBROCK\_100}
\end{cellvarwidth} & 6460 & 6211 & 6002 & 5916\tabularnewline
\hline 
\begin{cellvarwidth}[t]
\centering
{\tiny GRIEWANK}{\tiny\par}

{\tiny ROSENBROCK\_150}
\end{cellvarwidth} & 6542 & 6895 & 6164 & 6113\tabularnewline
\hline 
{\tiny GRIEWANK\_25} & 4466 & 4818 & 6939 & 7697\tabularnewline
\hline 
{\tiny GRIEWANK\_50} & 5325 & 7163 & 9255 & 11056\tabularnewline
\hline 
{\tiny GRIEWANK\_100} & 6465 & 9992 & 11001 & 15311\tabularnewline
\hline 
{\tiny GRIEWANK\_150} & 7272 & 12350 & 12543 & 19125\tabularnewline
\hline 
{\tiny RARSTIGIN\_25} & 5950 & 5909(0.03) & 7955 & 8447(0.03)\tabularnewline
\hline 
{\tiny RARSTIGIN\_50} & 8963 & 10112(0.03) & 13057 & 13669(0.03)\tabularnewline
\hline 
{\tiny RARSTIGIN\_100} & 15930 & 16541(0.03) & 22348 & 23760(0.03)\tabularnewline
\hline 
{\tiny RARSTIGIN\_150} & 22135 & 23181(0.03) & 31562 & 33005(0.03)\tabularnewline
\hline 
{\tiny ROSENBROCK\_25} & 4577(0.90) & 9432 & 10242(0.90) & 18663\tabularnewline
\hline 
{\tiny ROSENBROCK\_50} & 7746(0.50) & 11863 & 12740(0.50) & 27806\tabularnewline
\hline 
{\tiny ROSENBROCK\_100} & 9147(0.53) & 15307 & 13184(0.53) & 28064\tabularnewline
\hline 
{\tiny ROSENBROCK\_150} & 11620(0.27) & 18904 & 15602(0.27) & 43292\tabularnewline
\hline 
{\tiny SHARP RIDGE\_25} & 1481 & 1498 & 1512 & 1528\tabularnewline
\hline 
{\tiny SHARP RIDGE\_50} & 1509 & 1516 & 1539 & 1548\tabularnewline
\hline 
{\tiny SHARP RIDGE\_100} & 1524 & 1531 & 1556 & 1559\tabularnewline
\hline 
{\tiny SHARP RIDGE\_150} & 1535 & 1548 & 1567 & 1565\tabularnewline
\hline 
{\tiny SPHERE\_25} & 1625(0.90) & 2733 & 2090(0.90) & 7103\tabularnewline
\hline 
{\tiny SPHERE\_50} & 2300(0.50) & 3173 & 4021(0.50) & 6384\tabularnewline
\hline 
{\tiny SPHERE\_100} & 2465(0.53) & 3654 & 1571(0.53) & 5873\tabularnewline
\hline 
{\tiny SPHERE\_150} & 3143(0.27) & 4073 & 1521(0.27) & 5149\tabularnewline
\hline 
{\tiny STEP ELLIPSOIDAL\_25} & 5104 & 5014(0.03) & 6026 & 6699(0.03)\tabularnewline
\hline 
{\tiny STEP ELLIPSOIDAL\_50} & 5226 & 5581(0.03) & 7123 & 7205(0.03)\tabularnewline
\hline 
{\tiny STEP ELLIPSOIDAL\_100} & 5995 & 6091(0.03) & 7649 & 7893(0.03)\tabularnewline
\hline 
{\tiny STEP ELLIPSOIDAL\_150} & 6481 & 5996(0.03) & 8237 & 8037(0.03)\tabularnewline
\hline 
{\tiny ZAKHAROV\_25} & 2185 & 2283 & 2639 & 2797\tabularnewline
\hline 
{\tiny ZAKHAROV\_50} & 3027 & 2901 & 3864 & 3743\tabularnewline
\hline 
{\tiny ZAKHAROV\_100} & 5572 & 4122 & 5634 & 5936\tabularnewline
\hline 
{\tiny ZAKHAROV\_150} & 6304 & 5282 & 7553 & 7460\tabularnewline
\hline 
 & 387335(0.85) & 529063(0.71) & 543201(0.85) & 844408(0.69)\tabularnewline
\hline 
\end{tabular}}{\tiny\par}
\end{table}
}{\scriptsize\par}

Figure 4 presents the pairwise t-test statistical comparison of the
four strategies MIGRANT (K), RANDOM (K), MIGRANT (U) and RANDOM (U)
based on their distributions of function calls. The Kruskal--Wallis
test indicates a statistically significant overall difference among
the groups (p = 0.034), suggesting that at least one strategy differs
from the others in terms of computational cost. To further examine
these differences, pairwise t-tests were conducted, with p-values
annotated using conventional significance notation (ns: p \textgreater{}
0.05, : p \textless{} 0.05, {*}: p \textless{} 0.01, {*}{*}{*}: p
\textless{} 0.001, {*}{*}{*}{*}: p \textless{} 0.0001). The pairwise
comparisons indicate that MIGRANT (K) differs significantly from RANDOM
(K) and from MIGRANT (U), while no statistically significant difference
is observed between MIGRANT (U) and RANDOM (U). Comparisons labeled
as “ns” suggest statistically indistinguishable behavior between the
corresponding strategy pairs. Overall, the distributions of function
calls show lower evaluation counts for MIGRANT (K) relative to some
of the other strategies under the examined conditions. The U-type
variants exhibit similar behavior to the RANDOM (U) baseline. These
results describe differences in computational cost among the considered
strategies, without implying uniform or dominant superiority across
all comparisons.

\begin{figure}
\centering{}\includegraphics[scale=0.5]{sample}\caption{Statistical Comparison of Different Sampling Method Combinations in
DE Performance.}
\end{figure}


\subsection{The effect of local search rate }

From Table \ref{tab:Experiments-on-the-1} we observe the influence
of periodic local optimization on the performance of the MIGRANT method,
considering four different local search rates: 0.005, 0.01, 0.03,
and 0.05. Among all settings, the 0.005 rate achieves the lowest total
number of function calls (148,027) while maintaining a high success
rate of 0.85, thus providing the best balance between computational
efficiency and optimization reliability. This advantage is consistently
reflected across the benchmark functions. In the Attractive Sector
functions (25--150 dimensions), the 0.005 rate clearly outperforms
the higher-rate configurations. For instance, in Attractive Sector\_25,
it requires only 1441 calls, compared to 1603 for the 0.03 rate and
1697 for the 0.05 rate. The improvement persists as the dimensionality
increases: Attractive Sector\_150 is solved with 1536 calls at the
0.005 rate, while the 0.05 rate requires 1867 calls. The improvement
becomes dramatically more pronounced in the Buche--Rastrigin family,
where the complexity and multimodality amplify the benefit of lower
local search frequency. For Buche Rastrigin\_25, the 0.005 method
requires 2035 function calls (0.90 success), whereas the 0.05 rate
jumps to 5893 calls nearly triple. In the high-dimensional case Buche
Rastrigin\_150, the difference is even more striking: 5900 calls at
the 0.005 rate versus 23,466 for the 0.05 rate. A similar trend can
be seen in the Discus, Sharp Ridge, and Step Ellipsoidal functions.
In Step Ellipsoidal\_50, the 0.005 rate achieves 2136 calls (0.50),
far below the 0.05 rate (2300). For Step Ellipsoidal\_150, the 0.005
variant uses 2914 calls (0.27), while the 0.05 rate needs 3143 calls.
Even in unimodal functions like Discus, the 0.005 method consistently
leads to lower evaluation costs e.g., Discus\_25 requires 1525 calls
vs. 1992 for the 0.05 rate. The Sharp Ridge functions highlight this
behavior even more strongly. For Sharp Ridge\_25, the 0.005 rate requires
only 1934 calls, in contrast to 5104 calls for the 0.05 rate more
than a 2.5\texttimes{} increase. Similar improvements appear in Sharp
Ridge\_150, where the function calls rise from 2350 at 0.005 to 6481
at 0.05. 

In summary, using a lower local search rate specifically the 0.005
setting results in the most efficient optimization behavior across
all tested functions. This variant provides the lowest objective function
calls without compromising success rate, making it the optimal choice
when both efficiency and reliability are essential in high-dimensional
optimization tasks.

{\scriptsize{}
\begin{table}[H]
{\scriptsize\caption{Experiments on the Effect of Local Search Rate on Optimization Performance
in DE\textbf{.\label{tab:Experiments-on-the-1}}}
}{\scriptsize\par}

{\tiny{}%
\begin{tabular}{|c|c|c|c|c|}
\hline 
{\tiny\textbf{FUNCTION}} & {\footnotesize MIGRANT(0.005)} & {\footnotesize MIGRANT(0.01)} & {\footnotesize MIGRANT(0.03)} & {\footnotesize MIGRANT(0.05)}\tabularnewline
\hline 
\hline 
\begin{cellvarwidth}[t]
\centering
{\tiny ATTRACTIVE}{\tiny\par}

{\tiny SECTOR\_25}
\end{cellvarwidth} & 1441 & 1472 & 1603 & 1697\tabularnewline
\hline 
\begin{cellvarwidth}[t]
\centering
{\tiny ATTRACTIVE}{\tiny\par}

{\tiny SECTOR\_50}
\end{cellvarwidth} & 1582 & 1522 & 1674 & 1761\tabularnewline
\hline 
\begin{cellvarwidth}[t]
\centering
{\tiny ATTRACTIVE}{\tiny\par}

{\tiny SECTOR\_100}
\end{cellvarwidth} & 1516 & 1552 & 1699 & 1832\tabularnewline
\hline 
\begin{cellvarwidth}[t]
\centering
{\tiny ATTRACTIVE}{\tiny\par}

{\tiny SECTOR\_150}
\end{cellvarwidth} & 1536 & 1560 & 1726 & 1867\tabularnewline
\hline 
{\tiny BUCHE RASTRIGIN\_25} & 2035(0.90) & 2502(0.90) & 4323(0.90) & 5893(0.90)\tabularnewline
\hline 
{\tiny BUCHE RASTRIGIN\_50} & 3468(0.50) & 4319(0.50) & 8496(0.50) & 12585(0.50)\tabularnewline
\hline 
{\tiny BUCHE RASTRIGIN\_100} & 4179(0.53) & 5700(0.53) & 10756(0.53) & 16490(0.53)\tabularnewline
\hline 
{\tiny BUCHE RASTRIGIN\_150} & 5900(0.27) & 7794(0.27) & 14818(0.27) & 23466(0.27)\tabularnewline
\hline 
{\tiny DISCUS\_25} & 1525 & 1616 & 1841 & 1992\tabularnewline
\hline 
{\tiny DISCUS\_50} & 1615 & 1658 & 1919 & 2060\tabularnewline
\hline 
{\tiny DISCUS\_100} & 1578 & 1655 & 1979 & 2104\tabularnewline
\hline 
{\tiny DISCUS\_150} & 1590 & 1663 & 1987 & 2144\tabularnewline
\hline 
{\tiny DIFFERENTPOWERS\_25} & 2296 & 2855 & 4661 & 6478\tabularnewline
\hline 
{\tiny DIFFERENTPOWERS\_50} & 3011 & 3807 & 7580 & 11183\tabularnewline
\hline 
{\tiny DIFFERENTPOWERS\_100} & 3827 & 5693 & 11214 & 16225\tabularnewline
\hline 
{\tiny DIFFERENTPOWERS\_150} & 4736 & 7158 & 15238 & 21495\tabularnewline
\hline 
{\tiny ELLIPSOIDAL\_25} & 1765 & 2011 & 2940 & 3590\tabularnewline
\hline 
{\tiny ELLIPSOIDAL\_50} & 2235 & 2844 & 4854 & 6424\tabularnewline
\hline 
{\tiny ELLIPSOIDAL\_100} & 3234 & 4557 & 9215 & 11549\tabularnewline
\hline 
{\tiny ELLIPSOIDAL\_150} & 4581 & 6620 & 12510 & 16930\tabularnewline
\hline 
{\tiny GALLAGHER21\_25} & 1751(0.90) & 1804(0.90) & 2049(0.90) & 2261(0.90)\tabularnewline
\hline 
{\tiny GALLAGHER21\_50} & 2842(0.50) & 3012(0.50) & 3765(0.50) & 4503(0.50)\tabularnewline
\hline 
{\tiny GALLAGHER21\_100} & 1432(0.53) & 1470(0.53) & 1609(0.53) & 1756(0.53)\tabularnewline
\hline 
{\tiny GALLAGHER21\_150} & 1434(0.27) & 1455(0.27) & 1554(0.27) & 1662(0.27)\tabularnewline
\hline 
{\tiny GALLAGHER101\_25} & 1778(0.90) & 1896(0.90) & 2359(0.90) & 2769(0.90)\tabularnewline
\hline 
{\tiny GALLAGHER101\_50} & 3287(0.50) & 3470(0.50) & 4186(0.50) & 4890(0.50)\tabularnewline
\hline 
{\tiny GALLAGHER101\_100} & 3550(0.53) & 3804(0.53) & 4851(0.53) & 5886(0.53)\tabularnewline
\hline 
{\tiny GALLAGHER101\_150} & 4726(0.27) & 5208(0.27) & 6959(0.27) & 8646(0.27)\tabularnewline
\hline 
{\tiny GRIEWANK \_25} & 1858 & 2102 & 3137 & 4084\tabularnewline
\hline 
{\tiny GRIEWANK \_50} & 2154 & 2407 & 3859 & 5039\tabularnewline
\hline 
{\tiny GRIEWANK \_100} & 2135 & 2688 & 4542 & 6460\tabularnewline
\hline 
{\tiny GRIEWANK \_150} & 2298 & 2937 & 4919 & 6542\tabularnewline
\hline 
{\tiny GRIEWANK\_ROSENBROCK\_25} & 1840 & 2116 & 3292 & 4466\tabularnewline
\hline 
{\tiny GRIEWANK\_ROSENBROCK\_50} & 2091 & 2661 & 4199 & 5325\tabularnewline
\hline 
{\tiny GRIEWANK\_ROSENBROCK\_100} & 2343 & 2969 & 4868 & 6465\tabularnewline
\hline 
{\tiny GRIEWANK\_ROSENBROCK\_150} & 2512 & 3295 & 5496 & 7272\tabularnewline
\hline 
{\tiny ROSENBROCK\_25} & 1964 & 2450 & 4091 & 5950\tabularnewline
\hline 
{\tiny ROSENBROCK\_50} & 2675 & 3619 & 6578 & 8963\tabularnewline
\hline 
{\tiny ROSENBROCK\_100} & 3616 & 5278 & 10570 & 15930\tabularnewline
\hline 
{\tiny ROSENBROCK\_150} & 5326 & 7819 & 15572 & 22135\tabularnewline
\hline 
{\tiny RARSTIGIN\_25} & 1831(0.90) & 2135(0.90) & 3346(0.90) & 4577(0.90)\tabularnewline
\hline 
{\tiny RARSTIGIN\_50} & 2858(0.50) & 3334(0.50) & 5664(0.50) & 7746(0.50)\tabularnewline
\hline 
{\tiny RARSTIGIN\_100} & 2941(0.53) & 3697(0.53) & 6336(0.53) & 9147(0.53)\tabularnewline
\hline 
{\tiny RARSTIGIN\_150} & 3997(0.27) & 4826(0.27) & 8275(0.27) & 11620(0.27)\tabularnewline
\hline 
{\tiny SPHERE\_25} & 1402 & 1411 & 1455 & 1481\tabularnewline
\hline 
{\tiny SPHERE\_50} & 1537 & 1444 & 1475 & 1509\tabularnewline
\hline 
{\tiny SPHERE\_100} & 1463 & 1454 & 1489 & 1524\tabularnewline
\hline 
{\tiny SPHERE\_150} & 1481 & 1469 & 1494 & 1535\tabularnewline
\hline 
{\tiny STEP ELLIPSOIDAL\_25} & 1513(0.90) & 1526(0.90) & 1576(0.90) & 1625(0.90)\tabularnewline
\hline 
{\tiny STEP ELLIPSOIDAL\_50} & 2136(0.50) & 2155(0.50) & 2229(0.50) & 2300(0.50)\tabularnewline
\hline 
{\tiny STEP ELLIPSOIDAL\_100} & 2286(0.53) & 2308(0.53) & 2389(0.53) & 2465(0.53)\tabularnewline
\hline 
{\tiny STEP ELLIPSOIDAL\_150} & 2914(0.27) & 2938(0.27) & 3040(0.27) & 3143(0.27)\tabularnewline
\hline 
{\tiny SHARP RIDGE\_25} & 1934 & 2269 & 3453 & 5104\tabularnewline
\hline 
{\tiny SHARP RIDGE\_50} & 2130 & 2680 & 4042 & 5226\tabularnewline
\hline 
{\tiny SHARP RIDGE\_100} & 2190 & 2718 & 4489 & 5995\tabularnewline
\hline 
{\tiny SHARP RIDGE\_150} & 2350 & 3107 & 5131 & 6481\tabularnewline
\hline 
{\tiny ZAKHAROV\_25} & 1570 & 1635 & 1912 & 2185\tabularnewline
\hline 
{\tiny ZAKHAROV\_50} & 1714 & 1884 & 2505 & 3027\tabularnewline
\hline 
{\tiny ZAKHAROV\_100} & 2428 & 2677 & 4597 & 5572\tabularnewline
\hline 
{\tiny ZAKHAROV\_150} & 2090 & 2314 & 4721 & 6304\tabularnewline
\hline 
 & 148027(0.85) & 178999(0.85) & 289106(0.85) & 387335(0.85)\tabularnewline
\hline 
\end{tabular}}{\tiny\par}
\end{table}
}{\scriptsize\par}

Figure 5 presents pairwise statistical comparisons among the four
parameter configurations (0.001, 0.01, 0.03, and 0.05) based on their
distributions of function evaluations. The global Kruskal--Wallis
test indicates a statistically significant overall difference among
the groups (p = 2.9e\textminus 08), suggesting that the choice of
parameter value is associated with differences in computational cost.
Pairwise comparisons were conducted using independent t-tests, with
p-values interpreted according to conventional significance notation
(ns: p \textgreater{} 0.05, : p \textless{} 0.05, {*}: p \textless{}
0.01, {*}{*}{*}: p \textless{} 0.001, {*}{*}{*}{*}: p \textless{}
0.0001). As shown in the figure, a substantial number of pairwise
comparisons reach statistically significant levels, while a smaller
subset does not exhibit statistically significant differences. This
indicates that, for many comparisons, the examined parameter configurations
are associated with distinguishable behavior in terms of the number
of function calls. Overall, the results indicate that different parameter
settings correspond to varying distributions of function evaluations
under the considered experimental conditions. The observed differences
suggest a systematic influence of the parameter choice on optimizer
behavior, without implying uniform or absolute superiority of a single
configuration across all comparisons.

\begin{figure}
\centering{}\includegraphics[scale=0.5]{lrate}\caption{Statistical comparison for the proposed method and different values
of parameter $p_{l}$.}
\end{figure}


\subsection*{3.7 Parameter Sensitivity Analysis}

To assess the robustness of the proposed method with respect to its
main control parameters, a sensitivity analysis was conducted focusing
on the local search rate $p_{l}$, the population size $NP$, and
the tournament size $N_{t}$. Each parameter was varied independently,
while all remaining parameters were fixed to their default values.
For each configuration, the algorithm was executed multiple times
under identical experimental conditions, and the average number of
function calls was recorded.

\begin{figure}

\includegraphics{lrate.jpeg}\caption{Sensitivity of function calls to the Local Search}

\end{figure}

Figure 6 illustrates the sensitivity of the number of function calls
with respect to the local search rate$p_{l}$. The results indicate
a gradual increase in computational cost as $p_{l}$increases, which
can be attributed to the more frequent activation of the local refinement
procedure. Nevertheless, the observed variation is smooth, and no
abrupt performance degradation is observed for moderate changes in
the local search rate.

\begin{figure}

\includegraphics{population.jpeg}\caption{Sensitivity of function calls to the Population Size}

\end{figure}

Figure 7 presents the effect of the population size $NP$ on the number
of function calls. As expected, increasing the population size leads
to a higher computational cost due to the larger number of individuals
evaluated at each iteration. However, the trend remains monotonic
and predictable, indicating stable scaling behavior rather than sensitivity
to a particular population size.

\begin{figure}

\includegraphics{tournamentsize.jpeg}\caption{Sensitivity of function calls to the Tournament Size}

\end{figure}

Figure 8 shows the sensitivity of the algorithm to the tournament
size $N_{t}$ when tournament-based parent selection is employed.
The results demonstrate that moderate values of $N_{t}$ yield comparable
performance, while larger tournament sizes slightly reduce the number
of function calls by increasing selection pressure. Importantly, the
performance differences across the tested range remain limited, suggesting
that the algorithm is not highly sensitive to the precise choice of
$N_{t}$ .

Overall, the sensitivity analysis indicates that the proposed method
exhibits stable behavior across a broad range of parameter values.
Performance variations are gradual and mainly occur at extreme settings,
suggesting that the method can be transferred to different problem
instances without requiring extensive parameter tuning. Regarding
the termination limits, no sensitivity analysis was performed, as
the same termination criterion was applied uniformly to all experiments
and all reference functions.

\subsection*{3.8 The proposed method in comparison with others}

The Table \ref{tab:Experimental-results-using} presents the results
of a comparative analysis of various optimization methods (BICCA\citep{key-32},
MLSHADESPA\citep{key-30}, SHADE\_ILS\citep{key-31}, Differential
Evolution (DE)\citep{diffe1,diffe2}, Genetic Algorithm (GA)\citep{genetic2,genetic3},
Whale Optimization Algorithm (WOA)\citep{WOA,WOA1}, Particle Swarm
Optimization (IPSO)\citep{pso_major,pso1}, PROPOSED) across a wide
range of test functions with dimensions of 25, 50, 100, and 150. Each
row corresponds to a test function, while the columns represent the
methods. The numerical values in each cell indicate the number of
objective function calls required to find the minimum, while the values
in parentheses show the success rate of each method in each case.
In the last row (TOTAL), the total sum of function calls for each
method is displayed, along with the average success rate. The best
methods should simultaneously exhibit a low number of function calls
(efficiency) and a high success rate (reliability). The analysis shows
that the Proposed method delivers strong and consistent performance.
Its overall success rate (0.85) is comparable to those of GA, MLSHADESPA,
SHADE\_ILS, IPSO, DE and WOA (around 0.90), and distinctly higher
than the BICCA method (0.73). This indicates that the Proposed method
remains dependable in locating the global minimum even when faced
with complex or high-dimensional search spaces. In terms of computational
cost, the Proposed method requires a total of 387,335 objective function
evaluations, which is substantially lower than most competing techniques.
This advantage appears consistently across the majority of tested
functions. For example, in the DifferentPowers function, the Proposed
method significantly outperforms GA across all dimensionalities: at
25 dimensions it uses 6,478 evaluations compared to 14,495 for GA
at 100 dimensions, 16,225 versus 28,413 and at 150 dimensions, 21,495
versus 33,569. Similar observations are made for the GriewankRosenbrock
function, where the Proposed method demonstrates clear efficiency
benefits: at 100 dimensions it requires 6,465 evaluations whereas
BICCA needs 20,462, and at 150 dimensions the gap widens further with
7,272 evaluations compared to 30,604 for BICCA. These differences
illustrate the method’s robustness and its ability to maintain low
computational demands in highly nonlinear and difficult optimization
landscapes.

In summary, the findings suggest that the Proposed method offers a
strong balance between reliability and computational efficiency. It
competes effectively with and in many cases surpasses widely used
optimization algorithms, while maintaining a consistently lower number
of objective function evaluations. Its stability across different
functions and dimensions confirms its applicability to a broad range
of optimization scenarios, making it a promising and efficient alternative
within the field of evolutionary and metaheuristic optimization.

{\scriptsize{}
\begin{table}[H]
{\scriptsize\caption{Experimental results using different optimization methods. Numbers
in cells represent sum function calls.\label{tab:Experimental-results-using}}
}{\scriptsize\par}

{\tiny{}%
\begin{tabular}{|c|c|c|c|c|c|c|c|c|}
\hline 
{\tiny\textbf{FUNCTION}} & {\tiny\textbf{BICCA}} & {\tiny\textbf{MLSHADESPA}} & {\tiny\textbf{SHADE\_ILS}} & {\tiny\textbf{DE}} & {\tiny\textbf{GA}} & {\tiny\textbf{WOA}} & {\tiny\textbf{IPSO}} & {\tiny\textbf{PROPOSED}}\tabularnewline
\hline 
\hline 
\begin{cellvarwidth}[t]
\centering
{\tiny ATTRACTIVE}{\tiny\par}

{\tiny SECTOR\_25}
\end{cellvarwidth} & 5130 & 950 & 452 & 4439 & 2208 & 2641 & 2120 & 1697\tabularnewline
\hline 
\begin{cellvarwidth}[t]
\centering
{\tiny ATTRACTIVE}{\tiny\par}

{\tiny SECTOR\_50}
\end{cellvarwidth} & 10097 & 994 & 558 & 18104 & 2230 & 5700 & 2167 & 1761\tabularnewline
\hline 
\begin{cellvarwidth}[t]
\centering
{\tiny ATTRACTIVE}{\tiny\par}

{\tiny SECTOR\_100}
\end{cellvarwidth} & 20178 & 989 & 748 & 15246 & 2231 & 5785 & 2179 & 1832\tabularnewline
\hline 
\begin{cellvarwidth}[t]
\centering
{\tiny ATTRACTIVE}{\tiny\par}

{\tiny SECTOR\_150}
\end{cellvarwidth} & 30259 & 1047 & 959 & 6646 & 2232 & 9248 & 2196 & 1867\tabularnewline
\hline 
{\tiny BUCHE RASTRIGIN\_25} & 5144(0.33) & 9420(0.90) & 2093(0.90) & 1466(0.90) & 12979(0.90) & 15048(0.93) & 12115(0.90) & 5893(0.90)\tabularnewline
\hline 
{\tiny BUCHE RASTRIGIN\_50} & 10345(0.03) & 18003(0.50) & 3440(0.50) & 1894(0.50) & 20711(0.50) & 58557(0.77) & 30866(0.50) & 12585(0.50)\tabularnewline
\hline 
{\tiny BUCHE RASTRIGIN\_100} & 20676(0.03) & 30652(0.53) & 5428(0.53) & 2020(0.53) & 29121(0.53) & 43001(0.97) & 39680(0.53) & 16490(0.53)\tabularnewline
\hline 
{\tiny BUCHE RASTRIGIN\_150} & 30894(0.03) & 47160(0.27) & 7663(0.27) & 2511(0.27) & 37696(0.27) & 54641 & 53060(0.27) & 23466(0.27)\tabularnewline
\hline 
{\tiny DISCUS\_25} & 5125 & 1365 & 536 & 4255 & 2656 & 3006 & 2452 & 1992\tabularnewline
\hline 
{\tiny DISCUS\_50} & 10101 & 1425 & 642 & 10297 & 2663 & 6310 & 2498 & 2060\tabularnewline
\hline 
{\tiny DISCUS\_100} & 20189 & 1402 & 826 & 8284 & 2631 & 5835 & 2523 & 2104\tabularnewline
\hline 
{\tiny DISCUS\_150} & 30265 & 1487 & 1042 & 8548 & 2620 & 8227 & 2548 & 2144\tabularnewline
\hline 
{\tiny DIFFERENTPOWERS\_25} & 5144 & 13007 & 2644 & 4786 & 14495 & 14921 & 13313 & 6478\tabularnewline
\hline 
{\tiny DIFFERENTPOWERS\_50} & 10389 & 20029 & 3860 & 14391 & 20539 & 35828 & 19839 & 11183\tabularnewline
\hline 
{\tiny DIFFERENTPOWERS\_100} & 20644 & 27859 & 5450 & 7355 & 28413 & 52081 & 28379 & 16225\tabularnewline
\hline 
{\tiny DIFFERENTPOWERS\_150} & 30877 & 36894 & 7059 & 6266 & 33569 & 93074 & 36287 & 21495\tabularnewline
\hline 
{\tiny ELLIPSOIDAL\_25} & 5139(0.87) & 4227 & 1117 & 4161 & 5955 & 7299 & 6375 & 3590\tabularnewline
\hline 
{\tiny ELLIPSOIDAL\_50} & 10247 & 9146 & 2178 & 16624 & 10892 & 19281 & 11641 & 6424\tabularnewline
\hline 
{\tiny ELLIPSOIDAL\_100} & 20492 & 18062 & 3966 & 12708 & 20202 & 38501 & 20736 & 11549\tabularnewline
\hline 
{\tiny ELLIPSOIDAL\_150} & 30708 & 26835 & 5993 & 21936 & 36236 & 63093 & 29414 & 16930\tabularnewline
\hline 
{\tiny GALLAGHER21\_25} & 5122(0.46) & 1304(0.90) & 503(0.90) & 4180(0.90) & 3346(0.90) & 9210(0.90) & 3605(0.90) & 2261(0.90)\tabularnewline
\hline 
{\tiny GALLAGHER21\_50} & 10119(0.03) & 1757(0.50) & 701(0.50) & 7938(0.50) & 3192(0.50) & 35580(0.50) & 8866(0.50) & 4503(0.50)\tabularnewline
\hline 
{\tiny GALLAGHER21\_100} & 20167 & 392(0.53) & 637(0.53) & 1323(0.53) & 1593(0.53) & 1950(0.53) & 5363(0.53) & 1756(0.53)\tabularnewline
\hline 
{\tiny GALLAGHER21\_150} & 30248 & 385(0.27) & 825(0.27) & 1313(0.27) & 1582(0.27) & 1738(0.27) & 2050(0.27) & 1662(0.27)\tabularnewline
\hline 
{\tiny GALLAGHER101\_25} & 5117(0.07) & 1270 & 501(0.90) & 3625(0.90) & 3340(0.90) & 7664(0.90) & 3473(0.90) & 2769(0.90)\tabularnewline
\hline 
{\tiny GALLAGHER101\_50} & 10114(0.03) & 1396 & 634(0.50) & 18470(0.50) & 7134(0.50) & 38817(0.50) & 8796(0.50) & 4890(0.50)\tabularnewline
\hline 
{\tiny GALLAGHER101\_100} & 20193(0.03) & 1868 & 901(0.53) & 14700(0.53) & 5794(0.53) & 39700(0.53) & 9257(0.53) & 5886(0.53)\tabularnewline
\hline 
{\tiny GALLAGHER101\_150} & 30269(0.03) & 1922 & 1127(0.27) & 24214(0.27) & 7210(0.27) & 36525(0.27) & 14076(0.27) & 8646(0.27)\tabularnewline
\hline 
{\tiny GRIEWANK \_25} & 5173(0.70) & 7828 & 1811 & 4123(0.97) & 9733 & 10166 & 9454 & 4084\tabularnewline
\hline 
{\tiny GRIEWANK \_50} & 10138 & 3434 & 1061 & 17524(0.93) & 5410 & 18966 & 9827 & 5039\tabularnewline
\hline 
{\tiny GRIEWANK \_100} & 20208 & 2825 & 1124 & 14809 & 4982 & 19318 & 10369 & 6460\tabularnewline
\hline 
{\tiny GRIEWANK \_150} & 30290 & 3035 & 1391 & 6335(0.97) & 5221 & 28823 & 10741 & 6542\tabularnewline
\hline 
{\tiny GRIEWANK\_ROSENBROCK\_25} & 5180 & 14086 & 3132 & 3238 & 17038 & 10630 & 9698 & 4466\tabularnewline
\hline 
{\tiny GRIEWANK\_ROSENBROCK\_50} & 10362 & 20021 & 4319 & 16379 & 23217 & 22912 & 11610 & 5325\tabularnewline
\hline 
{\tiny GRIEWANK\_ROSENBROCK\_100} & 20462 & 23913 & 4925 & 11375 & 31195 & 24543 & 13409 & 6465\tabularnewline
\hline 
{\tiny GRIEWANK\_ROSENBROCK\_150} & 30604 & 29813 & 6080 & 4446 & 37364 & 33948 & 15075 & 7272\tabularnewline
\hline 
{\tiny ROSENBROCK\_25} & 5163 & 12518 & 2793 & 3543 & 15493 & 13642 & 13642 & 5950\tabularnewline
\hline 
{\tiny ROSENBROCK\_50} & 10451 & 21195 & 4555 & 12085 & 24602 & 33038 & 22317 & 8963\tabularnewline
\hline 
{\tiny ROSENBROCK\_100} & 20785 & 35136 & 7151 & 6038 & 39496 & 48451 & 36400 & 15930\tabularnewline
\hline 
{\tiny ROSENBROCK\_150} & 31103 & 50850 & 10669 & 4203 & 53211 & 75425 & 50281 & 22135\tabularnewline
\hline 
{\tiny RARSTIGIN\_25} & 5139(0.36) & 7826(0.90) & 1767(0.90) & 1574(0.90) & 9581(0.90) & 15530(0.90) & 9826(0.90) & 4577(0.90)\tabularnewline
\hline 
{\tiny RARSTIGIN\_50} & 10208(0.03) & 10741(0.50) & 2091(0.50) & 1895(0.50) & 12272(0.50) & 41187(0.73) & 17354(0.50) & 7746(0.50)\tabularnewline
\hline 
{\tiny RARSTIGIN\_100} & 20358(0.03) & 11464(0.53) & 2338(0.53) & 1869(0.53) & 2134(0.53) & 27383(0.90) & 19347(0.53) & 9147(0.53)\tabularnewline
\hline 
{\tiny RARSTIGIN\_150} & 30561(0.03) & 14002(0.27) & 2942(0.27) & 2122(0.27) & 13990(0.27) & 32297(0.93) & 27682(0.27) & 11620(0.27)\tabularnewline
\hline 
{\tiny SPHERE\_25} & 5134 & 482 & 358 & 4131 & 1689 & 2206 & 1611 & 1481\tabularnewline
\hline 
{\tiny SPHERE\_50} & 10088 & 500 & 459 & 18098 & 1700 & 5111 & 1633 & 1509\tabularnewline
\hline 
{\tiny SPHERE\_100} & 20169 & 498 & 655 & 15241 & 1699 & 5107 & 1639 & 1524\tabularnewline
\hline 
{\tiny SPHERE\_150} & 30250 & 523 & 858 & 6639 & 1700 & 7347 & 1645 & 1535\tabularnewline
\hline 
{\tiny STEP ELLIPSOIDAL\_25} & 5114(0.70) & 375(0.90) & 313(0.90) & 1857(0.93) & 2069(0.90) & 1812(0.97) & 1846(0.90) & 1625(0.90)\tabularnewline
\hline 
{\tiny STEP ELLIPSOIDAL\_50} & 10086(0.03) & 375(0.50) & 391(0.50) & 6493(0.67) & 2469(0.50) & 2541(0.50) & 3993(0.50) & 2300(0.50)\tabularnewline
\hline 
{\tiny STEP ELLIPSOIDAL\_100} & 20167(0.03) & 377(0.53) & 541(0.53) & 5658(0.53) & 1681(0.53) & 2405(0.53) & 3946(0.53) & 2465(0.53)\tabularnewline
\hline 
{\tiny STEP ELLIPSOIDAL\_150} & 30248(0.03) & 383(0.27) & 695(0.27) & 5588(0.27) & 1673(0.27) & 2854(0.27) & 5065(0.27) & 3143(0.27)\tabularnewline
\hline 
{\tiny SHARP RIDGE\_25} & 5125 & 9281 & 2193 & 5153 & 11536 & 11398 & 11371 & 5104\tabularnewline
\hline 
{\tiny SHARP RIDGE\_50} & 10261 & 9843 & 2284 & 18677 & 11818 & 19405 & 12550 & 5226\tabularnewline
\hline 
{\tiny SHARP RIDGE\_100} & 20366 & 10190 & 2403 & 15159 & 11659 & 20507 & 13017 & 5995\tabularnewline
\hline 
{\tiny SHARP RIDGE\_150} & 30458 & 11205 & 2885 & 7476(0.97) & 11866 & 25983 & 13776 & 6481\tabularnewline
\hline 
{\tiny ZAKHAROV\_25} & 5120 & 4383 & 1177 & 1735 & 5756 & 9556 & 3449 & 2185\tabularnewline
\hline 
{\tiny ZAKHAROV\_50} & 10118 & 18043 & 3584 & 2371 & 15522 & 23884 & 6469 & 3027\tabularnewline
\hline 
{\tiny ZAKHAROV\_100} & 20211 & 45770 & 8470 & 2216 & 38359 & 29581 & 16562 & 5572\tabularnewline
\hline 
{\tiny ZAKHAROV\_150} & 30293 & 46497 & 9315 & 2503 & 36399 & 32379 & 21273 & 6304\tabularnewline
\hline 
 & 992785(0.73) & 708659(0.85) & 157213(0.85) & 478253(0.85) & 786004(0.85) & 1371596(0.90) & 782751(0.85) & 387335(0.85)\tabularnewline
\hline 
\end{tabular}}{\tiny\par}
\end{table}
}{\scriptsize\par}

Figure 6 illustrates the distributions of function evaluations for
the PROPOSED optimizer and the baseline algorithms. The Kruskal--Wallis
test indicates a statistically significant overall difference among
the methods (p \textless{} 2.2e\textminus 16), suggesting that at
least one optimizer differs from the others with respect to the distribution
of function-call counts. As shown in the figure, the PROPOSED method
is associated with lower median values and reduced dispersion in the
number of function evaluations compared with the baseline algorithms.
These differences in the distributions indicate that the PROPOSED
optimizer exhibits distinct computational behavior under the examined
experimental conditions. Overall, the statistical analysis and the
observed distributions suggest that the PROPOSED method requires fewer
function evaluations relative to the considered baselines. The results
describe systematic differences in computational cost among the optimizers,
without attributing the observed behavior to a single dominant factor.

\begin{figure}
\centering{}\includegraphics[scale=0.5]{methods}\caption{A statistical comparison of the proposed with other optimization methods.}
\end{figure}

To further investigate the scalability of the proposed optimization
framework, additional experiments were conducted in very high dimensions
and more specifically for dimensions 200,300,600,1100.

\begin{table}[H]
\caption{Experimental results using different very high-dimensional optimization
methods.}

\centering{}%
\begin{tabular}{|c|c|c|c|c|c|c|c|c|}
\hline 
{\tiny\textbf{FUNCTION}} & {\tiny\textbf{WOA}} & {\tiny\textbf{BICCA}} & {\tiny\textbf{SHADE\_ILS}} & {\tiny\textbf{MLSHADESPA}} & {\tiny\textbf{DE}} & {\tiny\textbf{PROPOSED}} & {\tiny\textbf{GA}} & {\tiny\textbf{IPSO}}\tabularnewline
\hline 
\hline 
{\tiny ELLIPSOIDAL\_200} & 49353 & 40915 & 9635 & 35675 & 27985 & 22123 & 31875 & 30853\tabularnewline
\hline 
{\tiny ELLIPSOIDAL\_300} & 51152 & 61333 & 9903 & 43473 & 43569 & 39328 & 52414 & 53185\tabularnewline
\hline 
{\tiny ELLIPSOIDAL\_600} & 80978 & 122625 & 21797 & 85276 & 73034 & 70020 & 78370 & 81032\tabularnewline
\hline 
{\tiny ELLIPSOIDAL\_1100} & 124378 & 223787 & 25607 & 123451 & 84912 & 80815 & 114632 & 113982\tabularnewline
\hline 
\end{tabular}
\end{table}
The results of these additional experiments are reported in Table
8. As the dimensionality increases, the proposed method is associated
with consistently lower numbers of objective function evaluations
compared with the baseline algorithms, indicating favorable scalability
characteristics in very high-dimensional optimization scenarios.

\begin{table}[H]
\caption{Experimental results using mean and standard deviation.}

\centering{}%
\begin{tabular}{|c|c|c|c|c|c|c|c|c|}
\hline 
{\tiny\textbf{FUNCTION}} & {\tiny\textbf{WOA MEAN }} & {\tiny\textbf{WOA STANDARD DEVIATION }} & {\tiny\textbf{BICCA MEAN }} & {\tiny\textbf{BICCASTANDARD DEVIATION }} & {\tiny\textbf{SHADE\_ILS MEAN STANDARD DEVIATION }} & {\tiny\textbf{SHADE\_ILS}} & {\tiny\textbf{MLSHADESPA MEAN }} & {\tiny\textbf{MLSHADESPASTANDARD DEVIATION }}\tabularnewline
\hline 
\hline 
{\tiny RASTRIGIN\_25} & 8.42 & 26.07 & 45.36 & 44.73 & 8.55 & 26.28 & 6.26 & 19.39\tabularnewline
\hline 
{\tiny RASTRIGIN\_50} & 27.79 & 47.44 & 86.06 & 23.96 & 52.16 & 54.44 & 44.27 & 45.56\tabularnewline
\hline 
{\tiny RASTRIGIN\_100} & 19.07 & 58.51 & 258.95 & 63.67 & 70.24 & 80.80 & 51.24 & 55.91\tabularnewline
\hline 
{\tiny RASTRIGIN\_150} & 12.17 & 46.32 & 312.08 & 68.05 & 128.81 & 90.00 & 102.81 & 64.24\tabularnewline
\hline 
{\tiny STEPELLIPSOIDAL\_25} & 1.18 & 6.48 & 2107.69 & 2049.63 & 387.30 & 1193.21 & 387.30 & 1193.21\tabularnewline
\hline 
{\tiny STEPELLIPSOIDAL\_50} & 0 & 0 & 6975.21 & 1772.99 & 3876.67 & 4034.57 & 3876.67 & 4034.57\tabularnewline
\hline 
{\tiny STEPELLIPSOIDAL\_100} & 0 & 0 & 10197.95 & 1242.43 & 5232.05 & 5786.36 & 5232.05 & 5786.36\tabularnewline
\hline 
{\tiny STEPELLIPSOIDAL\_150} & 0 & 0 & 9769.58 & 1069.59 & 10796.25 & 6776.41 & 10796.25 & 6776.41\tabularnewline
\hline 
\end{tabular}
\end{table}

\begin{table}[H]
\caption{Experimental results using mean and standard deviation.}

\centering{}%
\begin{tabular}{|c|c|c|c|c|c|c|c|c|}
\hline 
{\tiny\textbf{FUNCTION}} & {\tiny\textbf{DE MEAN}} & {\tiny\textbf{DE STANDARD DEVIATION }} & {\tiny\textbf{PROPOSED MEAN}} & {\tiny\textbf{PROPOSED STANDARD DEVIATION }} & {\tiny\textbf{GA MEAN}} & {\tiny\textbf{GA STANDARD DEVIATION }} & {\tiny\textbf{IPSO MEAN}} & {\tiny\textbf{IPSO STANDARD DEVIATION }}\tabularnewline
\hline 
\hline 
{\tiny RASTRIGIN\_25} & 1.49 & 4.60 & 2,88 & 8.8 & 5.33 & 16.87 & 0.36 & 1.34\tabularnewline
\hline 
{\tiny RASTRIGIN\_50} & 23.71 & 24.60 & 19.96 & 21.33 & 51.30 & 52.68 & 8.42 & 0.09\tabularnewline
\hline 
{\tiny RASTRIGIN\_100} & 55.12 & 61.30 & 28.92 & 32.80 & 62.51 & 68.96 & 15.02 & 16.65\tabularnewline
\hline 
{\tiny RASTRIGIN\_150} & 132.52 & 83.51 & 65.10 & 41.07 & 126.06 & 79.14 & 35.71 & 22.56\tabularnewline
\hline 
{\tiny STEPELLIPSOIDAL\_25} & 0 & 0 & 36.38 & 131.59 & 143.05 & 443.83 & 2.39 & 7.74\tabularnewline
\hline 
{\tiny STEPELLIPSOIDAL\_50} & 0 & 0 & 625.24 & 709.06 & 3076.57 & 3214.39 & 129.48 & 159.90\tabularnewline
\hline 
{\tiny STEPELLIPSOIDAL\_100} & 0 & 0 & 1124.16 & 1293.24 & 5163.19 & 5722.02 & 379.90 & 471.85\tabularnewline
\hline 
{\tiny STEPELLIPSOIDAL\_150} & 24.60 & 16.03 & 2791.42 & 1817.21 & 10676.31 & 6679.24 & 1007.36 & 777.37\tabularnewline
\hline 
\end{tabular}
\end{table}

In addition to the success rates, Tables 9 and 10 present the mean
value and standard deviation of the objective function across 30 independent
runs, in order to provide a more informative comparison between the
methods, especially in cases where the success rates are the same
or lower.

\subsection*{3.9 Practical problems}

To further examine the practical efficiency and scalability of the
proposed optimization algorithm, two real-world engineering design
problems were investigated: the GasCycle\citep{key-41} and the Tandem
Queueing System\citep{key-42}. These problems were selected because
they differ significantly in mathematical formulation and computational
complexity, providing a comprehensive framework for evaluating the
algorithm’s performance under diverse and realistic conditions.

Each problem was tested across multiple dimensional configurations,
ranging from 25 to 500 variables, in order to assess how the algorithm
behaves as the search space becomes more complex. For every configuration,
the execution time in seconds was recorded as the main performance
indicator. This experimental setup enables a direct comparison of
how computational efficiency changes with increasing dimensionality.
\begin{itemize}
\item \textbf{GasCycle Thermal Cycle}

Vars: $\bm{x}=[T_{1},T_{3},P_{1},P_{3}]^{\top}$. \quad{}$r=P_{3}/P_{1},\ \gamma=1.4.$
\[
\eta(\bm{x})=1-r^{-(\gamma-1)/\gamma}\,\frac{T_{1}}{T_{3}},\qquad\min_{\bm{x}}f(\bm{x})=-\eta(\bm{x}).
\]
Bounds: $300\!\le\!T_{1}\!\le\!1500,\;1200\!\le\!T_{3}\!\le\!2000,\;1\!\le\!P_{1},P_{3}\!\le\!20.$

Penalty: infeasible $\Rightarrow f=10^{20}$.

The GasCycle scenario presents a more computationally demanding optimization
problem, allowing a clearer assessment of algorithmic scalability
under increased complexity.

\begin{figure}

\includegraphics{GasCycle2.jpeg}\caption{Comparison of Function Calls Across Dimension (GasCycle)}

\end{figure}

For GasCycle, the proposed algorithm maintains a stable and competitive
number of function calls across all dimensions. Compared to methods
that show pronounced growth in evaluations at higher dimensions, the
proposed approach exhibits a more controlled increase, indicating
effective adaptation to the structure of the GasCycle problem. This
behavior suggests that the method can efficiently utilize function
evaluations without excessive computational overhead in large-scale
cases.

\begin{figure}
\includegraphics{GasCycle1.jpeg}

\caption{Comparison of Execution Time Across Dimension (GasCycle)}

\end{figure}

The execution time analysis for GasCycle aligns closely with the function
call results. The proposed algorithm achieves a balanced runtime profile,
with execution time increasing smoothly as dimensionality grows. In
contrast to approaches that suffer from substantial runtime escalation,
the proposed method maintains reasonable computational demands even
at high dimensions, highlighting its suitability for complex, large-scale
optimization tasks.
\item \textbf{Tandem Space Trajectory} (MGA-1DSM, EVEEJ + 2$\times$Saturn)

Vars\textbf{ ($D{=}18$):} $\bm{x}=[t_{0},T_{1},T_{2},T_{3},T_{4},T_{5A},T_{5B},s_{1},s_{2},s_{3},s_{4},s_{5A},s_{5B},r_{p},k_{A1},k_{A2},k_{B1},k_{B2}]^{\top}$.
\[
\begin{aligned} & 7000\le t_{0}\le10000,\\
 & 30\le T_{1}\le500,\;30\le T_{2}\le600,\;30\le T_{3}\le1200,\\
 & 30\le T_{4}\le1600,\;30\le T_{5A},T_{5B}\le2000,\\
 & 0\le s_{1..4},s_{5A},s_{5B},r_{p},k_{A1},k_{A2},k_{B1},k_{B2}\le1.
\end{aligned}
\]

Objective: 
\[
\min_{\bm{x}}\ \Delta V_{\text{tot}}=\Delta V_{\text{launch}}(T_{1})+\Delta V_{\text{legs}}(T_{1}{:}T_{4})+\Delta V_{A}+\Delta V_{B}+\Delta V_{\text{DSM}}(\bm{s},r_{p})-G_{\text{GA}}-G_{J}+P_{\text{hard}}+P_{\text{soft}},
\]
\medskip{}
 
\[
P_{\text{soft}}=\beta\max\!\Big\{0,\ (T_{1}{+}\cdots{+}T_{4}+\tfrac{1}{2}(T_{5A}{+}T_{5B}))-3500\Big\}.
\]
Notes: $\Delta V_{\text{launch}}$ decreases (log-like) in $T_{1}$
($\ge6$ km/s floor), leg/branch costs decrease with TOF.

\medskip{}

The figures corresponding to the Tandem scenario illustrate the behavior
of the evaluated algorithms in terms of function calls and execution
time as the problem dimension increases. As expected, higher dimensionality
leads to increased computational effort for all methods; however,
notable differences in scalability can be observed.

\begin{figure}

\includegraphics{Tandem2.jpeg}\caption{Comparison of Function Calls Across Dimension (Tandem)}

\end{figure}

In the Tandem case, the proposed algorithm demonstrates stable and
consistent behavior across all tested dimensions, maintaining a relatively
low number of function calls. Its performance remains competitive
with the most efficient approaches and is clearly more scalable than
methods such as GENETIC, BICCA, and IPSO, which exhibit a rapid increase
in function calls as dimensionality grows. The controlled growth observed
for the proposed method indicates effective search dynamics and an
appropriate balance between exploration and exploitation in large-scale
settings.

The execution time results further confirm these observations. The
proposed algorithm shows smooth and predictable scaling with increasing
problem dimension, avoiding the steep runtime growth observed in more
computationally demanding methods. Although execution time naturally
increases for larger dimensions, the rate of increase remains moderate,
suggesting that the internal computational cost of the proposed approach
is well managed and suitable for practical large-scale applications
in the Tandem scenario.

\begin{figure}
\centering{}\includegraphics{Tandem1.jpeg}\caption{Comparison of Execution Time Across Dimensions (Tandem)}
\end{figure}

Across both Tandem and GasCycle scenarios, the proposed algorithm
demonstrates consistent scalability in terms of both function evaluations
and execution time. Its stable behavior under increasing dimensionality
indicates that it represents a reliable and efficient alternative
for large-scale optimization problems, without incurring excessive
computational cost.
\end{itemize}

\section{Discussion\label{sec:Discussion}}

The experimental results provide important insights into how design
choices in sampling, parameter adaptation, selection pressure, and
local refinement collectively shape the behavior of DE in large-scale
optimization settings. Rather than acting independently, these components
interact in ways that significantly influence both convergence efficiency
and stability. When such interactions are structured and coordinated,
the algorithm exhibits more reliable behavior and reduced sensitivity
to stochastic effects.

A key observation concerns the role of sampling strategies. The use
of k-means clustering to guide population sampling consistently leads
to lower computational cost and reduced variance compared to uniform
sampling. This behavior can be attributed to the preservation of population
structure, which helps prevent redundant exploration and encourages
a more balanced coverage of promising regions of the search space.
Similar findings have been reported in previous studies on clustering-based
diversity preservation and structured population management in evolutionary
algorithms, particularly in high-dimensional settings where random
sampling often becomes inefficient. The tighter distributions observed
for configurations employing k-means indicate improved stability,
an essential property for large-scale optimization.

The differential weight mechanism further reinforces this structured
behavior. Among the examined strategies, the MIGRANT-based approach
demonstrates consistently lower function evaluation requirements while
maintaining identical success rates compared to static or random schemes.
This suggests that exploiting feedback from the evolving population
enables the algorithm to adapt its step sizes more effectively, leading
to more economical progress through the search space. Such observations
are in line with existing literature on adaptive parameter control
in DE, where learning-based or feedback-driven strategies are known
to outperform fixed parameter choices, especially as problem dimensionality
increases.

Selection pressure also plays a critical role in shaping the overall
dynamics of the search. Tournament selection, when combined with adaptive
differential weighting, provides a controlled bias toward high-quality
solutions without excessively reducing population diversity. In contrast,
random selection introduces unnecessary stochasticity that can disrupt
the learning process, particularly when coupled with adaptive mechanisms
that rely on informative population feedback. The results indicate
that even moderate selection pressure can significantly improve the
reliability of adaptive strategies by ensuring that useful information
is retained and exploited across generations.

The influence of local search frequency highlights the importance
of moderation in hybrid metaheuristics. While local refinement can
enhance solution quality, excessive application increases computational
overhead and may interfere with global exploration. The experimental
findings show that a very low local search rate achieves the best
trade-off between exploitation and efficiency, supporting the view
that local search should act as a complementary mechanism rather than
a dominant driver of the optimization process. This observation aligns
with prior work on memetic and hybrid evolutionary algorithms, where
controlled and infrequent local search is often more effective than
aggressive refinement.

When considered collectively, these observations explain the strong
and consistent performance of the proposed method relative to classical
and modern optimizers. Rather than relying on aggressive parameter
settings or complex hybridization, the method benefits from a balanced
integration of structured exploration and adaptive exploitation. The
comparative results indicate that the proposed approach achieves competitive
or superior performance with lower variability, suggesting improved
robustness across diverse problem landscapes. Importantly, where statistical
tests do not indicate significant pairwise differences, the observed
performance trends are interpreted as consistent empirical behavior
rather than strict dominance.

Overall, the discussion highlights that the performance gains achieved
by the proposed framework stem from thoughtful algorithmic structure
rather than brute-force complexity. By guiding the search through
informed sampling, adaptive weighting, and controlled selection pressure,
the algorithm avoids both premature convergence and inefficient exploration.
These findings reinforce a broader principle in large-scale evolutionary
optimization: intelligent structure and feedback-driven adaptation
are often more effective than increased randomness or parameter proliferation.

As is common in population-based metaheuristic methods, no formal
convergence or complexity guarantees are provided for the proposed
approach. Instead, scalability and computational behavior are assessed
empirically through extensive experimental evaluation. The inclusion
of large-scale benchmark functions and real-world engineering problems,
such as the Tandem and GasCycle scenarios, enables a practical assessment
of convergence trends and computational cost under increasing dimensionality.
The observed behavior in terms of function evaluations and execution
time indicates controlled scalability in large-scale optimization
settings.

\section{Conclusions\label{sec:Conclusions}}

This work explored large-scale optimization through a systematically
enhanced version of the DE algorithm. The improvements introduced
in this study were designed to address two persistent challenges in
high-dimensional optimization: efficiency and stability. Throughout
the experimental analysis, several key components proved crucial to
achieving these goals. A central contribution is the MIGRANT differential
weight mechanism, which consistently outperformed both the classic
NUMBER and RANDOM schemes. Across a wide variety of benchmark functions,
MIGRANT(T) required significantly fewer objective function evaluations
while maintaining identical success rates. This demonstrates that
an adaptive weight strategy can guide the search more intelligently,
reducing unnecessary evaluations and offering clear performance advantages
in complex landscapes. Equally important was the impact of the sampling
strategy. The results showed that k-means sampling (K) provides a
strong structural advantage compared to uniform sampling. Configurations
using MIGRANT(K) repeatedly achieved the lowest evaluation counts
and exhibited far smaller variance. Pairwise statistical tests confirmed
these differences, with several comparisons reaching high or very
high levels of significance. This indicates that exploiting cluster
information during sampling can greatly improve the quality and diversity
of candidate solutions. The study also highlighted the role of the
selection mechanism. Tournament selection consistently strengthened
the algorithm’s performance, enabling MIGRANT(T) to outperform all
Random-based variants. This confirms that introducing even a light
degree of selective pressure yields more reliable search dynamics,
while fully random selection tends to increase noise and computational
cost. Another important outcome relates to the local search rate.
Although local search can refine promising candidates, the experiments
showed that applying it too frequently becomes counterproductive.
The lowest tested rate (0.005) offered the best trade-off, achieving
lower computational cost and greater stability. In contrast, higher
rates (0.03 and 0.05) significantly increased function evaluations
without improving success rates. This emphasizes the need for careful
calibration of exploitation mechanisms in high-dimensional settings.
Finally, when compared to widely used algorithms such as Genetic Algorithms,
BICCA, LSHADE-SPA, SHADE-ILS, IPSO, WOA, and DE, the proposed method
consistently delivered superior performance. Taken together, these
findings highlight the effectiveness of combining structured sampling,
adaptive weighting, selective pressure, and controlled local search
within DE. The synergy of these components results in an optimizer
that is not only faster but also remarkably stable across different
problem types and dimensions.

A promising direction for future research is to explore how the proposed
framework could be integrated with other well-established metaheuristic
algorithms. Such a hybridization could leverage the strengths of different
search strategies and potentially lead to more effective optimization
performance. In addition, another interesting avenue is the incorporation
of learning mechanisms such as reinforcement learning or adaptive
parameter-learning techniques so that the algorithm can dynamically
adjust its strategies and parameters based on the characteristics
of the search landscape. Such a self-adaptive system could further
enhance the stability, robustness, and overall efficiency of the optimization
process.

Overall, this study demonstrates that carefully designed modifications
to DE can lead to substantial performance gains, and it sets the foundation
for developing even more powerful and general-purpose optimization
algorithms.

\vspace{6pt}


\authorcontributions{G.K., V.C. and I.G.T. conceived of the idea and the methodology,
and G.K. and V.C. implemented the corresponding software. G.K. conducted
the experiments, employing objective functions as test cases, and
provided the comparative experiments. I.G.T. performed the necessary
statistical tests. All authors have read and agreed to the published
version of the manuscript.}

\funding{This research has been financed by the European Union: Next Generation
EU through the Program Greece 2.0 National Recovery and Resilience
Plan, under the call RESEARCH--CREATE--INNOVATE, project name “iCREW:
Intelligent small craft simulator for advanced crew training using
Virtual Reality techniques” (project code: TAEDK-06195).}

\institutionalreview{Not Applicable.}

\informedconsent{Not Applicable.}

\dataavailability{The original contributions presented in this study are included in
the article. Further inquiries can be directed to the corresponding
author.}

\conflictsofinterest{The authors declare no conflicts of interest.}

\appendixtitles{no}

\begin{adjustwidth}{-\extralength}{0cm}{}


\reftitle{References}
\begin{thebibliography}{999}
\bibitem[Author1(year)]{go_math1} AutCarrizosa, E., Molero-Río, C.,
\& Romero Morales, D. (2021). Mathematical optimization in classification
and regression trees. Top, 29(1), 5-33.

\bibitem{go_math3}Legat, B., Dowson, O., Garcia, J. D., \& Lubin,
M. (2022). MathOptInterface: a data structure for mathematical optimization
problems. INFORMS Journal on Computing, 34(2), 672-689.

\bibitem{go_physics2}Su, H., Zhao, D., Heidari, A. A., Liu, L., Zhang,
X., Mafarja, M., \& Chen, H. (2023). RIME: A physics-based optimization.
Neurocomputing, 532, 183-214.

\bibitem{go_physics3}Stilck França, D., \& Garcia-Patron, R. (2021).
Limitations of optimization algorithms on noisy quantum devices. Nature
Physics, 17(11), 1221-1227.

\bibitem{go_chem1}Zhang, J., \& Glezakou, V. A. (2021). Global optimization
of chemical cluster structures: Methods, applications, and challenges.
International Journal of Quantum Chemistry, 121(7), e26553.

\bibitem{go_chem2}Hu, Y., Zang, Z., Chen, D., Ma, X., Liang, Y.,
You, W., \& Zhang, Z. (2022). Optimization and evaluation of SO2 emissions
based on WRF-Chem and 3DVAR data assimilation. Remote Sensing, 14(1),
220.

\bibitem{go_med2}Kaur, P., \& Singh, R. K. (2023). A review on optimization
techniques for medical image analysis. Concurrency and Computation:
Practice and Experience, 35(1), e7443.

\bibitem{medicine}Houssein, E. H., Hosney, M. E., Mohamed, W. M.,
Ali, A. A., \& Younis, E. M. (2023). Fuzzy-based hunger games search
algorithm for global optimization and feature selection using medical
data. Neural Computing and Applications, 35(7), 5251-5275.

\bibitem{go_bio1}Wang, L., Cao, Q., Zhang, Z., Mirjalili, S., \&
Zhao, W. (2022). Artificial rabbits optimization: A new bio-inspired
meta-heuristic algorithm for solving engineering optimization problems.
Engineering Applications of Artificial Intelligence, 114, 105082.

\bibitem{go_bio2}Li, X. (2024). Optimization of crop tissue culture
technology and its impact on biomolecular characteristics. Molecular
\& Cellular Biomechanics, 21(2).

\bibitem{go_agri1}Xiao, L., Wang, G., Wang, E., Liu, S., Chang, J.,
Zhang, P., ... \& Luo, Z. (2024). Spatiotemporal co-optimization of
agricultural management practices towards climate-smart crop production.
Nature Food, 5(1), 59-71.

\bibitem{go_agri2}Akintuyi, O. B. (2024). Adaptive AI in precision
agriculture: a review: investigating the use of self-learning algorithms
in optimizing farm operations based on real-time data. Research Journal
of Multidisciplinary Studies, 7(02), 016-030.

\bibitem{go_econ1}Hassan, M. H., Kamel, S., Jurado, F., \& Desideri,
U. (2024). Global optimization of economic load dispatch in large
scale power systems using an enhanced social network search algorithm.
International Journal of Electrical Power \& Energy Systems, 156,
109719.

\bibitem{go_econ2}Alirahmi, S. M., Mousavi, S. B., Razmi, A. R.,
\& Ahmadi, P. (2021). A comprehensive techno-economic analysis and
multi-criteria optimization of a compressed air energy storage (CAES)
hybridized with solar and desalination units. Energy Conversion and
Management, 236, 114053.

\bibitem{key-23}Tang, K., Li, X., Suganthan, P. N., Yang, Z., \&
Weise, T. (2007). Benchmark functions for the CEC’2010 special session
and competition on large-scale global optimization. Nature inspired
computation and applications laboratory, USTC, China, 24, 1-18.

\bibitem{key-24}Li, X., Tang, K., Omidvar, M. N., Yang, Z., Qin,
K., \& China, H. (2013). Benchmark functions for the CEC 2013 special
session and competition on large-scale global optimization. gene,
7(33), 8.

\bibitem{key-25}Molina, D., \& Herrera, F. (2015, May). Iterative
hybridization of DE with local search for the CEC'2015 special session
on large scale global optimization. In 2015 IEEE congress on evolutionary
computation (CEC) (pp. 1974-1978). IEEE.

\bibitem{key-6}Li, P., Hao, J., Tang, H., Fu, X., Zhen, Y., \& Tang,
K. (2024). Bridging evolutionary algorithms and reinforcement learning:
A comprehensive survey on hybrid algorithms. IEEE Transactions on
evolutionary computation.

\bibitem{key-7}Yuan, S., Song, K., Chen, J., Tan, X., Li, D., \&
Yang, D. (2025, April). Evoagent: Towards automatic multi-agent generation
via evolutionary algorithms. In Proceedings of the 2025 Conference
of the Nations of the Americas Chapter of the Association for Computational
Linguistics: Human Language Technologies (Volume 1: Long Papers) (pp.
6192-6217).

\bibitem{diffe1}Deng, W., Shang, S., Cai, X., Zhao, H., Song, Y.,
\& Xu, J. (2021). An improved differential evolution algorithm and
its application in optimization problem. Soft Computing, 25, 5277-5298.

\bibitem{diffe2}Pant, M., Zaheer, H., Garcia-Hernandez, L., \& Abraham,
A. (2020). Differential Evolution: A review of more than two decades
of research. Engineering Applications of Artificial Intelligence,
90, 103479.

\bibitem{genetic2}Sohail, A. (2023). Genetic algorithms in the fields
of artificial intelligence and data sciences. Annals of Data Science,
10(4), 1007-1018.

\bibitem{genetic3}Charilogis, V., Tsoulos, I. G., \& Stavrou, V.
N. (2023). An Intelligent Technique for Initial Distribution of Genetic
Algorithms. Axioms, 12(10), 980.

\bibitem{key-8}Lange, R., Tian, Y., \& Tang, Y. (2024, July). Large
language models as evolution strategies. In Proceedings of the Genetic
and Evolutionary Computation Conference Companion (pp. 579-582).

\bibitem{key-11}Gašparovič, M., Jungová, P., Tomášik, J., Mriňáková,
B., Hirjak, D., Timková, S., ... \& Thurzo, A. (2024). Evolving strategies
and materials for scaffold development in regenerative dentistry.
Applied Sciences, 14(6), 2270.

\bibitem{key-12}Cicirello, V. A. (2024). Evolutionary computation:
Theories, techniques, and applications. Applied Sciences, 14(6), 2542.

\bibitem{key-13}Porto, V. W. (2018). Evolutionary programming. In
Evolutionary Computation 1 (pp. 127-140). CRC Press.

\bibitem{key-14}Cheng, S., Wang, X., Zhang, M., Lei, X., Lu, H.,
\& Shi, Y. (2024). Solving multimodal optimization problems by a knowledge-driven
brain storm optimization algorithm. Applied Soft Computing, 150, 111105.

\bibitem{key-15}Huang, Y. B., Wang, Z. J., Zhang, Y. H., Wang, Y.
G., Kwong, S., \& Zhang, J. (2024). Wireless sensor networks-based
adaptive differential evolution for multimodal optimization problems.
Applied Soft Computing, 158, 111541.

\bibitem{key-27}Kong, L. S., Jasser, M. B., Ajibade, S. S. M., \&
Mohamed, A. W. (2024). A systematic review on software reliability
prediction via swarm intelligence algorithms. Journal of King Saud
University-Computer and Information Sciences, 36(7), 102132.

\bibitem{key-28}Nguyen, L. V. (2024). Swarm intelligence-based multi-robotics:
a comprehensive review. AppliedMath, 4(4), 1192-1210.

\bibitem{pso_major}Shami, T. M., El-Saleh, A. A., Alswaitti, M.,
Al-Tashi, Q., Summakieh, M. A., \& Mirjalili, S. (2022). Particle
swarm optimization: A comprehensive survey. Ieee Access, 10, 10031-10061.

\bibitem{pso1}Gad, A. G. (2022). Particle swarm optimization algorithm
and its applications: a systematic review. Archives of computational
methods in engineering, 29(5), 2531-2561. 

\bibitem{aco1}Rokbani, N., Kumar, R., Abraham, A., Alimi, A. M.,
Long, H. V., Priyadarshini, I., \& Son, L. H. (2021). Bi-heuristic
ant colony optimization-based approaches for traveling salesman problem.
Soft Computing, 25, 3775-3794.

\bibitem{aco2}Wu, L., Huang, X., Cui, J., Liu, C., \& Xiao, W. (2023).
Modified adaptive ant colony optimization algorithm and its application
for solving path planning of mobile robot. Expert Systems with Applications,
215, 119410.

\bibitem{key-18}Ibrahim, A. O., Elfadel, E. M. E., Hashem, I. A.
T., Syed, H. J., Ismail, M. A., Osman, A. H., \& Ahmed, A. (2025).
The Artificial Bee Colony Algorithm: A Comprehensive Survey of Variants,
Modifications, Applications, Developments, and Opportunities. Archives
of Computational Methods in Engineering, 1-35.

\bibitem{key-17}Devadason, J. R., Hepsiba, P. S., \& Solomon, D.
G. (2024). Case studies on the applications of the artificial bee
colony algorithm. Sādhanā, 49(2), 152.

\bibitem{key-19}Singh, A. K., \& Kumar, A. (2025). Multi-objective:
hybrid particle swarm optimization with firefly algorithm for feature
selection with Leaky ReLU. Discover Artificial Intelligence, 5(1),
192.

\bibitem{key-20}Yang, X. S., \& Slowik, A. (2020). Firefly algorithm.
In Swarm intelligence algorithms (pp. 163-174). CRC Press.

\bibitem{key-50}Umar, S. U., Rashid, T. A., Ahmed, A. M., Hassan,
B. A., \& Baker, M. R. (2024). Modified Bat Algorithm: a newly proposed
approach for solving complex and real-world problems. arXiv preprint
arXiv:2407.15318.

\bibitem{key-51}Dao, T. K., \& Nguyen, T. T. (2025). A review of
the bat algorithm and its varieties for industrial applications. Journal
of Intelligent Manufacturing, 36(8), 5327-5349.

\bibitem{key-1}Storn, R., \& Price, K. (1995). Differential evolution-a
simple and efficient adaptive scheme for global optimization over
continuous spaces. International computer science institute.

\bibitem{key-2-1}Storn, R., \& Price, K. (1997). Differential evolution--a
simple and efficient heuristic for global optimization over continuous
spaces. Journal of global optimization, 11, 341-359.

\bibitem{key-1-1}Bai, Y., Wu, X., \& Xia, A. (2021). An enhanced
multi‐objective differential evolution algorithm for dynamic environmental
economic dispatch of power system with wind power. Energy Science
\& Engineering, 9(3), 316-329.

\bibitem{key-2}Penenko, A. V., Konopleva, V. S., \& Penenko, V. V.
(2022, May). Inverse modeling of atmospheric chemistry with a differential
evolution solver: Inverse problem and Data assimilation. In IOP Conference
Series: Earth and Environmental Science (Vol. 1023, No. 1, p. 012015).
IOP Publishing.

\bibitem{key-3}Babanezhad, M., Behroyan, I., Nakhjiri, A. T., Marjani,
A., Rezakazemi, M., \& Shirazian, S. (2020). High-performance hybrid
modeling chemical reactors using differential evolution based fuzzy
inference system. Scientific Reports, 10(1), 21304.

\bibitem{key-4}Liu, L., Zhao, D., Yu, F., Heidari, A. A., Ru, J.,
Chen, H., ... \& Pan, Z. (2021). Performance optimization of differential
evolution with slime mould algorithm for multilevel breast cancer
image segmentation. Computers in Biology and Medicine, 138, 104910

\bibitem{large_co}Yao, X., \& Chong, S. Y. (2025). Cooperative Coevolution
for Large-Scale Optimization. In Coevolutionary Computation and Its
Applications (pp. 199-270). Singapore: Springer Nature Singapore.

\bibitem{large_pso}McGovarin, Z., Engelbrecht, A. P., \& Ombuki-Berman,
B. M. (2024). Stochastic Grouping and Subspace-Based Initialization
in Decomposition and Merging Cooperative Particle Swarm Optimization
for Large-Scale Optimization Problems. In Canadian AI.

\bibitem{large_memetic}Yue, X., Liao, Y., Peng, H., Kang, L., \&
Zeng, Y. (2025). A high-dimensional feature selection algorithm via
fast dimensionality reduction and multi-objective differential evolution.
Swarm and Evolutionary Computation, 94, 101899.

\bibitem{key-40}Chen, M., \& Tan, Y. (2023). SF-FWA: A self-adaptive
fast fireworks algorithm for effective large-scale optimization. Swarm
and Evolutionary Computation, 80, 101314.

\bibitem{key-60}Sun, Y., \& Cao, H. (2024). An agent-assisted heterogeneous
learning swarm optimizer for large-scale optimization. Swarm and Evolutionary
Computation, 89, 101627.

\bibitem{key-61}Wang, X., Wang, F., He, Q., \& Guo, Y. (2024). A
multi-swarm optimizer with a reinforcement learning mechanism for
large-scale optimization. Swarm and Evolutionary Computation, 86,
101486.

\bibitem{key-43}Li, J. Y., Zhan, Z. H., Tan, K. C., \& Zhang, J.
(2022). Dual differential grouping: A more general decomposition method
for large-scale optimization. IEEE Transactions on Cybernetics, 53(6),
3624-3638.

\bibitem{key-71}Li, J. Y., Du, K. J., Zhan, Z. H., Wang, H., \& Zhang,
J. (2022). Distributed differential evolution with adaptive resource
allocation. IEEE transactions on cybernetics, 53(5), 2791-2804.

\bibitem[Author1(year)]{de_char} Charilogis, V., Tsoulos, I. G.,
Tzallas, A., \& Karvounis, E. (2022). Modifications for the differential
evolution algorithm. Symmetry, 14(3), 447.

\bibitem{de_migrant}Cheng, J., Zhang, G., \& Neri, F. (2013). Enhancing
distributed differential evolution with multicultural migration for
global numerical optimization. Information Sciences, 247, 72-93.

\bibitem{powell}Powell, M. J. D. (1989). A tolerant algorithm for
linearly constrained optimization calculations. Mathematical Programming,
45, 547-566.

\bibitem{key-2-2}Abualigah, L., Diabat, A., Mirjalili, S., Abd Elaziz,
M., \& Gandomi, A. H. (2021). The arithmetic optimization algorithm.
Computer methods in applied mechanics and engineering, 376, 113609.

\bibitem{key-4-1}Sulaiman, A. T., Bello-Salau, H., Onumanyi, A. J.,
Mu’azu, M. B., Adedokun, E. A., Salawudeen, A. T., \& Adekale, A.
D. (2024). A particle swarm and smell agent-based hybrid algorithm
for enhanced optimization. Algorithms, 17(2), 53.

\bibitem{kmeans1}Li, Y., \& Wu, H. (2012). A clustering method based
on K-means algorithm. Physics Procedia, 25, 1104-1109.

\bibitem{kmeans2}Arora, P., \& Varshney, S. (2016). Analysis of k-means
and k-medoids algorithm for big data. Procedia Computer Science, 78,
507-512.

\bibitem{key-22}Price, K. V., Storn, R. M., \& Lampinen, J. A. (2005).
Differential evolution: a practical approach to global optimization.
Berlin, Heidelberg: Springer Berlin Heidelberg.

\bibitem{testfunc1}Ali, M. M., \& Kaelo, P. (2008). Improved particle
swarm algorithms for global optimization. Applied mathematics and
computation, 196(2), 578-593.

\bibitem{testfunc2}Koyuncu, H., \& Ceylan, R. (2019). A PSO based
approach: Scout particle swarm algorithm for continuous global optimization
problems. Journal of Computational Design and Engineering, 6(2), 129-142.

\bibitem{testfunc2-1}Siarry, P., Berthiau, G., Durdin, F., \& Haussy,
J. (1997). Enhanced simulated annealing for globally minimizing functions
of many-continuous variables. ACM Transactions on Mathematical Software
(TOMS), 23(2), 209-228.

\bibitem{testfunc4}LaTorre, A., Molina, D., Osaba, E., Poyatos, J.,
Del Ser, J., \& Herrera, F. (2021). A prescription of methodological
guidelines for comparing bio-inspired optimization algorithms. Swarm
and Evolutionary Computation, 67, 100973.

\bibitem[(2025)]{OPTIMUS}Tsoulos, I.G., Charilogis, V., Kyrou, G.,
Stavrou, V.N. \& Tzallas,A. (2025). OPTIMUS: A Multidimensional Global
Optimization Package. Journal of Open Source Software, 10(108), 7584.
Doi: https://doi.org/10.21105/joss.07584

\bibitem{key-32}Ge, H., Zhao, M., Hou, Y., Kai, Z., Sun, L., Tan,
G., ... \& Chen, C. P. (2020). Bi-space interactive cooperative coevolutionary
algorithm for large scale black-box optimization. Applied Soft Computing,
97, 106798.

\bibitem{key-30}Hadi, A. A., Mohamed, A. W., \& Jambi, K. M. (2019).
LSHADE-SPA memetic framework for solving large-scale optimization
problems. Complex \& Intelligent Systems, 5(1), 25-40.

\bibitem{key-31}Molina, D., LaTorre, A., \& Herrera, F. (2018, July).
SHADE with iterative local search for large-scale global optimization.
In 2018 IEEE congress on evolutionary computation (CEC) (pp. 1-8).
IEEE.

\bibitem[(2023)]{WOA}Nadimi-Shahraki, M. H., Zamani, H., Asghari
Varzaneh, Z., \& Mirjalili, S. (2023). A systematic review of the
whale optimization algorithm: theoretical foundation, improvements,
and hybridizations. Archives of Computational Methods in Engineering,
30(7), 4113-4159.

\bibitem{WOA1}Brodzicki, A., Piekarski, M., \& Jaworek-Korjakowska,
J. (2021). The whale optimization algorithm approach for deep neural
networks. Sensors, 21(23), 8003

\bibitem{MacQueen}MacQueen, J. (1967, June). Some methods for classification
and analysis of multivariate observations. In Proceedings of the fifth
Berkeley symposium on mathematical statistics and probability (Vol.
1, No. 14, pp. 281-297).

\bibitem{key-41}Luo, B., Su, X., Zhang, S., Yan, P., Liu, J., \&
Li, R. (2025). Analysis of a novel gas cycle cooler with large temperature
glide for space cooling. Energy, 326, 136294.

\bibitem{key-42}Keerthika, R., Niranjan, S. P., \& Komala Durga,
B. (2025). A Survey on the tandem queueing models. Scope, 14, 134-148.

\end{thebibliography}
%%%%%%%%%%%%%%%%%%%%%%%%%%%%%%%%%%%%%%%%%%
%% for journal Sci
%\reviewreports{\\
%Reviewer 1 comments and authors' response\\
%Reviewer 2 comments and authors' response\\
%Reviewer 3 comments and authors' response
%}
%%%%%%%%%%%%%%%%%%%%%%%%%%%%%%%%%%%%%%%%%%

\PublishersNote{}

\end{adjustwidth}{}
\end{document}
